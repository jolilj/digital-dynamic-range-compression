\documentclass[]{article}
\usepackage{natbib}
\usepackage{tikz}
\usetikzlibrary{shapes,arrows}
\usepackage{textcomp}

\begin{document}

\title{Digital dynamic range compression algorithm design att Klevgränd Production AB}
\author{J. Lilja, O. Karlsson}
\date{2014-02-01}
\maketitle

\begin{abstract}
In this study/paper/report, we investigate...
\end{abstract}

\clearpage
\section{Introduction}
\section{Background}
\subsection{Dynamical range compression}
The dynamic range compressor is a complex nonlinear time dependent system where the the amplitude peaks of a signal is suppressed, reducing its dynamic range. The complexity of the system together with the vast set of design choices makes it difficult to draw a generic block diagram of the process\cite{giannoullis}  and thus we will, in the analysis, focus on a digital implementation corresponding to ideal components of it's analogue counterpart. 

In figure \ref{fig:xy-graph} the eligible behaviour of a simple compression process is illustrated. The x- and y-axis represents the logarithmic amplitude of input and output signal respectively. As the input signal rise above the threshold level, compression is applied. Logarithmic scale is used since sound amplitude is ,by standard, expressed in the logarithmic unit decibel.

\begin{figure}[ht]
\centering
\begin{tikzpicture}[scale=0.9,baseline={(0,0)}]

% horizontal axis
\draw[->] (0,0) -- (5.5,0) node[anchor=north] {$X$};
% ranges
\draw	(0.9,2.2) node{{\scriptsize Threshold}};

% Knee label
\draw	(2.2,3.2)
node{{\scriptsize Knee}};
\draw[<-](2,2.1) -- (2.2,3);

% vertical axis
\draw[->] (0,0) -- (0,4) node[anchor=east] {$Y$};
% nominal speed
\draw[dotted] (0,2) -- (5,2);

% logy = logx
\draw[thick] (0,0) -- (2,2) -- (5,3);

\end{tikzpicture}

\caption{Output signal plotted against input signal. When the input signal rise above the threshold level, compression is applied.} 
\label{fig:xy-graph}
\end{figure}

\subsection{Gain-reduction}
We begin our derivation of a block diagram for a generic compressor with the gain reduction step. This is where the signals amplitude is modified and is achieved with a \emph{Voltage Controlled Amplifier} (VCA).

In the analog history of compressor design, various different circuit types have been implemented for this task (FET, optical circuits, tube amplifiers etc.), each with it's own characteristic. In this paper we will use the term VCA as a generic term for any controllable amplifier. This is not to be confused with modern solid state compressors that use integrated VCA circuits for gain control, sometimes called \emph{VCA compressors}.


\begin{figure}[ht]
\centering
\tikzset{%
  amp/.style	= {draw, regular polygon, regular polygon sides=3,
	shape border rotate=-90, node distance = 25mm},
  input/.style	= {coordinate}, % Input
  output/.style	= {coordinate} % Output
}

\begin{tikzpicture}[auto, thick, node distance=2cm, >=triangle 45]
\draw
	% Drawing the blocks of first filter :
	node at (0,0)[right=-3mm]{\Large \textopenbullet}
	node [input, name=input1] {} 

	node [amp, right of=input1, name=vca1, ] {VCA}
	node [output, right of=vca1, name=output1, xshift=10mm]{}
	
	node [below of=vca1, name=cv1, yshift=-5mm]{\Large \textopenbullet}
	;
    % Joining blocks. 
    % Commands \draw with options like [->] must be written individually
	\draw[->](input1) -- node {\it{in}}(vca1);
 	\draw[->](vca1) -- node {\it{out}} (output1);
	\draw[->](cv1) -- node [label={[xshift=5mm, yshift=-5mm]\it{cv}}]{} (vca1);

\end{tikzpicture}
\caption{Block diagram of gain reduction step} 
\label{fig:vca-generic-blockdiagram}
\end{figure}





\subsection{Feed-forward, feedback}
Now, we stand at an early cross-road for how to compute the control voltage. We can choose to have an open loop, feed-forward design, or a closed loop, feedback design.

\begin{figure}[ht]
\centering
\tikzset{%
  amp/.style	= {draw, regular polygon, regular polygon sides=3,
	shape border rotate=-90, node distance = 70mm},
  block/.style    = {draw, thick, rectangle, minimum height = 3em,
    minimum width = 3em},
  input/.style	= {coordinate}, % Input
  output/.style	= {coordinate} % Output
}

\begin{tikzpicture}[auto, thick, node distance=2cm, >=triangle 45]
\draw
	% Drawing the blocks of first filter :
	node at (0,0)[right=-3mm]{\Large \textopenbullet}
	node [input, name=input1] {} 

	node at (1.5,0){\textbullet}

	node [amp, right of=input1, name=vca1, ] {VCA}
	node [block, below left of=vca1, xshift=-15mm, yshift=-5mm, name=gain1]{Gain computer}
	node [output, right of=vca1, name=output1, xshift=10mm]{}
	;
    % Joining blocks. 
    % Commands \draw with options like [->] must be written individually
	\draw[->](input1) -- node [label={[xshift=-25mm]\it{in}}]{}(vca1);
 	\draw[->](vca1) -- node {\it{out}} (output1);
	\draw[->](1.5,0) |- node {} (gain1);
	\draw[->](gain1) -| node [label={[xshift=5mm]\it{cv}}]{} (vca1);

\end{tikzpicture}
\caption{Block diagram of feed forward design} 
\label{fig:feedforward-blockdiagram}
\end{figure}

\begin{figure}[ht]
\centering
\tikzset{%
  amp/.style	= {draw, regular polygon, regular polygon sides=3,
	shape border rotate=-90, node distance = 20mm},
  block/.style    = {draw, thick, rectangle, minimum height = 3em,
    minimum width = 3em},
  input/.style	= {coordinate}, % Input
  output/.style	= {coordinate, node distance=70mm} % Output
}

\begin{tikzpicture}[auto, thick, node distance=2cm, >=triangle 45]
\draw
	% Drawing the blocks of first filter :
	node at (0,0)[right=-3mm]{\Large \textopenbullet}
	node [input, name=input1] {} 
	node [amp, right of=input1, name=vca1, ] {VCA}
	
	node at (8,0){\textbullet}

	node [block, below right of=vca1, xshift=15mm, yshift=-5mm, name=gain1]{Gain computer}
	node [output, right of=vca1, name=output1, xshift=10mm]{}
	;
    % Joining blocks. 
    % Commands \draw with options like [->] must be written individually
	\draw[->](input1) -- node [label={\it{in}}]{}(vca1);
 	\draw[->](vca1) -- node [label={[xshift=20mm]\it{out}}]{} (output1);
	\draw[->](8,0) |- node {} (gain1);
	\draw[->](gain1) -| node [label={[xshift=5mm]\it{cv}}]{} (vca1);

\end{tikzpicture}
\caption{Block diagram of feedback design} 
\label{fig:feedback-blockdiagram}
\end{figure}

\subsection{Gain computer}
\subsection{Peak detector}

\bibliographystyle{plain}
\bibliography{reflist}
\end{document}