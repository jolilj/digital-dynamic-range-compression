\documentclass[]{article}
\usepackage{natbib}
\usepackage{tikz}

\begin{document}

\title{Title}
\author{Author}
\date{Today}
\maketitle
\section{Background}
The dynamic range compressor is a complex nonlinear time dependent system where the the amplitude peaks of a signal is suppressed, reducing its dynamic range. The complexity of the system together with the vast set of design choices makes it difficult to draw a generic block diagram of the process\cite{giannoullis}  and thus we will, in the analysis, focus on a digital implementation corresponding to ideal components of it's analogue counterpart. 

In figure \ref{fig:xy-graph} the eligible behaviour of a simple compression process is illustrated. The x- and y-axis represents the logarithmic amplitude of input and output signal respectively. As the input signal rise above the threshold level, compression is applied. Logarithmic scale is used since sound amplitude is ,by standard, expressed in the logarithmic unit decibel.

\begin{figure}[ht]
\centering
\begin{tikzpicture}

% horizontal axis
\draw[->] (0,0) -- (6,0) node[anchor=north] {$\log x$};
% ranges
\draw	(1,2.2) node{{\scriptsize Threshold}};

% vertical axis
\draw[->] (0,0) -- (0,4) node[anchor=east] {$\log y$};
% nominal speed
\draw[dotted] (0,2) -- (5,2);

% logy = logx
\draw[thick] (0,0) -- (2,2) -- (5,3);

\end{tikzpicture}

\caption{Output signal plotted against input signal. When the input signal rise above the threshold level, compression is applied.} 
\label{fig:xy-graph}
\end{figure}

\bibliographystyle{plain}
\bibliography{reflist}
\end{document}