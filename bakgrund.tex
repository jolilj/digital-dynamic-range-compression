\documentclass[]{article}
\usepackage{natbib}
\usepackage{tikz}
\usepackage{pgfplots}

\begin{document}

\title{Title}
\author{Author}
\date{Today}
\maketitle
\section{Background}
The dynamic range compressor is a complex nonlinear time dependent system where the the amplitude peaks of a signal is suppressed, reducing its dynamic range. The complexity of the system together with the vast set of design choices makes it difficult to draw a generic block diagram of the process\cite{giannoullis}  and thus we will, in the analysis, focus on a digital implementation corresponding to ideal components of it's analogue counterpart. 

In figure \ref{fig:xy-graph} the eligible behaviour of a simple compression process is illustrated. The x- and y-axis represents the logarithmic amplitude of input and output signal respectively. As the input signal rise above the threshold level, compression is applied. Logarithmic scale is used since sound amplitude is ,by standard, expressed in the logarithmic unit decibel.

\begin{figure}[ht]
\centering
\begin{tikzpicture}[scale=0.9,baseline={(0,0)}]

% horizontal axis
\draw[->] (0,0) -- (5.5,0) node[anchor=north] {$X$};
% ranges
\draw	(0.9,2.2) node{{\scriptsize Threshold}};

% Knee label
\draw	(2.2,3.2)
node{{\scriptsize Knee}};
\draw[<-](2,2.1) -- (2.2,3);

% vertical axis
\draw[->] (0,0) -- (0,4) node[anchor=east] {$Y$};
% nominal speed
\draw[dotted] (0,2) -- (5,2);

% logy = logx
\draw[thick] (0,0) -- (2,2) -- (5,3);

\end{tikzpicture}

\caption{Output signal plotted against input signal. When the input signal rise above the threshold level, compression is applied.} 
\label{fig:xy-graph}
\end{figure}

In application is is not desirable to apply instant compression since it introduces distortion\cite{giannoullis}. We thus want to smooth out the compression applied over time. This is illustrated in figure \ref{fig:envelope-graph}

\begin{figure}[ht]
\centering
\begin{tikzpicture}

% horizontal axis
\draw[->] (0,0) -- (6,0) node[anchor=north] {t};
% ranges
\draw	(0.7,1.7) node{{\scriptsize Threshold}};
%legend
\draw[thick] (4,4) -- (4.5,4);
\draw	(5,4) node{{\scriptsize input}};

\draw[dashed] (4,3.5) -- (4.5,3.5);
\draw	(5.05,3.5) node{{\scriptsize output}};

% vertical axis
\draw[->] (0,0) -- (0,4) node[anchor=east] {$\left|A\right|$};
\draw[dotted] (0,1.5) -- (6,1.5);
\draw[dashed,domain=1.5:4] plot (\x,{1.5+0.5*(exp(4*(1.5-\x))});
\draw[dashed,domain=4:6] plot (\x,{1-0.5*(exp(4*(4-\x))});
\draw[dashed] (4,2) -- (4,0.5);

% logy = logx
\draw[thick] (0,1) -- (1.5,1) -- (1.5,2) -- (4,2) -- (4,1) -- (6,1);

\end{tikzpicture}

\caption{Input signal plotted against time. The output is superimposed in the graph as a dashed line. When the input signal rise above the threshold level, compression is applied continously over time } 
\label{fig:envelope-graph}
\end{figure}

\bibliographystyle{plain}
\bibliography{reflist}
\end{document}