\documentclass[]{article}
\usepackage{natbib}
\usepackage{tikz}
\usetikzlibrary{shapes,arrows}
\usepackage{textcomp}
\usepackage{pgfplots}
\usepackage{caption}
\usepackage{subcaption}
\usepackage[swedish,english]{babel}
\usepackage[utf8]{inputenc}
\usepackage{amsmath}

\begin{document}

\title{\Large{Comparing Digital Dynamic Range Compression} \\ \large{A Numerical Approach} }
\author{J. Lilja, O. Karlsson}
\date{\today}
\maketitle

\begin{abstract}
Skrivs i slutet av rapportskrivandet. Sammanfattar metod och resultat.
\end{abstract}

\clearpage

\section{Introduction}
\begin{itemize}
\item{Kan parameterinställningarna hos en kompressor optimeras för att matcha en given signal med kompression på numerisk väg?}
\item{Varför? -  Ger en möjlighet att jämföra kompressorer på riktiga signaler. Har ej gjorts. Tidigare forskning jämför algoritmer kvalitativt. Se \cite{giannoullis}. Kompression är en subtil effekt, möjliggör att utforska var gränsen går för mänskliga örat}
\end{itemize}

\section{Background}

\subsection{Previous work}
Compressor design
\begin{itemize}
\item{Digital dynamic range compressor design—a tutorial and analysis\cite{giannoullis}}
\item{DAFX\cite{dafx}}
\item{Digital Signal Processing(Zoelzer) \cite{zolzer1997digital}}
\end{itemize}
Parameter determination
\begin{itemize}
\item{Reverse Engineering of a Mix \cite{reiss2010rev}}
\end{itemize}

\subsection{Theory}

\subsubsection{Digital signal processing}
Hur ljud är representerat i digital form

\subsubsection{Dynamic Range Compression}
Här beskrivs utförlig teori av hur kompressorn är byggd/implementerad (Det vi påbörjat tidigare som bakgrund). 

\subsubsection{Numerical analyisis}
\begin{itemize}
\item Linear least squares fitting
\item Nonlinear optimization (curve fitting/function minimization)

\begin{itemize}
\item Convex optimization?
\item Nelder-Mead simplex algorithm?
\end{itemize}

\end{itemize}

\section{Method}
Vi börjar att testa MATLABS fminsearch på de kompressoralgoritmer vi implementerat, enligt \cite{giannoullis}, i MATLAB och se vilka resultat vi får. I första hand kolla konvergens i specialfall, exempelvis hitta parameterinställningen för samma kompressor.

Metric: Euklidisk norm, tidsdomän? spektrum? envelope?

Hur kan vi förbättra fminsearch, givet vårt specifika problem?

I framtiden lyssnartest?

\section{Result and analysis}
Jämföra numeriska mått med lyssnartest?

\section{Discussion}
Ev. problem med den numeriska metoden?

\bibliographystyle{plain}
\bibliography{reflist}
\end{document}



