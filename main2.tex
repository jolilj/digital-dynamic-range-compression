\documentclass[]{article}

%document structure
\usepackage{subfiles}
%plots
\usepackage{tikz}
\usetikzlibrary{dsp,chains}
\usetikzlibrary{shapes,arrows}
\usetikzlibrary{calc}
\usepackage{pgfplots}
% encoding and symbols
\usepackage[swedish,english]{babel}
\usepackage[utf8]{inputenc}
\usepackage{amsmath,amssymb}
\usepackage{textcomp}
%formatting
\usepackage[titletoc,title]{appendix}
\usepackage{titlesec}
\usepackage{caption}
\usepackage{subcaption}
\usepackage{placeins}
\usepackage[colorinlistoftodos,prependcaption,textsize=tiny]{todonotes}
\usepackage{url}
\usepackage{enumitem} 
%\usepackage{natbib}

% table extenstions
\newcommand*\rot{\rotatebox{90}}
\usepackage{multirow}
\newcommand{\tblbox}[2][c]{%
  \begin{tabular}[#1]{@{}c@{}}#2\end{tabular}}

\usepackage[normalem]{ulem}%for strikethrough

% section numbering
\setcounter{secnumdepth}{3}
\setcounter{tocdepth}{3}

% make paragraphs behave like level 4 sections
\titleformat{\paragraph}[hang]
{\normalfont\normalsize\bfseries}{\theparagraph}{1em}{}
\titlespacing*{\paragraph} {0pt}{3.25ex plus 1ex minus .2ex}{1.5ex plus .2ex}


% MATLAB CODE
\usepackage{listings, color}
\definecolor{light-gray}{gray}{0.95}
\lstdefinestyle{customc}{
 backgroundcolor=\color{light-gray},
  belowcaptionskip=1\baselineskip,
  breaklines=true,
  frame=L,
  xleftmargin=\parindent,
  language=C,
  showstringspaces=false,
  basicstyle=\footnotesize\ttfamily,
  keywordstyle=\bfseries\color{green!40!black},
  commentstyle=\itshape\color{purple!40!black},
  identifierstyle=\color{blue},
  stringstyle=\color{orange},
}

\lstdefinestyle{output}{
 backgroundcolor=\color{light-gray},
 language=bash,
 xleftmargin=\parindent,
 keywordstyle=\color{black},
 basicstyle=\ttfamily,
 morekeywords={peter@kbpet},
 %alsoletter={:~$},
 morekeywords=[2]{peter@kbpet:},
}

% custom commands
\newcommand{\dB}		{\ensuremath{\mskip3mu\text{dB}}}
\newcommand{\dBFS}	{\ensuremath{\mskip3mu\text{dB~FS}}}
\newcommand{\dBSPL}	{\ensuremath{\mskip3mu\text{dB~SPL}}}
\newcommand{\mPa}	{\ensuremath{\mskip3mu\mu\text{Pa}}}
\newcommand{\logten}	{\ensuremath{\log_{10}}}
\newcommand{\THDF}	{\ensuremath{\text{THD}_\text{F}}}
\newcommand{\ECR}	{\ensuremath{\text{ECR}}}
\newcommand{\FES}	{\ensuremath{\text{FES}}}
\newcommand{\EN}		{\ensuremath{d_\text{e}}}
\newcommand{\crest}	{\ensuremath{f_\text{crest}}}
\newcommand{\Crest}	{\ensuremath{F_\text{crest}}}


\begin{document}
%set this directory to root directory, for subfiles and figures paths
%must be set here, so it's not included when compiling subfiles standalone
\newcommand{\rootdir}{.}

\title{Digital Dynamic Range Compression \\ \Large An Overview and Comparison of Feed-forward Designs}
\author{O. Karlsson, J. Lilja}
\date{\today}
\maketitle

\begin{abstract}
An overview of digital dynamic range compressor designs proposed in literature and scientific publications is presented. Four different peak-detecting feed-forward compressors are evaluated with respect to a defined ideal output for an artificial test signal using a parameter optimisation method. All four are shown to approximate the ideal case within a small error. A large set of components coded in MATLAB\textsuperscript{\textregistered}, provided in the appendix, lays out a foundation for future work.
\end{abstract}
\clearpage

\tableofcontents
\clearpage

\subfile{\rootdir/sections/introduction}
\clearpage

\subfile{\rootdir/sections/theory_dynamic_range}
\subfile{\rootdir/sections/theory_DDRC}
\subfile{\rootdir/sections/theory_metrics_ideal}

\subfile{\rootdir/sections/method}
\subfile{\rootdir/sections/result}

\subfile{\rootdir/sections/discussion}
\clearpage
\subfile{\rootdir/sections/appendix}

\FloatBarrier
\clearpage
\bibliographystyle{alpha}
\bibliography{reflist}
\end{document}
