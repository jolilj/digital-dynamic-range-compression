\documentclass[../main2.tex]{subfiles}
\begin{document}

\newcommand{\z}{z}
% FIR filter as block diagram
\begin{tikzpicture}

\tikzset{recblock/.style={dspfilter,minimum width=18mm, minimum height=10mm}};
\tikzset{sqrblock/.style={dspfilter,minimum size=10mm}};


	% Place nodes using a matrix
	\matrix (m1) [row sep=4.0mm, column sep=5mm]
	{
		%--------------------------------------------------------------------
		\node[coordinate] (m00) {};    &
		\node[coordinate]                  (m01) {};          &
		\node[coordinate]                 (m02) {};           &
		\node[coordinate] (m03) {};         &
	         \node[coordinate]                 (m04) {};           &
		\node[coordinate]                   (m05) {}; 	&
		\node[coordinate]                   (m08) {}; 	&
		\node[coordinate]                   (m09) {}; 	&
		\node[coordinate]                   (m010) {}; 	&
		\node[coordinate] (m0X) {};          \\
		%--------------------------------------------------------------------
		\node[dspnodeopen,dsp/label=above] (m10) {$x_n$};    &
		\node[coordinate]                 (m14) {};           &
		\node[sqrblock]                  (m11) {$z^{-d_{la}}$};          &
		\node[coordinate] (m15) {};         &
		\node[coordinate] (m17) {};    &
		\node[dspmixer] (m18) {};    &
		\node[coordinate]                   (m19) {}; 	&
		\node[coordinate]                   (m110) {}; 	&
		\node[dspnodeopen,dsp/label=above] (m1X) {$y_n$};          \\
		%--------------------------------------------------------------------
		\node[coordinate] (m20) {};    &
		\node[coordinate]                  (m21) {};          &
	         \node[coordinate]                 (m24) {};           &
		\node[coordinate]                   (m25) {}; 	&
		\node[coordinate]                   (m26) {}; 	&
		\node[coordinate]                   (m28) {}; 	&
		\node[coordinate]                   (m29) {}; 	&
		\node[coordinate]                   (m210) {}; 	&
		\node[coordinate] (m2X) {};          \\
		%--------------------------------------------------------------------
	};
	
	% Draw connections
	\begin{scope}[start chain]
		\chainin (m10);
		\chainin (m11) [join=by dspconn];
		\chainin (m18) [join=by dspconn];
		\chainin (m1X) [join=by dspconn];
	\end{scope}

	
	\draw[dspconn] (m28)++(0,-0.3) -- node[near start,right] {$g_n$} (m18);

	
	\end{tikzpicture}

\end{document}