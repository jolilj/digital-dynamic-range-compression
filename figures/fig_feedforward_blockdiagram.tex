\documentclass[tikz]{standalone}
\usetikzlibrary{shapes,arrows}
\usepackage{textcomp}
\begin{document}
\tikzset{%
  amp/.style	= {draw, regular polygon, regular polygon sides=3,
	shape border rotate=-90, node distance = 70mm},
  block/.style    = {draw, thick, rectangle, minimum height = 3em,
    minimum width = 3em},
  input/.style	= {coordinate}, % Input
  output/.style	= {coordinate}, % Output
cross/.style={path picture={ 
  \draw[black]
(path picture bounding box.south east) -- (path picture bounding box.north west) (path picture bounding box.south west) -- (path picture bounding box.north east);
}}
}

\begin{tikzpicture}[auto, thick, node distance=2cm, >=triangle 45]
\draw
	% Drawing the blocks of first filter :
	node at (0,0)[right=-3mm]{\Large \textopenbullet}
	node [input, name=input1] {} 

	node at (1.5,0){\textbullet}

	node [draw,circle,cross, minimum width = 0.7cm, node distance = 70mm, right of=input1, name=vca1, ] {}
	node [block, below left of=vca1, xshift=-15mm, yshift=-5mm, name=gain1]{Gain computer}
	node [output, right of=vca1, name=output1, xshift=10mm]{}
	;
    % Joining blocks. 
    % Commands \draw with options like [->] must be written individually
	\draw[->](input1) -- node [label={[xshift=-25mm]$x_n$}]{}(vca1);
 	\draw[->](vca1) -- node [label={$y_n$}]{} (output1);
	\draw[->](1.5,0) |- node {} (gain1);
	\draw[->](gain1) -| node [label={[xshift=5mm]$g_n$}]{} (vca1);

\end{tikzpicture}
\end{document}