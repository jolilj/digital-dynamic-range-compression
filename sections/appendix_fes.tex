\documentclass[../main2.tex]{subfiles}
%if compiling standalone, rootdir wil be previous folder,
%if compiling main document, rootdir will already be set by main file
\providecommand{\rootdir}{..}

\begin{document}
%================================
\section{Ideal Compression}
In order to develop a method for comparing DDRC designs a definition of ideal compression is needed. Let $x$ be the input and $y$ be the compressed output. Further assume that $x$ can be expressed as a product of it's envelope $E$ and fine-structure $F$ \begin{align}
x_n = E_nF_n.
\end{align}
The idea is illustrated in Fig. ~\ref{}. Let $G(x)$ be the gain computer, see section \ref{gain_computer}, and $g(x) = 10^{G/20}$. The ideal compression is then given by
\begin{align}
y_{n,I} \equiv g(E_n) F_n. 
\end{align}


\section{Fidelity of Ideal Envelope Shape} \label{fes}
In ~\cite{stone2007quantifying} a metric for measuring the correlation between the envelope shape before and after compression is proposed and further used in the design comparison conducted in ~\cite{reiss2012tutorial}. The Pearson correlation coefficient is defined as
\begin{align}
r_{xy} = \dfrac{\sum(x-\bar{x})(y-\bar{y})}{\sqrt{\sum(x-\bar{x})^2}\sqrt{\sum(y-\bar{y})^2}}
\end{align}
where $\bar{x}$ and $\bar{y}$ denote the mean value of $x$ and $y$, respectively. To relate to ideal compression, the Fidelity of Ideal Envelope Shape (FEIS) coefficient is defined as

\begin{align}
c_{f} = \dfrac{\sum(g(E)F-\bar{})(y-\bar{y})}{\sqrt{\sum(x-\bar{x})^2}\sqrt{\sum(y-\bar{y})^2}}
\end{align}
\end{document}