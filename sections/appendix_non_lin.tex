\documentclass[../main2.tex]{subfiles}
%if compiling standalone, rootdir wil be previous folder,
%if compiling main document, rootdir will already be set by main file
\providecommand{\rootdir}{..}

\begin{document}
%================================
\section{Effects of Non-Linear Operations} \label{non_lin_ops}
Non-linear operations introduces higher harmonics. A simple example of a pure sinusoidal signal illustrates this for the squaring operation
%================================
\begin{align}
(A\sin{\omega t})^2 = \frac{A^2}{2}(1-\cos{2\omega t}) 
\end{align}
%================================
showing that only the second harmonic is generated, which can be suppressed by a lowpass filter.
For the magnitude operation the Fourier expansion of a sinusoidal input yields
%================================
\begin{align}
\left |A\sin{\omega t}\right | = \frac{2A}{\pi}\left (1 - \sum_{n=1}^{\infty}\frac{4}{4n^2-1} \text{cos} (2n\omega t) \right) 
\end{align}
%================================
introducing an infinite number of even harmonics. A problem arises if the generated harmonic is close to the sample frequency, $f_s$ it would be aliased into zero frequency creating a static measurement error. This cannot be removed by a lowpass filter and has to be considered in implementations.

\end{document}