\documentclass[../main2.tex]{subfiles}
%if compiling standalone, rootdir wil be previous folder,
%if compiling main document, rootdir will already be set by main file
\providecommand{\rootdir}{..}

\begin{document}
\subsection{Conclusions}\label{discussion_results}
In this thesis a thorough compilation of the design choices proposed in literature and scientific publications was made and implemented. Furthermore four prominent DDRCs were tested and compared using a definition of ideal compression, and in an attempt to focus on the actual algorithms and not the parameter settings, a numerical optimisation of the compressor parameters were conducted. Here follows the conclusion drawn from the method and results of this thesis.

First of all, a decision has to be made wether the compressor should operate on the envelope or the RMS of the signal. In this thesis only peak detecting compressors were evaluated due to the inherent ambiguity of which \emph{time scale} to average over. 

The GMR12 compressors with full rectification as only level detection performs well and are fairly stable across the carrier frequency sweep in contrast with the design of \cite{mcnally1984dynamic} that does well in being optimised for a specific carrier frequency, but is being outperformed as the carrier frequency changes. However, this is based on a fairly uncomplicated test signal. Before any conclusions can be drawn regarding the general performance of the tested compressors, more work needs to be done. It is however worth to note that, with the user in mind, a minimum number of parameter settings might be preferable thus motivating further evaluation of the GMR12 compressors.

The results do, however, show that up-sampling is of interest. Which specific method to use was beyond the scope of this thesis, but the up-sampling peak detector was shown to generate the desired behavior reducing the spikes caused by the introduced harmonics.

The effect of look-ahead on performance is yet to be investigated.

\subsection{Future Work}
Although this thesis presents few general guidelines for digital feed forward compressor design, a foundation has been laid for future work. Further evaluation is easily carried out with the MATLAB\textsuperscript{\textregistered} code provided in appendix~\ref{appendix_code}. 

The end result of the complete compressors are extremely close to ideal, when parameters are optimised. This raises the suspicion that parameter settings may be more important the underlying algorithm. One line of further research is how easy the settings are to adjust for the user. How the compressor "responds" to different settings, thereby affecting how easy the wanted end result is to achieve. Not to mention the user interface and human computer interaction.

Another possibility is that the test signal wasn't complex enough to level detect and compress. The smooth attack and release of the raised cosine envelope closely resembles the natural attack and release step responses of the tested components (both level detector filters and gain smoothing filters). The method could easily be used to test on other envelopes. Either constructed ones like a skew triangle wave resembling fast attack att linear decay, or more elaborate models of real instrument envelopes.

This also opens up the possibility of investigating how well a compressor can \emph{deform} an envelope to a wanted shape. This was not included in this paper, and the effect of the gain smoothing stage was merely used to mitigate defects in the level detection, without investigating their envelope shaping potential.

Finally, the complex interaction of the nonlinear components needs a more systematic evaluation. Possibly the components need to be tested in isolation from the system and then compared to the final output of the whole compressor.  
\end{document}