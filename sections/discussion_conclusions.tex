\documentclass[../main2.tex]{subfiles}
%if compiling standalone, rootdir wil be previous folder,
%if compiling main document, rootdir will already be set by main file
\providecommand{\rootdir}{..}

\begin{document}
\subsection{Conclusions}\label{discussion_results}
The conclusions drawn are presented in accordance with the method beginning with the level detection stage, followed by the gain computation and ending with the complete compressor design.

First of all, a decision has to be made wether the compressor should operate on the envelope or the RMS of the signal. In the latter, the results show the two tested RMS detectors are close to the defined ideal RMS specified. However, due to the "kind" test signal, further evaluation has to be made with a different set of signals to draw valid conclusions to which RMS detector to use. The IIR RMS detector have the advantage of not necessarily depending on look-ahead, although it lowers it's performance considerably.

The focus  hence lies at the peak detecting compressors. The results suggests that the smoothing filters as level detection handles the carrier frequency sweep best. The window max filter is more sensitive to lower carrier frequencies leaving either the smooth decoupled or smooth branching are good candidates. However, the compressors with full rectification as only level detection performs well and are fairly stable across the carrier frequency sweep in contrast with the design of \cite{mcnally1984dynamic} that does well in being optimised for a specific carrier frequency, but is being outperformed as the carrier frequency changes.

The results further suggests a balance act between performance and delay. As the amount of delay depends on application, it is left for future work to further investigate this relationship closer.

Upsampling was shown to be of great use to reduce the introduced harmonics due to the non-linearities. It was however beyond the scope of this thesis to discuss such methods.

\subsection{Future Work}
The end result of the complete compressors are extremely close to ideal, when parameters are optimised. This raises the suspicion that parameter settings may be more important the underlying algorithm. One line of further research is how easy the settings are to adjust for the user. How the compressor "responds" to different settings, thereby affecting how easy the wanted endresult is to achieve. Not to mention the user interface and human computer interaction.

Another possibility is that the test signal was to "easy" to level detect and compress. The smooth attack and release of the raised cosine envelope closely resembles the natural attack and release step responses of the tested components (both level detector filters and gain smoothing filters).

The method could easily be used to test on other envelopes. Either constructed ones like a skew triangle wave resembling fast attack att linear decay, or more elaborate models of real instrument envelopes.

This also opens up the possibility of investigating how well a compressor can \emph{deform} an envelope to a wanted shape. This was not included in this paper, and the effect of the gain smoothing stage was merely used to mitigate defects in the level detection, without investigating their envelope shaping potential. 


\end{document}