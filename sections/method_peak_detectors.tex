\documentclass[../main2.tex]{subfiles}
%if compiling standalone, rootdir wil be previous folder,
%if compiling main document, rootdir will already be set by main file
\providecommand{\rootdir}{..}

\begin{document}
\subsection{Evaluation of Peak Detectors}\label{method_peak_detectors}
A test signal $x_n$ is generated, consisting of an envelope $e_n$ and a fine structure $f_n$:
\begin{align}
x_n = e_n f_n.
\end{align}
The envelope describes the instantaneous peak amplitude of the input signal, and the peak detectors' ability to extract this envelope was evaluated. To minimise the ambiguity of the definition of envelope, the test signal was chosen to be an amplitude modulated sine wave defined as
\begin{equation}
\begin{split}
	e_n &= a_\text{min} + \frac{a_\text{max}- a_\text{min}}{2} \left(1 + \cos(2 \pi f_m n T_s) \right), \\
	f_n &= \sin(2 \pi f_c n T_s),
\end{split} \label{eq:test_signal}
\end{equation}
where $a_\text{min}$ is the minimum instantaneous peak amplitude, $a_\text{max}$ is the maximum instantaneous peak amplitude, $f_m$ the modulation frequency and $f_c$ the carrier frequency.

\subsubsection{Distance to Ideal Envelope}
The level estimation produced by the peak detector, as before denoted by $L_n$, was compared to the specified envelope level
\begin{equation}
E_n = 20 \logten e_n \dBFS
\end{equation}
by the following error function
\begin{equation}
D_\text{error} = \sqrt{ \frac{1}{N}\sum_{n = 0}^{N-1} (E_n - L_n)^2 } \dB / \text{sample},
\end{equation}
where $D_\text{error}$ is the euclidian distance and N is the length of the test signal in samples. 

Given a modulation frequency, a peak detector's ability to extract the correct audio level depends greatly on the parameter settings, as these in essence define the time-scale the detector should operate on, see section \ref{theory_time_scales}. To be able to evaluate the algorithms, rather than their parameter settings, the parameter values was optimised numerically for each detector with a fixed modulation frequency $f_m = f_{m,0}$ and carrier frequency $f_c = f_{c,0}$, using $D_\text{error}$ as cost function.

\subsubsection{Carrier Frequency Dependence}
While the optimised parameter values are closely related to modulation frequency, possible dependencies on carrier frequency are unwanted, since the fine structure ideally should not affect the estimated envelope. To evaluate how peak detector performance depends on the fine structure, the parameter values as well as modulation frequency was fixed, while the carrier frequency was swept from 20 Hz to 20 kHz and the resulting error $D_\text{error}$ was plotted against $f_c$.

\subsubsection{Estimated Envelope Shape}
As the calculated error $D_\text{error}$ is the normalised distance between the estimated and the specified signal level, it does not distinguish between static errors and unwanted oscillations or other envelope distortions. Thus, FES was calculated as the correlation between the estimated envelope $\tilde{e}_n$ and the original envelope $e_n$ and plotted against carrier frequency as well. Note that while FES is used in \cite{stone1992syllabic}\cite{stone2007quantifying} and \cite{reiss2012tutorial} to correlate the envelopes of the input signal with the envelope of the output from a \emph{complete compressor}, it is used here to evaluate the different peak detector components in isolation. 
\end{document}