\documentclass[../main2.tex]{subfiles}
%if compiling standalone, rootdir wil be previous folder,
%if compiling main document, rootdir will already be set by main file
\providecommand{\rootdir}{..}

\begin{document}
\subsection{Critique}\label{discussion_results}
The results presented in this thesis are not to be seen as a final evaluation of DDRC design. The test signals show little of the complexity of real-world audio and the definition of ideal compression is debatable. As DRC is a popular effect in creative and artistic contexts, transparency might not be a desired property. Furthermore, the measured transparency of the DDRC design is theoretical. It's is unclear how this maps to the perceived notion of transparency. The results of this thesis should thus be followed up with listening tests. However, we argue that with a solid foundation to build on, transparency is always motivated. Non-linear processing is a one way street. Once compressed, it's difficult\todo{source} to reverse the process. 

Another issue is the rather large number of parameter settings, especially in the DDRC designs involving two filters, reducing the usability. This can be handled in a number of ways in practice. The parameters can be set to fixed values, defining the sound of the compressor. The user interface can present other parameters to the user, that the underlying parameters are calculated from. Another approach is parameter automation. This has been considered for a long time as even in the earlier analogue versions auto-settings were introduced\todo{source}, e.g. auto-release. This has been the subject for a number of scientific publications, see for example \todo{source}. We believe that the results presented in this thesis can be of use in developing these adaptive compressor algorithms.

\end{document}