\documentclass[../main2.tex]{subfiles}
%if compiling standalone, rootdir wil be previous folder,
%if compiling main document, rootdir will already be set by main file
\providecommand{\rootdir}{..}

\begin{document}
%================================
\subsection{Fidelity of Ideal Envelope Shape} \label{fes}
In ~\cite{stone2007quantifying} a metric for measuring the correlation between the envelope shape before and after compression is proposed and further used in the design comparison conducted in ~\cite{reiss2012tutorial}. 

The Fidelity of Envelope Shape intends to measure the correlation between the input and output envelope assuming ideal compression, see section \ref{ideal_compression}.
With $O(t)$ is the output envelope, $G(t)$ the gain and $E(t)$ the input envelope in the logarithmic domain the input output relationship can be expressed as
%================================
 \begin{align}
O(t) = G(t) + E(t).
\label{eq:out_env}
\end{align}
%================================
Assuming the envelope is above the threshold, $T$, thus escaping the discontinuity at the knee, the gain can be expressed as, see section ~\ref{gain_computer},
%================================
\begin{align}
G(t) = (R^{-1}-1)(X(t)-T) = C + (R^{-1}-1)\tilde{E}(t)
\label{eq:gain_env}
\end{align}
%================================
where $C = (1-R^{-1})T$, $R$ being the compression ratio and $\tilde{E}$ is the smoothed envelope detected by the compressor. Eq. \eqref{eq:out_env} with Eq. \eqref{eq:gain_env} yields
\begin{align}
O(t) = C + (R^{-1}-1)\tilde{E}(t) + E(t).
\end{align}
%================================
The Fidelity of Envelope Shape is thus defined as the correlation between $E(t)$ and $O(t)$ using the Pearson correlation coefficient
\begin{align}
FES = \dfrac{\sum(E-\overline{E})(O-\overline{O})}{\sqrt{\sum(E-\overline{E})^2}\sqrt{\sum(O-\overline{O})^2}}
\end{align}
where $\overline{E}$ and $\overline{O}$ denote the mean value of $E$ and $O$, respectively.
%================================

Another approach would be to include the nonlinearity of the dynamic range mapping introduced by either a smooth or sharp knee. Assuming the input signal can be expressed as a product of it's envelope and fine structure, $x  = e\cdot f$, see section \ref{ideal_compression}, we thus propose measuring the correlation between the ideally compressed envelope $e_{I}=\dfrac{y_{I}}{f} = g(e)$ and the compressed envelope  given by $e_c = \dfrac{y}{f}$. The Fidelity of Ideal Envelope Shape (FEIS) coefficient is defined as
%================================
\begin{align}
c_{f} = \dfrac{\sum\left(E_I-\overline{E_I}\right)\left(E_c-\overline{E_c}\right)}{\sqrt{\sum\left(E_I-\overline{E_I}\right)^2}\sqrt{\sum\left(E_c-\overline{E_c}\right)^2}}
\end{align}
%================================
where $E_I$ and $E_c$ are the logarithm of $e_I$ and $e_c$, respectively.
\todo[inline]{Eftersom vi nu jämför det idealt komprimerade envelopet med det "uppskattade" komprimerade envelopet så är det ju egentligen ganska dumt att använda korrelation..dom ska ju vara lika -.-}
\end{document}