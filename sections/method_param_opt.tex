\documentclass[../main2.tex]{subfiles}
\providecommand{\rootdir}{..}
\begin{document}

\subsection{The Importance of Parameter Settings}\label{method_param_opt}
In \cite{reiss2012tutorial} the placement, type and domain of the level detector is partly motivated by comparing step and down-step response of the compressors with equal parameter settings. To investigate the importance of the chosen parameter settings in such comparisons, the following test was conducted.

Two compressor designs presented in \cite{reiss2012tutorial} were tested
\begin{enumerate}[label=(\Alph*)]
 \item Full wave rectification with smooth branching log domain gain smoothing (recommended)
\item  Branching peak detector without gain smoothing (not recommended)
\end{enumerate}
First, a step and down-step response with equal parameter settings are generated. Second, the same step and down-step response is generated but with the parameter settings of compressor B optimised by fitting it's output to the output of compressor A, using the euclidean distance as cost function.
\end{document}