\documentclass[../main2.tex]{subfiles}
%if compiling standalone, rootdir wil be previous folder,
%if compiling main document, rootdir will already be set by main file
\providecommand{\rootdir}{..}

\begin{document}
\FloatBarrier
\subsection{Gain Smoothing}
To further smooth the control signal, various filters are introduced in the end of the side-chain, see Fig.~\ref{fig:block_genericDDRC}, operating in either the linear or logarithmic domain. In \cite{reiss2012tutorial} this is the only smoothing performed and thus paramount for avoiding distortion. The input to the smoothing step will be denoted by primed variables, and the smooth output with non-primed variables.

%================================
\subsubsection{Adaptive Filter} \label{adaptive_filter}
In \cite{mcnally1984dynamic} a first order lowpass filter is added in the linear domain. A block diagram of the filter can be seen in Fig.~\ref{fig:block_mcnally_theory_adap_filter} where the control block represents the branching condition as discussed below. The difference equation for one version of the filter is
%================================
\begin{equation}\label{eq:mcnally_gain_smoothing}
g_n = \begin{cases}
    \alpha_\text{sa} g'_n + (1-\alpha_\text{sa}) g_{n-1}   & g'_n \leq g_{n-1} \\
    \alpha_\text{sr} g'_n + (1-\alpha_\text{sr}) g_{n-1}	   & g'_n > g_{n-1}
\end{cases}
\end{equation}
%================================
where the attack and release coefficients $\alpha_\text{sa}$, $\alpha_\text{sr}$ are related to attack and release time constants $\tau_\text{sa}$, $\tau_\text{sr}$ by 
\begin{align}
\alpha_\text{sa} &= 1 - e^{-1/\tau_\text{sa} f_s } \label{eq:attack_coeff_def}\\
\alpha_\text{sr} &= 1 - e^{-1/\tau_\text{sr} f_s } \label{eq:release_coeff_def}.
\end{align}
The block diagrams in \cite{mcnally1984dynamic, dafx02, dafx11, zolzer1997digital, zolzer2008digital} all suggest that the branching condition is determined by $g'_n > g'_{n-1}$, that is, by comparing the current and previous \emph{input} samples instead of comparing the current input with the previous \emph{output}. However, when described in a code snippet appearing for the first time in \cite{dafx11}, the branching condition of equation \eqref{eq:mcnally_gain_smoothing} is actually used. In \cite{bitzer2006parameter} the branching condition of equation \eqref{eq:mcnally_gain_smoothing} is used while referring to \cite{mcnally1984dynamic, dafx02}. The original intent in \cite{mcnally1984dynamic} is unclear.

%================================
\begin{figure}
\centerline{\subfile{\rootdir/figures/block_mcnally_theory_adap_filter}}
\caption{Block diagram of the adaptive filter in \cite{mcnally1984dynamic}}
\label{fig:block_mcnally_theory_adap_filter}
\end{figure}
%================================
\subsubsection{Linear Smoothing in Log Domain}
In \cite{frindle1996implementation} a gain smoothing is implemented in the log domain as
\begin{equation}\label{eq:frindle_gainsmooth}
\begin{split}
Z_n &= \text{max}(C'_n, C_{n-1} - A_\text{sr} H_{n-1} )\\
C_n &= \text{min}(Z_n, C_{n-1} + A_\text{sa}) \\
H_n &=
\begin{cases}
    \text{min}(1,A_\text{sh} + H_{n-1})	& C_n < C_{n-1}; \\
    A_\text{sh}					& \text{otherwise}
\end{cases} \\
\end{split}
\end{equation}
where
\begin{equation}
\begin{split}
A_\text{sa} &= \frac{1}{f_s \tau_\text{sa}} \\
A_\text{sr} &= \frac{1}{f_s \tau_\text{sr}} \\
A_\text{sh} &= \frac{0.5}{f_s \tau_\text{sh}}. \\
\end{split}
\end{equation}
and the time constants $\tau_\text{sa}, \tau_\text{sr}$ and $\tau_\text{sh}$ are given in seconds per $\dBFS$. Note that the filter acts on the positive logarithmic quantity \emph{compression}, $C'_n = -G'_n$, instead of the actual gain reduction $G'$.

The attack and release trajectories are linear in the log domain, as can be seen in Fig.~\ref{fig:step_frindle_gain}. In the release phase the hold timer $H_n$ scales the release time constant up to it's final value. The goal of this is to get a smooth release characteristic with low distortion for low frequency content and a smooth transition from hold to release phase.
%================================
\begin{figure}[h]
\centerline{\subfile{\rootdir/figures/step_frindle_gain}}
\caption{Linear gain smoothing, as implemented in \cite{frindle1996implementation}, step and downstep response with $\tau_\text{sa}=0.02$, $\tau_\text{sr}=0.08$ and $\tau_\text{sa}= 0.1$ s/$\dBFS$.}
\label{fig:step_frindle_gain}
\end{figure}
%================================
\subsubsection{Smooth Decoupled and Branching Filters}
In \cite{reiss2012tutorial} the gain smoothing is referred to as \emph{level detection}, although it is a \emph{smoothing filter} applied to the calculated compression $C'_n$. Two different implementations are proposed, here named in accordance with \cite{reiss2012tutorial}.
\begin{itemize}
\item{Smooth decoupled filter}
\item{Smooth branching filter}
\end{itemize}
The difference equation of the \emph{smooth decoupled filter} is
%================================
\begin{equation}\label{eq:smooth_decoupled_det}
\begin{split}
Z_n &= \begin{cases}
   C'_n									& C'_n > Z_{n-1} \\
    \alpha_\text{sr} C'_n + (1-\alpha_\text{sr}) Z_{n-1} & C'_n \leq Z_{n-1}
\end{cases} \\
C_n &= \alpha_\text{sa} Z_n + (1-\alpha_\text{sa}) C_{n-1}
\end{split}
\end{equation}
%================================
$\alpha_\text{sa}$, $\alpha_\text{sr}$ are related to attack and release time constants $\tau_\text{sa}$, $\tau_\text{sr}$ as before. The output of this smoothing filter has a gentle transition between the attack and release phase as can be seen in Fig.~\ref{fig:step_reiss_filter}. The name is poorly chosen since it in fact couples the attack and release parameters in the release phase. In \cite{reiss2012tutorial} it is derived from it's analog counterpart, and it is shown that the actual release time is approximately $\tau_\text{sa} + \tau_\text{sr}$.

To obtain a filter with attack and release parameters independent of each other the \emph{smooth branching filter}, same as described in section \ref{adaptive_filter} above, is suggested, here operating in the logarithmic domain:
%================================
\begin{equation}\label{eq:smooth_branching_det}
C_n = \begin{cases}
    \alpha_\text{sa} C'_n + (1-\alpha_\text{sa}) C_{n-1} 	& C'_n > C_{n-1} \\
    \alpha_\text{sr} C'_n + (1-\alpha_\text{sr}) C_{n-1} 	& C'_n \leq C_{n-1}.
\end{cases}
\end{equation}
%================================
The step and down-step response in Fig.~\ref{fig:step_reiss_filter} shows the different release trajectories of the smooth decoupled and the smooth branching filter.%================================
\begin{figure}
\centerline{\subfile{\rootdir/figures/step_reiss_filter}}
\caption{Step and downstep response of the smooth decoupled and smooth branching filter, with $\tau_\text{sa} = 50$ ms, $\tau_\text{sr} = 150$ ms}
\label{fig:step_reiss_filter}
\end{figure}
%================================

Observe the similarities between the branching filter described in section \ref{peak_detection} and the \emph{smooth} branching filter described above. The latter was named \emph{smooth} in \cite{reiss2012tutorial} for it's return to signal behaviour in the release phase, in contrast to the return to zero behaviour of the former.
\end{document}