\documentclass[../main2.tex]{subfiles}
%if compiling standalone, rootdir wil be previous folder,
%if compiling main document, rootdir will already be set by main file
\providecommand{\rootdir}{..}

\begin{document}
\FloatBarrier
\subsection{Gain Smoothing}
To further smooth the control signal, various filters are introduced in the end of the side-chain, see Fig.~\ref{fig:fig:block_genericDDRC}, operating in either the linear or logarithmic domain. In \cite{reiss2012tutorial} this is the only smoothing performed and thus paramount for avoiding distortion.
%================================
\subsubsection{Adaptive Filter} \label{adaptive_filter}
In \cite{mcnally1984} a first order lowpass filter is added in the linear domain. A block diagram of the filter can be seen in Fig.~\ref{fig:block_mcnally_theory_adap_filter} where the control block represents the branching condition as discussed below. The difference equation for one version of the filter is
%================================
\begin{equation}\label{eq:mcnally_gain_smoothing}
g_n = \begin{cases}
    \alpha_{a} g'_n + (1-\alpha_{a}) g_{n-1} 	& g'_n \leq g_{n-1} \\
    \alpha_{r} g'_n + (1-\alpha_{r}) g_{n-1} 		& g'_n > g_{n-1}
\end{cases}
\end{equation}
%================================
The block diagrams in \cite{mcnally1984dynamic}\cite{dafx02, dafx11, zolzer1997digital, zolzer2008digital} all suggest that the branching condition is determined by $g'_n > g'_{n-1}$, that is, by comparing the current and previous \emph{input} samples instead of comparing the current input with the previous \emph{output}. However, when described in a code snippet appearing for the first time in \cite{dafx11}, the branching condition of equation \eqref{eq:mcnally_gain_smoothing} is actually used. In \cite{bitzer2006parameter} the branching condition of equation \eqref{eq:mcnally_gain_smoothing} is used while referring to \cite{mcnally1984dynamic, dafx02}. The original intent in \cite{mcnally1984dynamic} is unclear.

%================================
\begin{figure}
\centerline{\subfile{\rootdir/figures/block_mcnally_theory_adap_filter}}
\caption{Block diagram of the adaptive filter in \cite{mcnally1984dynamic}}
\label{fig:block_mcnally_theory_adap_filter}
\end{figure}
%================================
\subsubsection{Linear Smoothing in Log Domain}
In \cite{frindle1996implementation} a gain smoothing is implemented in the log domain as
\begin{equation}\label{eq:frindle_gainsmooth}
\begin{split}
Z_n &= \text{max}(C'_n, C_{n-1} - A_r H_{n-1} )\\
C_n &= \text{min}(Z_n, C_{n-1} + A_a) \\
H_n &=
\begin{cases}
    \text{min}(1,A_h + H_{n-1})	& C_n < C_{n-1}; \\
    A_h					& \text{otherwise}
\end{cases} \\
\end{split}
\end{equation}
where
\begin{equation}
\begin{split}
A_a &= \frac{1}{f_s \tau_a} \\
A_r &= \frac{1}{f_s \tau_r} \\
A_h &= \frac{0.5}{f_s \tau_h}. \\
\end{split}
\end{equation}
and the time constants $\tau_a, \tau_r$ and $\tau_h$ are given in seconds per dBFS. Note that the filter acts on the positive logarithmic quantity \emph{compression}, $C'_n = -G'_n$, instead of the actual gain reduction $G'$.

The attack and release trajectories are linear in the log domain, as can be seen in Fig.~\ref{fig:step_frindle_gain}. In the release phase the hold timer $H_n$ scales the release time constant up to it's final value. The goal of this is to get a smooth release characteristic with low distortion for low frequency content and a smooth transition from hold to release phase.
%================================
\begin{figure}[h]
\centerline{\subfile{\rootdir/figures/step_frindle_gain}}
\caption{Linear gain smoothing, as implemented in \cite{frindle1996implementation}, step and downstep response with $\tau_a=0.02$, $\tau_r=0.08$ and $\tau_h= 0.1$.}
\label{fig:step_frindle_gain}
\end{figure}
%================================
\subsubsection{Smooth Decoupled and Branching Filters}
In \cite{reiss2012tutorial} the gain smoothing is referred to as \emph{level detection}, although it is a \emph{smoothing filter} applied to the calculated compression $C'_n$. Two different implementations are proposed, here named in accordance with \cite{reiss2012tutorial}.
\begin{itemize}
\item{Smooth decoupled filter}
\item{Smooth branching filter}
\end{itemize}
The difference equation of the \emph{smooth decoupled filter} is
%================================
\begin{equation}\label{eq:smooth_decoupled_det}
\begin{split}
Z_n &= \begin{cases}
   C'_n								& C'_n > Z_{n-1} \\
    \alpha_{r} C'_n + (1-\alpha_{r}) Z_{n-1} 	& C'_n \leq Z_{n-1}
\end{cases} \\
C_n &= \alpha_{a} Z_n + (1-\alpha_{a}) C_{n-1}
\end{split}
\end{equation}
%================================
which result in a smooth transition between the attack and release phase without discontinuities. The name is poorly chosen since it in fact couples the attack and release parameters in the release phase. In \cite{reiss2012tutorial} it is derived from it's analog counterpart, hence the name, and it is shown that the actual release time is $\approx \tau_a + \tau_r$.

To obtain a filter with attack and release parameters independent of each other the \emph{smooth branching filter}, same as described in section \ref{adaptive_filter} above, is suggested, here operating in the logarithmic domain where 
%================================
\begin{equation}\label{eq:smooth_branching_det}
C_n = \begin{cases}
    \alpha_{a} C'_n + (1-\alpha_{a}) C_{n-1} 	& C'_n > C_{n-1} \\
    \alpha_{r} C'_n + (1-\alpha_{r}) C_{n-1} 	& C'_n \leq C_{n-1}.
\end{cases}
\end{equation}
%================================
The step and down-step response in Fig.~\ref{fig:step_reiss_filter} shows the different release trajectories of the smooth decoupled and the smooth branching filter.%================================
\begin{figure}
\centerline{\subfile{\rootdir/figures/step_reiss_filter}}
\caption{Step and downstep response of the smooth decoupled and smooth branching filter, with $\tau_a = 50$ ms, $\tau_r = 150$ ms}
\label{fig:step_reiss_filter}
\end{figure}
%================================

Observe the similarities between the branching filter described in section \ref{peak_detection} and the \emph{smooth} branching filter described above. The latter was named \emph{smooth} in \cite{reiss2012tutorial} for it's return to signal behaviour in the release phase, in contrast to the return to zero behaviour of the former.
\end{document}