\documentclass[../main2.tex]{subfiles}
%if compiling standalone, rootdir wil be previous folder,
%if compiling main document, rootdir will already be set by main file
\providecommand{\rootdir}{..}

\begin{document}
%================================
\paragraph{(1) McN84 Peak Detection}
The design proposed in \cite{mcnally1984dynamic} consists of either a peak or RMS level detector. In this study the peak detector is tested. Since this algorithm is presented in both \cite{zolzer2008digital} and \cite{dafx11} it is the first DDRC to be evaluated. The design is summarised in Table~\ref{tab:mcn84}.
\begin{table}[h]
\begin{center}
\caption{McN84 Peak Detection}
\label{tab:mcn84}
\begin{tabular}{| l | l |}
	\hline
	Component & Type \\ \hline
	Peak detection & Analog peak detector, Eq.~\eqref{eq:analog_det} \\
	Gain computer & Standard, Hard knee, $W=0$, Eq.~\eqref{eq:gain} \\
	Gain smoothing & Linear branching. Eq.~\eqref{eq:mcnally_gain_smoothing} \\
	\hline
\end{tabular}
\end{center}
\end{table}
\paragraph{(2) GMR12 Smooth Branching and (3) Smooth Decoupled Peak}
The scientific comparison and tutorial of DDRC design conducted in \cite{reiss2012tutorial} resulted in the recommendation of two peak detecting compressors summarised in Table~\ref{tab:gmr12_decoupled} and~\ref{tab:gmr12_branching}. Since these are the latest proposed designs, they were chosen as the second and third tested DDRCs.
\begin{table}[h]
\begin{center}
\caption{GMR12 Smooth branching}
\label{tab:gmr12_branching}
\begin{tabular}{| l | l |}
	\hline
	Component & Type \\ \hline
	Peak detection & Full wave rectification, section.~\ref{full_wave_rect} \\
	Gain computer & Standard. Eq.~\eqref{eq:gain} \\
	Gain smoothing & Smooth branching. Eq.~\eqref{eq:smooth_branching_det} \\
	\hline
\end{tabular}
\end{center}
\end{table}

\begin{table}[h]
\begin{center}
\caption{GMR12 Smooth decoupled}
\label{tab:gmr12_decoupled}
\begin{tabular}{| l | l |}
	\hline
	Component & Type \\ \hline
	Peak detection & Full wave rectification, section.~\ref{full_wave_rect} \\
	Gain computer & Standard. Eq.~\eqref{eq:gain} \\
	Gain smoothing & Smooth decoupled. Eq.~\eqref{eq:smooth_decoupled_det} \\
	\hline
\end{tabular}
\end{center}
\end{table}

\paragraph{(4) FE96}
In \cite{frindle1996implementation} an interesting design is proposed. Since this is the only publication mentioning up-sampling as a method for reducing distortion and oscillating errors it is chosen as the fourth DDRC for evaluation. A summary of the design is seen in Table~\ref{tab:fe96}.
\begin{table}[h]
\begin{center}
\caption{FE96}
\label{tab:fe96}
\begin{tabular}{| l | l |}
	\hline
	Component & Type \\ \hline
	Peak detection & Upsampling peak detector, section~\ref{theory_upsampling_peak_det} \\
	Gain computer & Standard. Eq.~\eqref{eq:gain} \\
	Gain smoothing & Linear in log domain. Eq.~\eqref{eq:frindle_gainsmooth} \\
	\hline
\end{tabular}
\end{center}
\end{table}
\end{document}