\documentclass[../main2.tex]{subfiles}
%if compiling standalone, rootdir wil be previous folder,
%if compiling main document, rootdir will already be set by main file
\providecommand{\rootdir}{..}

\begin{document}

\subsection{Peak Level Detection}

\subsubsection{Parameter Optimisation}
The frequency of the fine structure as defined in Eq.~\eqref{eq:eq_test_signal} was set to $f_c = 1000/ \pi$ Hz and the envelope modulation frequency was set to $f_m = 2$ Hz. The sampling frequency was 1000 Hz. The parameters for each peak detector was optimised, and their optimised values can be seen in Tab.~\ref{tab:peak_det_opt_params} together with the resulting normalised euclidian distance to ideal. The resulting envelopes can be seen in Fig.~\ref{fig:peak_det_opt_env}. The envelopes are offset in relation to each other for clarity, with the ideal envelope dotted for each offset. The sampling frequency was set to $f_s = 1000$.

\begin{table}[h]
\begin{center}
\caption{Optimised parameters for the various peak detectors, $f_c=1000/\pi $, $f_m=2 $}
\label{tab:peak_det_opt_params}
\caption*{(a) Attack and release peak detectors}
\begin{tabular}{| l | c c c | c |}
	\hline
	Detector 	& $\tau_\text{a}$ [ms] & $\tau_\text{r}$ [ms] & $d$ [samples] & error [1/sample]\\
	\hline
	
	Analog 			& 0 			& 171 	& 5		& 0.0263	\\ 
	Branching smooth 	& 0	 		& 66 	& 5		& 0.0233	\\ 
	Cascaded smooth	& 5			& 66 	& 10		& 0.0206	\\
	\hline
\end{tabular}
\end{center}

\begin{center}
\caption*{(b) Windowed peak detectors}
\label{tab:peak_det_instatt_opt_params}
 \begin{tabular}{| l | c c c | c |}
	\hline
	Detector & $w$ [samples] & $\tau_\text{av}$ [ms] & $d$ [samples] & error [1/sample] \\
	\hline
	Win. Max		& 21		& -		& 10		& 0.0174	\\ 
	Win. Max Filt.	& 20		& 11		& 10		& 0.0073	\\
	\hline
\end{tabular}
\end{center}

\end{table}

\begin{figure}[h]
\centerline{\subfile{\rootdir/figures/peak_det_opt_env}}
\caption{Peak detectors optimised envelope}
\label{fig:peak_det_opt_env}
\end{figure}

\subsubsection{Carrier Frequency Dependence}
The parameters were fixed from the previous optimisation, and the carrier frequency swept from 20 Hz to $f_s/2$ Hz as described in Sec.~\ref{sec:method}. The result can be seen in Fig.~\ref{fig:peak_det_opt_env_fc_dep}.

\begin{figure}[h]
\centerline{\subfile{\rootdir/figures/peak_det_opt_env_fc_dep}}
\caption{Peak detectors optimised, $f_c$ dependence}
\label{fig:peak_det_opt_env_fc_dep}
\end{figure}

\subsubsection{Theoretical Benefit of Upsampling}
The test signal was generated at 8 times the sample rate, $f_s = 8000$, and the same test as above was run again. The parameters values was optimised again yielding the values in Tab.~\ref{tab:peak_det_opt_params_up}. The result can be seen in Fig.~\ref{peak_det_opt_env_th_up} and Fig.~\ref{fig:peak_det_opt_env_fc_dep_th_up}.
The envelopes have now become extremely smooth, and the carrier frequencies that evenly divides the sampling frequency is considerably less problematic. The windowed detectors however get a rapidly increasing error as the carrier frequency drops below the frequency the parameters was optimised for.

In practice, the up-sampled signal would have to be interpolated by an interpolating filter, and then down-sampled again after the level detector or later in the side-chain. This will probably downgrade the results seen here.\todo{discussion?}

\begin{table}[h]
\begin{center}
\caption{Optimised parameters for the various peak detectors, $f_c=1000/\pi $, $f_m=2 $}
\label{tab:peak_det_opt_params_up}
\caption*{(a) Attack and release peak detectors}
\begin{tabular}{| l | c c c | c |}
	\hline
	Detector 	& $\tau_\text{a}$ [ms] & $\tau_\text{r}$ [ms] & $d$ [samples] & error [1/sample]\\
	\hline
	
	Analog 			& 0 			& 141 	& 8		& 0.0067	\\ 
	Branching smooth 	& 0	 		& 52 	& 8		& 0.0067	\\ 
	Cascaded smooth	& 1.8		& 52 	& 22		& 0.0056	\\
	\hline
\end{tabular}
\end{center}

\begin{center}
\caption*{(b) Windowed peak detectors}
\label{tab:peak_det_instatt_opt_params}
 \begin{tabular}{| l | c c c | c |}
	\hline
	Detector & $w$ [samples] & $\tau_\text{av}$ [ms] & $d$ [samples] & error [1/sample] \\
	\hline
	Win. Max		& 33		& -		& 16		& 0.0034	\\ 
	Win. Max Filt.	& 33		& 42		& 49		& 0.0017	\\
	\hline
\end{tabular}
\end{center}
\end{table}

\begin{figure}[h]
\centerline{\subfile{\rootdir/figures/peak_det_opt_env_th_up}}
\caption{Upsampled peak detectors optimised envelope}
\label{fig:peak_det_opt_env_th_up}
\end{figure}

\begin{figure}[h]
\centerline{\subfile{\rootdir/figures/peak_det_opt_env_fc_dep_th_up}}
\caption{Upsampled peak detectors optimised, $f_c$ dependence}
\label{fig:peak_det_opt_env_fc_dep_th_up}
\end{figure}



\subsection{RMS Level Detection}


\end{document}