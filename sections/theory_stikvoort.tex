\documentclass[../main2.tex]{subfiles}
%if compiling standalone, rootdir wil be previous folder,
%if compiling main document, rootdir will already be set by main file
\providecommand{\rootdir}{..}

\begin{document}
\subsection{Stikvoort 1986}
\FloatBarrier
Although the original paper \cite{stikvoort1986digital} describes a stereo compressor, a mono version is presented here as we are mainly interested in the side chain. The side chain is composed of a peak detector, release effect, a piecewise linear compression curve in the decibel domain, and a final smoothing of the gain with attack effect \todo{stikvoort calls this "loop filter" but the meaning is inclear}, see Fig.~\ref{fig:block_stikvoort}.

\begin{figure}[h]
\centerline{\subfile{\rootdir/figures/block_stikvoort}}
\caption{Stikvoort feed forward topology}
\label{fig:block_stikvoort}
\end{figure}

\subsubsection{Envelope Detection with Release Effect}
Level detection is performed by full wave rectification, that is, the absolute value of the input signal is taken as level. An envelope is created from the level by a third-order filter defined as

\begin{equation}\label{eq:stikvoort_release}
\begin{split}
u_k &= \text{max}(x_k, \alpha_{\text{rel}} u_{k-1}) \\
v_k &= \text{max}(x_k, \alpha_{\text{rel}} v_{k-1} + (1-\alpha_{\text{rel}}) u_k) \\
y_k &= \text{max}(x_k, \alpha_{\text{rel}} y_{k-1} + (1-\alpha_{\text{rel}}) v_k),\\
\end{split}
\end{equation}
where $x_k$ is the level and $y_k$ the calculated signal envelope.  The time constant $\alpha_{\text{rel}}$ is defined as
\begin{equation}
\alpha_\text{rel} = e^{-1/\tau_0 f_s}
\end{equation}
where the downstep has fallen to 63.2 \% of it's final value after approximately $3.25 \tau_0$. The filter has instantaneous attack, and functions as both peak-hold and smooth release effect, see Fig.~\ref{fig:step_stikvoort_release}.

\subsubsection{Gain computer}
The gain computer is the standard one defined in \eqref{eq:XXXX}, but with hard knee only.

\subsubsection{Gain smoothing}
At the end of the side chain, the calculated gain is smoothed by a forth-order transitional Butterworth-Thomson low-pass filter, implemented in two steps as

\begin{equation}\label{eq:stikvoort_attack}
\begin{split}
u_k &= b_0 x_k + b_1 x_{k-1} + b_2 x_{k-2} + a_1 u_{k-1} + a_2 u_{k-2}\\
y_k &= d_0 u_k + d_1 u_{k-1} + d_2 u_{k-2} + c_1 y_{k-1} + c_2 y_{k-2}.
\end{split}
\end{equation}
With the coefficients in Tab.~\ref{tab:coeff_stikvoort_attack} \todo{$c_2$ has a minus sign, print error in original!} the group delay is about 35 ms (in the low frequency limit, as the filter is not linear phase) and cut off frequency 8.5 Hz for a sample rate of 44.100 Hz. The step response can be seen in Fig.~\ref{fig:step_stikvoort_attack}. \todo{A slight static error can be seen, probably due to insufficient accuracy in the coefficents given in stikvoorts original paper.}

\begin{table}[h]
\begin{center}
 \begin{tabular}{ c l | c l}	
    \hline
    $b_0$ & $0.1308 \cdot 10^{-5}$ &              &                          \\ \hline
    $b_1$ & $0.2616 \cdot 10^{-5}$ & $a_1$ & 1.9967655     \\ \hline
    $b_2$ & $0.1308 \cdot 10^{-5}$ & $a_2$ & -0.99677073  \\ \hline \hline

    $d_0$ & $0.9464  \cdot 10^{-6}$   &         &                             \\ \hline
    $d_1$ & $0.18928 \cdot 10^{-5}$ & $c_1$ & 1.9962282     \\ \hline
    $d_2$ & $0.9464  \cdot 10^{-6}$  & $c_2$ & -0.99623194  \\ \hline
\end{tabular}
\caption{Coefficents for Stikvoort gain smoothing}
\label{tab:coeff_stikvoort_attack}
\end{center}
\end{table}

\FloatBarrier
%=========================
\begin{figure}
\centerline{\subfile{\rootdir/figures/step_stikvoort_release}}
\caption{Step and downstep response of third order release filter as defined in Eq.~\eqref{eq:stikvoort_release}, with $\tau_0 = 150 \text{ms}$.}
\label{fig:step_stikvoort_release}
\end{figure}
%=========================

%=========================
\begin{figure}
\centerline{\subfile{\rootdir/figures/step_stikvoort_attack}}
\caption{Step and downstep response of attack filter as defined in Eq.~\eqref{eq:stikvoort_attack}, with coefficients as in Tab.~\ref{tab:stikvoort_attack}.}
\label{fig:step_stikvoort_attack}
\end{figure}
%=========================


\end{document}