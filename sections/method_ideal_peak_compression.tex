\documentclass[../main2.tex]{subfiles}
%if compiling standalone, rootdir wil be previous folder,
%if compiling main document, rootdir will already be set by main file
\providecommand{\rootdir}{..}

\begin{document}

%================================
\FloatBarrier
\subsection{Ideal Peak Level Compression}\label{method_ideal_peak_compression}
Given an input signal $x_n$, as before consisting of an envelope $e_n$ and a fine structure $f_n$ so that
\begin{equation}\label{eq:input_signal_env_finestruct}
x_n = e_n f_n,
\end{equation}
the dynamic range, Eq. \eqref{eq:s_peak}, can be expressed as
\begin{equation}
S_\text{pk} =A_\text{max} - A_\text{min} = 20 \logten \left(\frac{\max(e_n)}{\min(e_n)}\right) \dBFS
\end{equation}
where $A_\text{max} = \max(A_{\text{pk},n})$ and $A_\text{min} =  \min(A_{\text{pk},n})$.
 
In this thesis \emph{ideal compression} is defined as the processes of mapping all or a part of this dynamic range to a smaller range, without changing the fine structure. This mapping is further defined to be linear in the logarithmic domain and with a threshold below which the signal level is unchanged. For the input signal in Eq.\eqref{eq:input_signal_env_finestruct}, the envelope of the ideal output signal can thus be described by the following equation
\begin{equation}\label{eq:dynamic_range_mapping}
O_n =
\begin{cases}
	K E_n + M 					& E_n \geq T'  \\
	E_n							& \text{otherwise},
\end{cases}
\end{equation}
where $K$, $M$ and $T'$ are constants defining the wanted mapping. This clearly corresponds to the static curve in Fig.\ref{fig:typical_static_detailed} described by Eq.\eqref{eq:gaincomp} with $W=0$ and the following change of variables:
\begin{equation}
\begin{split}
K &= R^{-1}, \\
M &= (R^{-1}-1)T,\\
T' &= T
\end{split}
\end{equation}
where $T$ is the threshold and $R$ the ratio.

The ideal output signal $y_{I,n}$ can then be written as
\begin{equation}\label{eq:ideal_output}
y_{I,n} = o_n f_n
\end{equation}
where the fine-structure $f_n$ is the same as in Eq.\eqref{eq:input_signal_env_finestruct} and $o_n$ is the mapped output envelope in linear domain. It is shown below that this definition of ideal compression yields optimal performance in terms of the proposed metrics in previous work, with $\FES = 1$, $\THDF=0$\%, and $\ECR=R$.
% ========================================
\subsubsection{FES of Ideal Compression}
Given that the lower bound of the dynamic range of the input signal $A_\text{min}$ is above the defined threshold $T$, the output envelope level determined by Eq.\eqref{eq:dynamic_range_mapping} is reduced to
\begin{equation}
O_n = K E_n + M.
\end{equation}
In \cite{XXXX} it is proven that the Pearson correlation coefficient is equal to 1 for linearly dependent variables. Thus, the definition of ideal compression above yields $r_\text{FES} = 1$.

If $A_\text{min} < T$ this is not the case. As the dynamic range of the input signal spans the knee of the static compression curve, the envelope will inevitably be distorted and $r_\text{FES} < 1$. However, this distortion is not regarded as an unwanted artefact in this thesis since both $T$ and $R$ are regarded as user parameters defining the wanted dynamic range mapping. Thus FES is only considered a valid metric of compressor performance for the case when $A_\text{min} \geq T$.\footnote{Of course, when used to quantify the effect of compression on the envelope of an audio signal, as is done in \cite{XXXX} and \cite{XXXX}, there is no need to put constraints on $A_\text{min}$ or $T$. It is merely when used to evaluate compressor \emph{performance} this restriction is necessary.}

% ========================================
\subsubsection{THD of Ideal Compression}
The input signal used to measure THD as defined in \ref{theory_metrics_thd} is a pure sine wave with amplitude $a$ and frequency $f$:
\begin{equation}
x_n = a \sin( 2 \pi f n T_s ).
\end{equation}
Identifying the terms in Eq.\eqref{eq:input_signal_env_finestruct}, we have
\begin{equation}
\begin{split}
e_n &= a \\
f_n &= \sin( 2 \pi f n T_s ).
\end{split}
\end{equation}
The constant amplitude $a$ gives $A_\text{min} = A_\text{max} = E_n = 20 \logten a \dBFS$ and given that $A_\text{min} \geq T$, the output level of ideal compression as defined by Eq.~\eqref{eq:eq_dynamic_range_mapping} is given by
\begin{equation}
\begin{split}
O_n &= K (20 \logten a) + M.
\end{split}
\end{equation}
The ideal output signal as defined by Eq.~\eqref{eq:ideal_output} can then be written as
\begin{equation}
y_n = o_n f_n = 10^{Ma^K/20} \sin(2 \pi f n T_s).
\end{equation}
Clearly, in the absence of higher harmonics in the output,
\begin{equation}
\THDF = \frac{\sqrt{0^2}}{10^{Ma^K/20}} = 0.
\end{equation}
If $A_\text{min} < T$ no compression is applied at all and $y_n = x_n$, again yielding $\THDF = 0$.

% ========================================
\subsubsection{ECR of Ideal Compression}
Assuming $A_\text{min} \geq T$ the lower and upper bound of the output level is given by Eq.\eqref{eq:dynamic_range_mapping} as
\begin{equation}
\begin{split}
O_\text{max} &= K A_\text{max} + M\\
O_\text{min} &= K A_\text{min} + M.
\end{split}
\end{equation}
The output signal's dynamic range $S_\text{pk, out}$ is thus
\begin{equation}
S_\text{pk,out} = O_\text{max} - O_\text{min} = K (A_\text{max} - A_\text{min}).\end{equation}
ECR can then be calculated as
\begin{equation}
ECR = \frac{S_\text{pk,in}}{S_\text{pk,out}} = \frac{A_\text{max} - A_\text{min}}{K(A_\text{max} - A_\text{min})} = K^{-1} = R
\end{equation}
Thus, the effective compression ratio is equal to the specified $R$.

Observe that, as was the case for FES, ECR will only be considered a valid metric of compressor performance for the case $A_\text{min} \geq T$.

\end{document}
