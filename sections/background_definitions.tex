\documentclass[../main2.tex]{subfiles}
%if compiling standalone, rootdir wil be previous folder,
%if compiling main document, rootdir will already be set by main file
\providecommand{\rootdir}{..}

\begin{document}

\subsection{Definitions}\label{theory_definitions}
A DDRC maps the dynamic range of a signal to a smaller, compressed, range. Let $x_n$ be the input signal, $y_n$ the output signal and $g_n$ the gain factor. In a compressor $0<g_n\leq 1$. The input signal, in general delayed, is multiplied by the gain factor, see Fig.~\ref{fig:block_gain},
%================================
\begin{figure}
\centerline{\subfile{\rootdir/figures/block_gain}}
\caption{Input signal multiplied by time-variant gain factor}
\label{fig:block_gain}
\end{figure}
%================================
\begin{align}
y_n = g_nx_{n-d_{la}}.
\label{eq:gainfactor}
\end{align}
%================================
The task is then to find the time-variant $g_n$ corresponding to the desired behaviour of the DDRC. In order to do so, the following parameters, corresponding to static and temporal characteristics, are introduced.
%================================
\begin{itemize}
\item{Static}
	\begin{itemize}
	\item \textbf{Threshold} $T$ - The defined limit above which compression is applied.
	\item \textbf{Ratio} $R$ - The input/output ratio above the threshold level.
	\item \textbf{Knee width}  $W$ - Controls the sharpness of the knee.
	\item \textbf{Make-up gain}  $M$ - The amount of gain applied in the final step to balance the perceived loudness of the output signal with the input signal.
\end{itemize}
\item{Temporal}
	\begin{itemize}
	\item \textbf{Attack time} $\tau_{a}$ - Determines how quickly the compression ratio is applied.
	\item \textbf{Release time} $\tau_{r}$ - Determines how quickly the compression is released as the input signal drops below the threshold level.
	\item \textbf{Look-ahead} $d_{la}$ - Difference in delay between direct signal path and side chain. 
	\end{itemize}
\end{itemize}

\end{document}