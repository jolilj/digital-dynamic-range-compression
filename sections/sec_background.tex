\documentclass[../main2.tex]{subfiles}
%if compiling standalone, rootdir wil be previous folder,
%if compiling main document, rootdir will already be set by main file
\providecommand{\rootdir}{..}

\begin{document}

\subsection{Earlier work}

\begin{itemize}
\item McNally
\item Zoelzer
\item DAFX (compilation of the above (and others? who?), adds anything new?)
\item RMS = Abs (who?)
\item Fred Floru: Log = Lin
\item Reiss
\end{itemize}

\subsection{Reiss}
In Reiss, the comparisons are made with all parameters equal, without knee width and for the most discriminating case, namely the square wave step. A valid approach to find the qualitative \emph{differences}. In this paper we instead focus on the \emph{similarities} between different topologies, and make quantitative comparisons complemented with listening tests.

\subsection{Theory}
The dynamic range compressor is a complex nonlinear time dependent system where the the amplitude peaks of a signal is suppressed, reducing its dynamic range. In figure \ref{fig:xy-graph} the eligible behaviour of a simple compression process is illustrated. The x- and y-axis represents the logarithmic amplitude of input and output signal respectively. As the input signal rise above the threshold level, compression is applied. Logarithmic scale is used since sound amplitude is, by standard, expressed in the logarithmic unit decibel.

In practice, however, is is not desirable to apply instant compression since it introduces distortion\cite{giannoullis}. We thus want to smooth out the compression applied over time. This is illustrated in figure \ref{fig:envelope-graph}.

\begin{figure}[ht]
\captionsetup{justification=centering}
\begin{minipage}[t]{.5\textwidth}
 \centering
\subfile{../figures/fig_gain_comp}
\caption{Output signal vs input signal.} 
\label{fig:xy-graph}
\end{minipage}%
\begin{minipage}[t]{.5\textwidth}
\centering
\subfile{../figures/fig_envelope}
\caption{Input and output signal plotted over time. } 
\label{fig:envelope-graph}
\end{minipage}
\end{figure}

In order to analyse the compression process we define the following parameters
\begin{itemize}
\item{\textbf{Threshold} - The defined limit above which compression is applied }
\item{\textbf{Ratio} - The input/output ratio above the threshold level. Determines the amount of compression}
\item{\textbf{Attack time} - Determines how quickly the compression ratio is applied.}
\item{\textbf{Release time} - Determines how quickly the compression is released as the input signal drops below the threshold level.}
\item{\textbf{Knee width} - Controls the sharpness of the knee, see Fig. \ref{fig:xy-graph}.}
\end{itemize}



\subsection{Gain Computer}
The heart in any compressor design is where the gain reduction is applied. Here the amplitude of the input signal is reduced by the gain factor $g_n$, see Fig. \ref{fig:gain-step-blockdiagram}. Denoting input and output signal $x_n$ and $y_n$ respectively, we have
\begin{align}
y_n = x_n g_n.
\label{eq:inout}
\end{align}
To derive the gain factor, $g_n$ we begin by examining the eligible behaviour of the compressor according to figure \ref{fig:xy-graph}. Denoting the threshold level $T$ and the ratio $R$, we get
\begin{equation} \label{eq:gaincomp}
Y = \left\{ 
  \begin{array}{l l}
    T+ \dfrac{X-T}{R} & \quad \text{if $X > T$}\\
    X & \quad \text{otherwise.}
  \end{array} \right.
\end{equation}
The component generating the gain factor is defined as the \emph{gain computer}. We now face two different approaches: feeding the gain computer with the uncompressed input or the compressed output signal resulting in a feed-forward or feedback system, see Fig. \ref{fig:feedforward-blockdiagram} and \ref{fig:feedback-blockdiagram}. 
\begin{figure}[ht]
 \centering
\subfile{../figures/fig_feedforward_blockdiagram}
\caption{Block diagram of a feed-forward design.}
\label{fig:feedforward-blockdiagram}
\end{figure}
\begin{figure}[ht]
\centering
\subfile{../figures/fig_feedback_blockdiagram}
\caption{Block diagram of feedback design.} 
\label{fig:feedback-blockdiagram}
\end{figure}
\\ To derive $g_n$ we begin with the relationship between the input and output signal.
\begin{align}
y_n &= x_ng_n   \\
\log|y_n| & = \log|x_n| + \log|g_n|   \\
Y &= X + G \label{eq:cv}
\end{align}
\\ where the logarithmic variables are denoted by capital letters. For a compressor the output is never amplified, hence $g_n \leq 1$. In \cite{giannoullis} it is shown that a feedback compressor cannot function as a perfect limiter. Furthermore it lacks the ability for look-ahead, see TODO refer to section explaining look ahead. Therefor we leave the feedback compressor design aside. 

In a feed-forward design the logarithmic input signal, $X$,  is inserted into the gain computer, thus equation (\ref{eq:cv}) and (\ref{eq:gaincomp}) yields
\begin{equation}
X+G = \left\{ 
  \begin{array}{l l}
    T+ \dfrac{X-T}{R} & \quad \text{if $X > T$}\\
    X & \quad \text{otherwise.}
  \end{array} \right.
\end{equation}
\\Solving for $G \rm \Rightarrow$
\begin{equation} \label{eq:c}
G = \left\{ 
  \begin{array}{l l}
    \left(R^{-1}-1\right)(X-T)& \quad \text{if $X > T$}\\
    0 & \quad \text{otherwise.}
  \end{array} \right.
\end{equation}
\begin{align}
G = (R^{-1}-1)\cdot \max\left(Y-T,0\right).
\end{align}

\subsection{Level Detection}
So far, the gain reduction is calculated and applied instantly and thus changes the wave shape of the input signal introducing new frequencies, i.e. distortion, see Fig. \ref{fig:instant_comp}. 
\begin{figure}[ht]
\centering
\subfile{../figures/fig_instant_comp}
\caption{Instant compression changes the shape of the wave and thus introduces distortion.} 
\label{fig:instant_comp}
\end{figure}
There are various methods to avoid this.
\subsubsection{RMS Detection}
The root mean square taken over a window of $M$ samples is a smoothed average of a signal and corresponds well to perceived loudness\cite{giannoullis}. It is defined as 
\begin{align}
RMS = \frac{1}{M}\sum_{m=-M/2}^{M/2}\;x_{n-m}^2.
\end{align}
This is not, however, practical in real-time implementations since it introduces a delay of $M/2$ samples. \todo{DAFX?}Another approach, described in DafX?...,  is instead to apply an infinite impulse response(IIR), low pass filter. Such filter will smooth out the gain reduction, leaving the wave shape intact. The filter can operate on either the squared signal or the absolute value of the input signal. \todo{generally equivalent? } In REF... it was shown that these both produce generally equivalent behaviour apart from a scaling of the filter time constants. As in \cite{gianoullis}, we will focus on the latter one.  

\subsubsection{IIR Low Pass Filter}
The purpose of the level detection filter is 
\begin{itemize}
\item{Smooth out}
\end{itemize}
The desired step response of the level detection filter is illustrated in  A digital one-pole IIR filter can be implemented as\cite{giannoulis}.
\begin{align}
y_n = \alpha x_{n} + (1-\alpha)y_{n-1}
\end{align}
\begin{figure}[ht]
\centering
\subfile{../figures/fig_attack_release}
\caption{Desired peak detector step response, attack and release phase} 
\label{fig:attack-release-graph}
\end{figure}
\emph{Derivation of the below expression here!}

\begin{equation}
y_k = \left\{
  \begin{array}{ll}
    \alpha_{att} x_k + (1-\alpha_{att})y_{k-1} & \text{if }  x_k > y_{k-1} \\
    \alpha_{rel} x_k + (1-\alpha_{rel})y_{k-1} & \text{otherwise} 
  \end{array}
\right.
\end{equation}
where
\begin{equation}
\begin{array}{lr}
\alpha_{att} = 1-e^{1/f_s \tau_{att}}, & \alpha_{rel} = 1-e^{1/f_s \tau_{rel}}
\end{array}
\end{equation}
\subsubsection{Peak, RMS and Crest factor}
We can further smooth the peak detector envelope by taking the RMS of the input signal. \emph{something about number of samples to take mean over}

This will also allow us to calculate the \emph{Crest factor} defined as
\begin{equation}
y_{crest} = \frac{y_{peak}}{y_{RMS}}
\end{equation}
which can be used to detect transients.

\subsubsection{Log or linear domain}
Assuming that we choose the exponential function as our attack and release trajectory, we can either implement this in the logarithmic or the linear domain. In the logarithmic domain, we will get an exponential temporal evolution of the compression in dB (\emph{need to have text about log and dB somewhere explaining this...}.) On the other hand, if the function is exponential in the linear domain, it will be linear in the logarithmic domain, which presumably will not be as smooth to the human ear (but worth investigating!).

\subsubsection{Placement in block diagram}
Furthermore, we can choose to place the peak detector either before or after the gain computer. Placed before the gain computer, we will feed the gain computer not with the input signal, but with our computed envelope. \emph{This will lead to slow attack, since the attack time will begin below the threshold etc...}
Another option is to place it \emph{after} the gain computer. This way the gain computer acts on the input signal, and we feed the peak detector with the control voltage instead of the raw input signal, calculating a smooth envelope on this instead. \emph{maybe some graphs and block diagrams depicting this difference...}

\emph{TODO: Investigate how this affects the transfer functions etc... If it was a linear system this would not matter? Maybe interesting to do some math on the importance of the order of components in this nonlinear system}

\end{document)
