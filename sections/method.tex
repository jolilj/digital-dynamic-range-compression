\documentclass[../main2.tex]{subfiles}
\providecommand{\rootdir}{..}
\begin{document}

\section{Method}\label{method}
By combining the various peak and RMS detectors with the gain smoothing filters treated in section~\ref{theory_DDRC}, a large amount of side-chain configurations are possible. Due to lack of time and computational resources, a selection of the described variants had to be made. To narrow the scope of this thesis, four different compressor algorithms was selected and evaluated.

Two tests was conducted. First of all, the importance of parameter settings when comparing compressor characteristics was investigated. This process is described in section \ref{method_param_opt}. Following this, the four chosen compressor algorithms was compared and evaluated in more detail. The parameter settings were optimised for a specific test signal to achieve a measure of how close to ideal the chosen DDRCs were able to perform. This process is described in section~\ref{method_peak_compressors}.

\subsection{Implementation} \label{method_implementation}
The compressors were implemented and tested in MATLAB\textsuperscript{\textregistered}. The parameter optimisation was conducted using the built in function \emph{fminsearch}\footnote{For documentation see \cite{fminsearch}} which uses the Nelder-Mead simplex algorithm. 

The code for the implemented DDRC's is available in appendix~\ref{appendix_code}.

\subfile{\rootdir/sections/method_param_opt}
\subfile{\rootdir/sections/method_peak_compressors}
%\subfile{\rootdir/sections/method_chosen_DDRC}

%\subfile{\rootdir/sections/method_rms_compressors}

%upsampling?


\subfile{\rootdir/sections/method_implementation}

\end{document}
