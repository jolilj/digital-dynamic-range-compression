\documentclass[../main2.tex]{subfiles}
\providecommand{\rootdir}{..}
\begin{document}

\section{Method}\label{method}
\todo{introduction to the method}
Since the side-chain is a sequence of non-linear components that does not obey the superposition principle, an investigation of both their individual performance as well as their interaction with each other is justified. First, the peak- and RMS level detectors was evaluated separately. Secondly, complete compressor designs was composed by combining different level detection and gain smoothing components and evaluated.

The method for evaluating the level detectors is outlined in sections \ref{method_peak_detectors} and \ref{method_rms_detectors}. The method for evaluating complete DDRC designs is derived in sections \ref{method_ideal_compression}, \ref{method_distance_to_ideal} and \ref{method_compressor_thd}.

When comparing both level detectors and the complete DDRC designs, the parameter settings was optimised numerically. This approach is motivated and described in section \ref{method_param_opt}.

\subfile{\rootdir/sections/method_peak_detectors}
\subfile{\rootdir/sections/method_rms_detectors}
%upsampling?



\subfile{\rootdir/sections/method_peak_compressors}
\subfile{\rootdir/sections/method_rms_compressors}
%upsampling?

\subfile{\rootdir/sections/method_param_opt}

\end{document}