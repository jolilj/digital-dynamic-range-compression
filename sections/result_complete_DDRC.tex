\documentclass[../main2.tex]{subfiles}
%if compiling standalone, rootdir wil be previous folder,
%if compiling main document, rootdir will already be set by main file
\providecommand{\rootdir}{..}

\begin{document}

\subsection{Complete DDRC algorithms}

\subsubsection{Distance to Ideal}
The optimised parameter values are presented in Tab.~\ref{tab:complete_DDRC_opt_params} together with the resulting normalised euclidian distance to ideal output, $\EN$. There is no guarantee that the found parameter settings in Tab.~\ref{tab:complete_DDRC_opt_params} are at the global minimum of the cost function $\EN$. There may exist parameter settings other than those presented here that yield a smaller error. However, these results establishes an \emph{upper bound} for the distance to ideal for each compressor.

With this in mind, the key result presented here is that all tested compressors were able to approximate the ideal output within an error of less than $10^{-3}$ per sample ($-60 \dBFS$/sample). Furthermore, it was possible to achieve this with threshold and ratio very close to the defined goal compression parameters.

\begin{table}[h]
\small
\begin{center}
\caption{Optimised parameters for the various peak detectors. Test signal: $f_c=$~10,000~Hz, $f_m=2$ Hz. Goal compression: $T_\text{def}=-12 \dBFS$, $R_\text{def}=4$}
\label{tab:complete_DDRC_opt_params}
\subfile{\rootdir/tables/complete_DDRC_params}
\end{center}
\end{table}

The difference $y_{\text{I},n} - y_n$ can be seen plotted against time for each of the tested compressors in Fig.~\ref{fig:complete_DDRC_opt_diff}, vertically offset by 0.002 for clarity. The vertical lines indicate the range in where \EN was calculated in the optimisation. The best local minima was found for the McN84 compressor.

\begin{figure}[h]
\centerline{\subfile{\rootdir/figures/complete_DDRC_opt_diff}}
\caption{DDRCs optimised, difference $y_{\text{I},n} - y_n$ plotted against time vertically offset by 0.002 for clarity. The best local minima was found for compressor (1) McN84.}
\label{fig:complete_DDRC_opt_diff}
\end{figure}

The gain factors, $g_n$, can be seen plotted against time for each of the tested compressors in Fig.~\ref{complete_DDRC_opt_gain}, vertically offset by 0.01. The McN84 has small very fast ripples in gain factor, while the GMR12 decoupled design is smoother but with a small static error. GMR12 branching and FE96 oscillate considerably more around the ideal curve.

\begin{figure}
\captionsetup{justification=centering}
\begin{subfigure}{\linewidth}
\centering
\centerline{\subfile{\rootdir/figures/complete_DDRC_opt_gain_max}}
\caption{Optimised DDRC gain factors, zoomed in at the minimum gain reduction}
\end{subfigure}

\par\bigskip

\captionsetup{justification=centering}
\begin{subfigure}{\linewidth}
\centering
\centerline{\subfile{\rootdir/figures/complete_DDRC_opt_gain_min}}
\caption{Optimised DDRC gain factors, zoomed in at the maximum gain reduction}
\label{fig:complete_DDRC_opt_gain}
\end{subfigure}

\caption{Zoomed in versions of optimised DDRC gain factors, vertically offset by 0.01 for clarity}
\label{fig:peak_det_opt_env_zoom}
\end{figure}

Inspecting the optimised parameters in Table~\ref{tab:complete_DDRC_opt_params}, one sees that although the static parameters are similar between the compressors, the time constants vary. It is evident that these parameters  yield practically the same output, but for this test signal only. To compare the similarity of the compressors given another input, but the \emph{same} parameters, a step and down-step response was compressed. The results can be seen in Fig.~\ref{fig:complete_DDRC_opt_step_response}. Evidently, the compressors do approximate the ideal compression of the test signal well, but by completely different means. The step and down-step responses show \emph{completely} different characteristics, thus illustrating the non-linearity of the system. The \emph{mutual} difference in output between the DDRCs are highly dependent on the input.
\begin{figure}[h]
\centerline{\subfile{\rootdir/figures/complete_DDRC_opt_step_response}}
\caption{Step response of DDRC with optimised parameters for amplitude modulated sinewave with $f_m=2$ Hz and $f_c=10000$ Hz.}
\label{fig:complete_DDRC_opt_step_response}
\end{figure}

%======================
\subsubsection{Carrier Frequency Dependence}
Looking at the carrier frequency dependency in Fig.~\ref{fig:complete_DDRC_opt_fc_dep}, McN84 is the one suffering the most for low frequencies, followed by FE96. Both GMR designs are clearly more robust to lower carrier frequencies than they were optimised for. Worth noting is that the branching design manages to maintain a small error for low frequencies with only 1 ms of look-ahead. 

The McN84 and GMR12 designs have spikes in error at certain carrier frequencies. A closer inspection shows that the spikes are located at exactly the carrier frequencies that evenly divides the sampling frequency, as predicted in section \ref{full_wave_rect_det}. This problem is significantly reduced in FE96 with its up-sampling peak detector.

\begin{figure}[h]
\centerline{\subfile{\rootdir/figures/complete_DDRC_opt_fc_dep}}
\caption{DDRCs optimised, $f_c$ dependence}
\label{fig:complete_DDRC_opt_fc_dep}
\end{figure}

\end{document}