\documentclass[../main2.tex]{subfiles}
%if compiling standalone, rootdir wil be previous folder,
%if compiling main document, rootdir will already be set by main file
\providecommand{\rootdir}{..}

\begin{document}

\subsection{Complete DDRC algorithms}

\subsubsection{Parameter Optimisation}
The frequency of the fine structure as defined in Eq.~\eqref{eq:eq_test_signal} was set to $f_c = $~10,000~Hz and the envelope modulation frequency was set to $f_m = 2$ Hz. The sampling frequency was $fs = $~44,100 Hz. The parameters for each peak detector was optimised, and their optimised values can be seen in Tab.~\ref{tab:complete_DDRC_opt_params} together with the resulting normalised euclidian distance to ideal output. The resulting gain factors plotted against time can be seen in Fig.~\ref{fig:peak_det_opt_env}. The curves are offset in relation to each other for clarity, with the ideal gain factor dotted for each offset. 

\begin{table}[h]
\small
\begin{center}
\caption{Optimised parameters for the various peak detectors, $f_c=$~10,000~Hz, $f_m=2$ Hz.}
\label{tab:complete_DDRC_opt_params}
\subfile{\rootdir/tables/complete_DDRC_params}
\end{center}
\end{table}

\begin{figure}
\captionsetup{justification=centering}
\begin{subfigure}{\linewidth}
\centering
\centerline{\subfile{\rootdir/figures/complete_DDRC_opt_gain_max}}
\caption{Optimised DDRC gain factors, zoomed in at the minimum gain reduction}
\end{subfigure}

\par\bigskip

\captionsetup{justification=centering}
\begin{subfigure}{\linewidth}
\centering
\centerline{\subfile{\rootdir/figures/complete_DDRC_opt_gain_min}}
\caption{Optimised DDRC gain factors, zoomed in at the maximum gain reduction}
\end{subfigure}

\caption{Zoomed in versions of optimised DDRC gain factors, vertically offset by 0.01 for clarity}
\label{fig:peak_det_opt_env_zoom}
\end{figure}
%======================

\subsubsection{Carrier Frequency Dependence}
The parameters were fixed from the previous optimisation, and the carrier frequency swept from 20 Hz to 20,000 Hz as described in Sec.~\ref{sec:method}. The result can be seen in Fig.~\ref{fig:complete_DDRC_opt_fc_dep}.

\begin{figure}[h]
\centerline{\subfile{\rootdir/figures/complete_DDRC_opt_fc_dep}}
\caption{DDRCs optimised, $f_c$ dependence}
\label{fig:complete_DDRC_opt_fc_dep}
\end{figure}

\end{document}