\documentclass[../main2.tex]{subfiles}
%if compiling standalone, rootdir wil be previous folder,
%if compiling main document, rootdir will already be set by main file
\providecommand{\rootdir}{..}

\begin{document}

\section{Introduction}
The dynamic range compressor is a nonlinear time dependent system with memory, in the family of dynamic range controllers, that maps an input signal to a smaller dynamic range~\cite{dafx11}. Although the exact behaviour depends on both implementation and input signal, the typical static and dynamic properties of a dynamic range compressor (DRC) are illustrated in Fig.~\ref{fig:typical_static} and Fig.~\ref{fig:typical_envelope}.

\begin{figure}[ht]
\captionsetup{justification=centering}

\begin{minipage}[t]{.5\textwidth}
 \centering
\subfile{\rootdir/figures/typical_static}
\caption{Output amplitude vs input amplitude} 
\label{fig:typical_static}
\end{minipage}%
\begin{minipage}[t]{.5\textwidth}
\centering
\subfile{\rootdir/figures/typical_envelope}
\caption{Input and output signal plotted over time} 
\label{fig:typical_envelope}
\end{minipage}
\end{figure}

Various designs have been proposed for a digital dynamic range compressor (DDRC) in the literature and there exists an abundance of commercial implementations on the market.\footnote{For examples of commercial compressors, see this listing~\cite{commercial}.} However, the subjective nature of audio quality together with the inherent nonlinear behaviour of DRC make mathematical analysis and formal comparisons of different designs difficult.

Some studies have been made, outlining different possible implementations and comparing their respective characteristics. However, the main results are often restricted to a limited parameter set, and the most discriminating input signals.

In this thesis, we develop a generic model for feed forward type DDRCs based on previous work. \todo{others work, that is. Not our own. How to emphazise this?} With this model

\end{document}