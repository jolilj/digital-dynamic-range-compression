\documentclass[../main2.tex]{subfiles}
%if compiling standalone, rootdir wil be previous folder,
%if compiling main document, rootdir will already be set by main file
\providecommand{\rootdir}{..}

\begin{document}
\subsection{Discarding Evaluation of RMS Level Compressors}\label{method_rms_compressors}
The evaluation of the RMS Detector uses as ideal case the envelope divided by the crest factor. However, in here lies an ambiguity. The RMS is by definition the \emph{average} of the signal over a \emph{specified} time. That the signal has a well defined envelope \emph{does not} imply that the RMS of the signal should track that envelope as in the peak detection case. This makes the evaluation of the RMS detectors restricted to the case where the \emph{specified} time scale is of order $\sim T_c$ where $T_c$ is the period of the carrier frequency.

This makes it difficult to define ideal compression in the RMS detection case and the evaluation of RMS compressors are therefore left for future work. It would be possible to define the ideal compression as that of the peak case, with the envelope divided by the crest factor as in the evaluation of the RMS detectors, but as this is not uniquely defined as the ideal RMS, the results would be ambiguous for the overall compressor design.
\end{document}