\documentclass[../main2.tex]{subfiles}
%if compiling standalone, rootdir wil be previous folder,
%if compiling main document, rootdir will already be set by main file
\providecommand{\rootdir}{..}

\begin{document}
%================================
\subsection{Implementation} \label{method_implementation}
\subsubsection{Peak Detection}
As discussed in section \ref{dynamic_range}, the level detection stage is of great importance in constructing a DDRC motivating an evaluation of the various level detectors separated from the rest of the side-chain, as done in \cite{reiss2012tutorial}. The following detectors were tested with and without upsampled signal:
\begin{itemize}
\item{Analog Peak Detector} 
\item{Branching Smooth Peak Detector}
\item{Decoupled Smooth Peak Detector}
\item{Windowed Peak Max Detector}
\item{Windowed Peak Max Filtered Detector}
\end{itemize}
Results show, see section \ref{results} that the optimised parameters of the branching smooth and analog peak detector indicate instant attack, $\tau_a = 0$. These detectors were thus implemented and tested with instant attack.

\begin{itemize}
\item{Analog Instant Attack Peak Detector}
\item{Smooth Instant Attack Peak Detector}
\end{itemize}

\subsubsection{RMS Detection}

\subsubsection{Gain Smoothing}
The gain smoothing stage is considered to be used mainly for artistic purposes where the end user can tweak the attack/release parameters to achieve the preferred sound. The filters used for gain smoothing are for this reason not tested separately, although a few of them are the same filters as in the level detection, see especially \cite{reiss2012tutorial}.
\subsubsection{DDRC Design}
The following designs that have been recommended in previous work were tested, without upsampling:
\begin{itemize}
\item{\cite{mcnally1984dynamic} Analog peak detection and branching smooth gain}
\item{\cite{dafx11} Branching peak detection and branching smooth gain}
\item{\cite{reiss2012tutorial} Full rectification peak detection and branching smooth gain}
\item{\cite{reiss2012tutorial} Full rectification peak detection and decoupled smooth gain}
\item{\cite{frindle1996implementation} Upsampled peak detector and linear hold smooth gain}
\end{itemize}
Together with the following DDRC designed based on the results:
\begin{itemize}
\item{Upsampling}
\item{Windowed filtered max peak detector??}
\item{Branching gain smoothing??}
\end{itemize}

\end{document}