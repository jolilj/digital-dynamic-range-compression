\documentclass[../main2.tex]{subfiles}
%if compiling standalone, rootdir wil be previous folder,
%if compiling main document, rootdir will already be set by main file
\providecommand{\rootdir}{..}

\begin{document}

\section{Previous Work}\label{sec_theory}
In this section the theory behind feed-forward digital dynamic range compression is described. It is structured in relation to previous research where each subsection deals with the theory and method of corresponding work. However, prerequisites and definitions of necessary parameters are reviewed in the first paragraph. 

\subsection{Definitions} \label{theory_definitions}
A DDRC maps the dynamic range of a signal to a smaller, compressed, range. Let $x_n$ be the input signal, $y_n$ the output signal and $g_n$ the gain factor. In a compressor $0<g_n\leq 1$. The input signal, in general delayed, is multiplied by the gain factor, see Fig.~\ref{fig:block_gain},
%================================
\begin{figure}
\centerline{\subfile{\rootdir/figures/block_gain}}
\caption{Input signal multiplied by time-variant gain factor}
\label{fig:block_gain}
\end{figure}
%================================
\begin{align}
y_n = g_nx_{n-d_{la}}.
\label{eq:gainfactor}
\end{align}
%================================
The task is then to find the time-variant $g_n$ corresponding to the desired behaviour of the DDRC. In order to do so, the following parameters, corresponding to static and temporal characteristics, are introduced.
%================================
\begin{itemize}
\item{Static}
	\begin{itemize}
	\item \textbf{Threshold} $T$ - The defined limit above which compression is applied.
	\item \textbf{Ratio} $R$ - The input/output ratio above the threshold level.
	\item \textbf{Knee width}  $W$ - Controls the sharpness of the knee.
	\item \textbf{Make-up gain}  $M$ - The amount of gain applied in the final step to balance the perceived loudness of the output signal with the input signal.
\end{itemize}
\item{Temporal}
	\begin{itemize}
	\item \textbf{Attack time} $\tau_{a}$ - Determines how quickly the compression ratio is applied.
	\item \textbf{Release time} $\tau_{r}$ - Determines how quickly the compression is released as the input signal drops below the threshold level.
	\item \textbf{Look-ahead} $d_{la}$ - Difference in delay between direct signal path and side chain. 
	\item \textbf{Sample frequency} $f_{s}$ - Indirectly connected to the temporal characteristics through the time constants
	\end{itemize}
\end{itemize}
%================================
\begin{figure}
\centerline{\subfile{\rootdir/figures/typical_static_detailed}}
\caption{Typical static characteristics of a DDRC}
\label{fig:typical_static_detailed}
\end{figure}
%================================
\subsubsection{Static Characteristics}
The static characteristics are depicted in Fig.~\ref{fig:typical_static_detailed}. Do note that the static curve is defined in the log-domain, hence capital letters. For signal levels below $T$ the input is left unaffected while at levels above $T$ the output is compressed, resulting in the linear curve with slope $1/R$. 

Observing the discontinuity at $T$, a smooth transition into the compressed curve, with slope $1/R$, is motivated. This is achieved by replacing the sharp knee by a second degree polynomial\cite{frindle1996implementation}\cite{reiss2012tutorial} across the knee. The knee width $W$ is defined as the range in dB spanning either side of $T$ where such a polynomial is connected. The conditions to be met is for $Y(X)$ to be continuous and have continuous derivatives at the points $X=T-W/2$ and $X=T+W/2$. A knee width of $W=0$ corresponds to a sharp knee. 
\subsubsection{Gain Computer} \label{gain_computer}
The gain computer calculates logarithmic gain factor, $G_n(X)$, given the static characteristics and the signal. Derivation of $G_n$ is done by first examining the static curve of Fig.~\ref{fig:typical_static_detailed} and calculating the intermediate polynomial which yields the following piecewise continuous function
%================================
\begin{equation} \label{eq:gaincomp}
Y(X) = \begin{cases}
    X & \quad \text{if }X> T-\frac{1}{2}W \\[0.8em]
    \dfrac{X+(R^{-1}-1)(X-T+\frac{1}{2}W)^2}{2W}& \quad \text{if } |X-T| \leq \frac{1}{2}W\\[1.2em]
    T+ R^{-1}(X-T) & \quad \text{if } X > T + \frac{1}{2}W.
\end{cases}
\end{equation}
%================================
Now  by expressing equation \eqref{eq:gainfactor} in the logarithmic domain
%================================
\begin{align}
y_n &= x_ng_n   \\
\log|y_n| & = \log|x_n| + \log|g_n|   \\
Y &= X + G \label{eq:cv}
\end{align}
%================================
and by inserting equation (\ref{eq:cv}) into (\ref{eq:gaincomp})
%================================
\begin{equation}
X+G = \begin{cases}
    X & \quad \text{if }X > T-\frac{1}{2}W \\[0.8em]
    X + \dfrac{(1/R-1)(X-T+\frac{1}{2}W)^2}{2W}& \quad \text{if } |X-T| \leq \frac{1}{2}W\\[1.2em]
    T+ R^{-1}(X-T) & \quad \text{if } X > T + \frac{1}{2}W.
\end{cases}
\end{equation}
%================================
Solving for $G$ yields
\begin{equation} \label{eq:c}
G = \begin{cases}
    0 & \quad \text{if }X >T -\frac{1}{2}W \\[0.8em]
    \dfrac{(R^{-1}-1)(X-T+\frac{1}{2}W)^2}{2W}& \quad \text{if } |X-T| \leq \frac{1}{2}W\\[1.2em]
    \left(R^{-1}-1\right)\left(X-T\right) & \quad \text{if } X > T + \frac{1}{2}W.
 \end{cases}
\end{equation}
%================================
The typical gain factor curve is depicted in Fig.~\ref{fig:typical_gain_detailed}.
%================================
\begin{figure}
\centerline{\subfile{\rootdir/figures/typical_gain_detailed}}
\caption{Typical gain curve in log domain.}
\label{fig:typical_gain_detailed}
\end{figure}
%================================

The make-up gain, $M$, is defined as a constant amount of gain added to the signal. Since the compressor lowers the amplitude of the signal, the perceived loudness decreases. Make-up gain is used to balance the loudness of the output with the input.

With instant gain reduction the wave shape of the input signal is changed introducing higher harmonics, i.e. distortion, see Fig. \ref{fig:instant_comp}. To avoid this the compressor needs to be operating on the envelope of the signal which is achieved by introducing the temporal characteristics, see Fig. ~\ref{fig:typical_envelope_detailed}.
%================================
\begin{figure}[ht]
\centering
\subfile{\rootdir/figures/fig_instant_comp}
\caption{Instant compression changes the shape of the wave and thus introduces distortion.} 
\label{fig:instant_comp}
\end{figure}
%================================
\subsubsection{Temporal Characteristics}
The time constants, $\tau_{att}$ and $\tau_{rel}$, referred to as attack time and release time, can be seen as a measure of the reaction time of the compressor. There are various ways of defining them. In \cite{mcnally1984dynamic} the attack time is defined as the time it takes to achieve 63.2\% of the final change in gain in accordance with the conventional notion. The same applies to the release time such as the time it takes for the output to reach 63.2\% of the input value as it has decreased below the threshold level. The definitions can be understood by noting that with exponential characteristics
%================================
\begin{align}
1-e^{-t / \tau}\rvert_{t=\tau} = 1-e^{-1} \approx 63.2\% \label{eq:time_const}
\end{align}
%================================
If not specified otherwise this definition will be used throughout the thesis.

Another definition of the time constant, used in a more general DRC context ~\cite{mcnally1984dynamic}, is the time it takes to achieve 90\% of the final gain change. These time constants will be referred to as $\tau_{a90}$ and $\tau_{r90}$ for attack and release respectively, see Fig.~\ref{fig:time_constants}.
%================================
\begin{figure}
\centerline{\subfile{\rootdir/figures/typical_envelope_detailed}}
\caption{Typical dynamic characteristics of a DDRC. With a look-ahead delay sharp transients are effectively compressed.}
\label{fig:typical_envelope_detailed}
\end{figure}
%================================
%================================
\begin{figure}
\centerline{\subfile{\rootdir/figures/time_constants}}
\caption{Illustration of the defined time constants.}
\label{fig:time_constants}
\end{figure}
%================================
As can be seen in Fig.~\ref{fig:typical_envelope_detailed} the look-ahead, $d_{la}$, is defined as the applied delay to the input signal before the gain reduction, thus having the gain applied time-shifted. Sharp transients can thus be compressed effectively despite the smooth attack phase.
\end{document}