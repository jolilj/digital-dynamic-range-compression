\documentclass[../main2.tex]{subfiles}
%if compiling standalone, rootdir wil be previous folder,
%if compiling main document, rootdir will already be set by main file
\providecommand{\rootdir}{..}

\usepackage[autostyle]{csquotes}  

\begin{document}

\FloatBarrier
\section{Background} \label{background}


Over the years, a variety of DDRC designs have been proposed in academic literature. 

Many of the designs are based on the same basic feed-forward side-chain that dates back to \cite{mcnally1984dynamic}.

The possible variations on the theme is considerable.

In the absence of a broadly accepted definition of \emph{ideal} compression, "artefact-free", "transparent" or "low distortion" are common design goals.

The performance is compromised, more or less implicitly, by hardware limitations of the time.

Vague design goals make evaluation and comparisons between the resulting designs difficult.

Few studies have been conducted that qualitatively evaluates and compares different designs. 

\todo{Something about research in hearing aids, and the development of FES}

In~\cite{reiss2012tutorial}, some common designs are analysed and formally compared using effective compression ratio (ECR), total harmonic distortion (THD), and fidelity of envelope shape (FES) as metrics. 

The list of designs that is analysed by Reiss is far from complete.




\textbf{Background}
\begin{itemize}
\item no definition of good
\item many variations but on the same basic principle
\item many proposed designs, but few comparisons
\item When comparison is made, it's made with FES but on the wrong envelope and with a limited parameter set.
\end{itemize}
\textbf{This paper}
\begin{itemize}
\item Continue the work of Reiss and investigate some more designs
\item Continue the work of Reiss and compare with modified FES as metric
\item Take inspiration from Bitzer and try to optimise parameters to get as fair results as possible for different designs.
\end{itemize}
\textbf{Limitations}
\begin{itemize}
\item mono, single band compression
\item feed forward topology
\end{itemize}

\end{document}