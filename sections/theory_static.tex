\documentclass[../main2.tex]{subfiles}
%if compiling standalone, rootdir wil be previous folder,
%if compiling main document, rootdir will already be set by main file
\providecommand{\rootdir}{..}

\begin{document}

\FloatBarrier
\subsection{Static Characteristics}
The gain factor $g_n$ is calculated based on the level of the signal. In order to derive the $g_n$, the desired input/output relation of the signal level is defined. This is referred to as the static curve of the DDRC.
\subsubsection{Static Curve}
The static curve is depicted in Fig.~\ref{fig:typical_static_detailed}. For signal levels below $T$ the input is left unaffected while at levels above $T$ the output is compressed, resulting in the linear curve with slope $1/R$.  

%================================
\begin{figure}
\centerline{\subfile{\rootdir/figures/typical_static_detailed}}
\caption{Typical static characteristics of a DDRC}
\label{fig:typical_static_detailed}
\end{figure}
%================================

Observing the discontinuity at $T$, a smooth transition into the compressed curve, with slope $1/R$, is motivated. This is achieved by replacing the sharp knee by a second degree polynomial\cite{frindle1996implementation}\cite{reiss2012tutorial} across the knee. The knee width $W$ is defined as the range in dB spanning either side of $T$ where such a polynomial is connected. The conditions to be met is for $Y(X)$ to be continuous and have continuous derivatives at the points $X=T-W/2$ and $X=T+W/2$. A knee width of $W=0$ corresponds to a sharp knee. 
%================================
\subsubsection{Gain Computer} \label{gain_computer}
The gain computer, Fig. ~\ref{fig:block_genericDDRC}, calculates logarithmic gain factor, $G_n(X)$, given the static characteristics and the signal. Derivation of $G_n$ is done by first examining the static curve of Fig.~\ref{fig:typical_static_detailed} and calculating the intermediate polynomial which yields the following piecewise continuous function
%================================
\begin{equation} \label{eq:gaincomp}
Y(X) = \begin{cases}
    X & \quad \text{if }X> T-\frac{1}{2}W \\[0.8em]
    \dfrac{X+(R^{-1}-1)(X-T+\frac{1}{2}W)^2}{2W}& \quad \text{if } |X-T| \leq \frac{1}{2}W\\[1.2em]
    T+ R^{-1}(X-T) & \quad \text{if } X > T + \frac{1}{2}W.
\end{cases}
\end{equation}
%================================
Now  by expressing equation \eqref{eq:gainfactor} in the logarithmic domain
%================================
\begin{align}
y_n &= x_ng_n   \\
\log|y_n| & = \log|x_n| + \log|g_n|   \\
Y &= X + G \label{eq:cv}
\end{align}
%================================
and by inserting equation (\ref{eq:cv}) into (\ref{eq:gaincomp})
%================================
\begin{equation}
X+G = \begin{cases}
    X & \quad \text{if }X > T-\frac{1}{2}W \\[0.8em]
    X + \dfrac{(1/R-1)(X-T+\frac{1}{2}W)^2}{2W}& \quad \text{if } |X-T| \leq \frac{1}{2}W\\[1.2em]
    T+ R^{-1}(X-T) & \quad \text{if } X > T + \frac{1}{2}W.
\end{cases}
\end{equation}
%================================
Solving for $G$ yields
\begin{equation} \label{eq:c}
G = \begin{cases}
    0 & \quad \text{if }X >T -\frac{1}{2}W \\[0.8em]
    \dfrac{(R^{-1}-1)(X-T+\frac{1}{2}W)^2}{2W}& \quad \text{if } |X-T| \leq \frac{1}{2}W\\[1.2em]
    \left(R^{-1}-1\right)\left(X-T\right) & \quad \text{if } X > T + \frac{1}{2}W.
 \end{cases}
\end{equation}
%================================
The typical gain factor curve is depicted in Fig.~\ref{fig:typical_gain_detailed}.
%================================
\begin{figure}[h]
\centerline{\subfile{\rootdir/figures/typical_gain_detailed}}
\caption{Typical gain curve in log domain.}
\label{fig:typical_gain_detailed}
\end{figure}
%================================

The make-up gain, $M$, is defined as a constant amount of gain added to the signal. Since the compressor lowers the amplitude of the signal, the perceived loudness decreases. Make-up gain is used to balance the loudness of the output with the input.

With instant gain reduction the wave shape of the input signal is changed introducing higher harmonics, i.e. distortion, see Fig. \ref{fig:instant_comp}. To avoid this the compressor needs to be operating on the envelope of the signal which is achieved by introducing the level detector.
%================================
\begin{figure}[ht]
\centering
\subfile{\rootdir/figures/fig_instant_comp}
\caption{Instant compression changes the shape of the wave and thus introduces distortion.} 
\label{fig:instant_comp}
\end{figure}
%================================

\end{document}