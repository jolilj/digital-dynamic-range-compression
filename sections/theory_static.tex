\documentclass[../main2.tex]{subfiles}
%if compiling standalone, rootdir wil be previous folder,
%if compiling main document, rootdir will already be set by main file
\providecommand{\rootdir}{..}

\begin{document}

\FloatBarrier
\subsection{Static Characteristics}
The gain factor $g_n$ is calculated based on the level of the signal. In order to derive the $g_n$, the desired input/output relation of the signal level is defined. This is referred to as the static curve of the DDRC.

\subsubsection{Static Curve}
The static curve is depicted in Fig.~\ref{fig:typical_static_detailed} where $Y(L)$ is the desired output signal level and $L$ is the input signal level calculated in the level detector. For signal levels below $T$ the input is left unaffected while at levels above $T$ the output is compressed, resulting in the linear curve with slope $1/R$.  

%================================
\begin{figure}
\centerline{\subfile{\rootdir/figures/typical_static_detailed}}
\caption{Typical static characteristics of a DDRC showing output signal level $Y$ against input signal level $L$.}
\label{fig:typical_static_detailed}
\end{figure}
%================================

Observing the discontinuity at $T$, a smooth transition into the compressed curve, with slope $1/R$, is motivated. This can be achieved by replacing the sharp knee by a second degree polynomial across the knee\cite{frindle1996implementation}\cite{reiss2012tutorial}. The knee width $W$ is defined as the range in dB spanning either side of $T$ where such a polynomial is connected. The conditions to be met is for $Y(L)$ to be continuous and have continuous derivatives at the points $L=T-W/2$ and $L=T+W/2$. A knee width of $W=0$ corresponds to a sharp knee. 
%================================
\subsubsection{Gain Computer} \label{gain_computer}
The gain computer in Fig. ~\ref{fig:block_genericDDRC}, calculates the \emph{non-smoothed} logarithmic gain factor, $G'_n(L)$, given the static characteristics and the signal. Derivation of $G'_n$ is done by first examining the static curve of Fig.~\ref{fig:typical_static_detailed} and calculating the intermediate polynomial which yields the following piecewise continuous function
%================================
\begin{equation} \label{eq:gaincomp}
Y(L) = \begin{cases}
    L & \quad \text{if }L<T-\frac{1}{2}W \\[0.8em]
    \dfrac{L+(R^{-1}-1)(L-T+\frac{1}{2}W)^2}{2W}& \quad \text{if } |L-T| \leq \frac{1}{2}W\\[1.2em]
    T+ R^{-1}(L-T) & \quad \text{if } L > T + \frac{1}{2}W.
\end{cases}
\end{equation}
%================================
Now  by expressing equation \eqref{eq:gainfactor} in the logarithmic domain
%================================
\begin{align}
y_n &= x_ng_n   \\
\log|y_n| & = \log|x_n| + \log|g_n|   \\
Y &= L + G' \label{eq:cv}
\end{align}
%================================
and by inserting equation (\ref{eq:cv}) into (\ref{eq:gaincomp})
%================================
\begin{equation}
L+G' = \begin{cases}
    L & \quad \text{if }L < T-\frac{1}{2}W \\[0.8em]
    L + \dfrac{(1/R-1)(L-T+\frac{1}{2}W)^2}{2W}& \quad \text{if } |L-T| \leq \frac{1}{2}W\\[1.2em]
    T+ R^{-1}(L-T) & \quad \text{if } L > T + \frac{1}{2}W.
\end{cases}
\end{equation}
%================================
Solving for $G'$ yields
\begin{equation} \label{eq:c}
G' = \begin{cases}
    0 & \quad \text{if }L <T -\frac{1}{2}W \\[0.8em]
    \dfrac{(R^{-1}-1)(L-T+\frac{1}{2}W)^2}{2W}& \quad \text{if } |L-T| \leq \frac{1}{2}W\\[1.2em]
    \left(R^{-1}-1\right)\left(L-T\right) & \quad \text{if } L > T + \frac{1}{2}W.
 \end{cases}
\end{equation}
%================================
The typical gain factor curve is depicted in Fig.~\ref{fig:typical_gain_detailed}.
%================================
\begin{figure}[h]
\centerline{\subfile{\rootdir/figures/typical_gain_detailed}}
\caption{Typical gain curve in log domain.}
\label{fig:typical_gain_detailed}
\end{figure}
%================================

 A common parameter introduced in the end of the side-chain it the \emph{make-up gain} defined as a constant amount of gain added to the signal level. Since the compressor lowers the level of the signal, the perceived loudness decreases. Make-up gain is used to balance the loudness of the output with the input. It is omitted in the derivations for convenience with no loss of generality.

\end{document}