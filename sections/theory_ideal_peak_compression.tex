\documentclass[../main2.tex]{subfiles}
%if compiling standalone, rootdir wil be previous folder,
%if compiling main document, rootdir will already be set by main file
\providecommand{\rootdir}{..}

\begin{document}

\subsubsection{Defintion of Ideal Compression}
%================================
\FloatBarrier
Given an input signal $x_n$, as before consisting of an envelope $e_n$ and a fine structure $f_n$ so that
%================================
\begin{equation}\label{eq:input_signal_env_finestruct}
x_n = e_n f_n,
\end{equation}
%================================
the dynamic range, Eq. \eqref{eq:s_peak}, can be expressed as
%================================
\begin{equation}
S_\text{pk} =A_\text{max} - A_\text{min} = 20 \logten \left(\frac{\max(e_n)}{\min(e_n)}\right) \dBFS
\end{equation}
%================================
since $A_\text{max} = \max(A_{\text{pk},n}) = \max(e_n)$ and $A_\text{min} =  \min(A_{\text{pk},n}=\min(e_n))$.
 
In this thesis \emph{ideal compression} is defined as the processes of mapping all or a part of this dynamic range to a smaller range, without changing the fine structure. This mapping is further defined to be linear in the logarithmic domain and with a threshold below which the signal level is unchanged. For the input signal in Eq.\eqref{eq:input_signal_env_finestruct}, the envelope of the ideal output signal can thus be described by the following equation
\begin{equation}\label{eq:dynamic_range_mapping}
O_n =
\begin{cases}
	K E_n + M 					& E_n \geq T'  \\
	E_n							& \text{otherwise},
\end{cases} 
\end{equation}\label{eq:ideal_output_envelope}
where $K$, $M$ and $T'$ are constants defining the wanted mapping. This clearly corresponds to the static curve in Fig.\ref{fig:typical_static_detailed} described by Eq.\eqref{eq:gaincomp} with $W=0$ and the following change of variables:
\begin{equation}
\begin{split}
K &= R^{-1}, \\
M &= (R^{-1}-1)T,\\
T' &= T
\end{split}
\end{equation}
where $T$ is the threshold and $R$ the ratio.

The ideal output signal $y_{I,n}$ can then be written as
\begin{equation}\label{eq:ideal_output}
y_{I,n} = o_n f_n
\end{equation}
where the fine-structure $f_n$ is the same as in Eq.\eqref{eq:input_signal_env_finestruct} and $o_n$ is the mapped output envelope in linear domain. It is shown below that this definition of ideal compression yields optimal performance in terms of the proposed metrics in previous work, with $\FES = 1$, $\THDF=0$\%, and $\ECR$ equal to the specified ratio $R$, given certain limitations on the input signal.
% ========================================

\end{document}
