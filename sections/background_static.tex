\documentclass[../main2.tex]{subfiles}
%if compiling standalone, rootdir wil be previous folder,
%if compiling main document, rootdir will already be set by main file
\providecommand{\rootdir}{..}

\begin{document}

\subsection{Static Characteristics}
The static characteristics are depicted in Fig.~\ref{fig:typical_static_detailed}. Do note that the static curve is defined in the log-domain, hence capital letters. For signal levels below $T$ the input is left unaffected while at levels above $T$ the output is compressed, resulting in the linear curve with slope $1/R$. 

%================================
\begin{figure}
\centerline{\subfile{\rootdir/figures/typical_static_detailed}}
\caption{Typical static characteristics of a DDRC}
\label{fig:typical_static_detailed}
\end{figure}
%================================

Observing the discontinuity at $T$, a smooth transition into the compressed curve, with slope $1/R$, is motivated. This is achieved by replacing the sharp knee by a second degree polynomial\cite{frindle1996implementation}\cite{reiss2012tutorial} across the knee. The knee width $W$ is defined as the range in dB spanning either side of $T$ where such a polynomial is connected. The conditions to be met is for $Y(X)$ to be continuous and have continuous derivatives at the points $X=T-W/2$ and $X=T+W/2$. A knee width of $W=0$ corresponds to a sharp knee. 
%================================
\end{document}