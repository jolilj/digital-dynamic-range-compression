\documentclass[../main2.tex]{subfiles}
%if compiling standalone, rootdir wil be previous folder,
%if compiling main document, rootdir will already be set by main file
\providecommand{\rootdir}{..}

\begin{document}
\subsection{Envelope and Fine Structure}
The envelope of an analytic signal $x(t) = a(t)e^{i(\omega t)}$ is defined as \cite{bedrosian1962analytic}
\begin{align}
e(t) \equiv |x(t)| = a(t)
\end{align}
where $e^{i(\omega t)}$ is referred to as the \emph{fine structure} of the signal.

With more complex signals it is however difficult to clearly define an envelope. As expressed in \cite{bedrosian1962analytic}: \emph{"Theoretically, the signal must be analytic to permit envelope detection"}. In such cases it can be seen as the line outlining the extremes of the signal and is acquired by various envelope detection methods, as discussed in \ref{level_detection}.

\end{document}