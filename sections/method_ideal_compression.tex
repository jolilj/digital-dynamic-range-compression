\documentclass[../main2.tex]{subfiles}
%if compiling standalone, rootdir wil be previous folder,
%if compiling main document, rootdir will already be set by main file
\providecommand{\rootdir}{..}

\begin{document}

%================================
\FloatBarrier
\subsection{Definition of Ideal Compression} \label{method_ideal_compression}
Given the metrics in Section \ref{XXXXX} ideal compression can be defined on certain types of signals where the envelope is known beforehand. There is however fundamental differences between peak based and RMS based compression, so the two cases are treated separately.

\subsubsection{Peak detecting compressors}
Let $x$ be the input and $y$ be the compressed output. Further assume that $x$ can be expressed as a product of an \emph{envelope} $e$ and a real-valued \emph{fine structure} $f$ 
%================================
\begin{align}
x_n = e_n\cdot f_n.
\end{align}
%================================
Let $G(x)$ be the gain computer, see section \ref{gain_computer}, and $g(x) = 10^{G/20}$. The ideal compression is then given by
\begin{align}
y_{n,I} \equiv g(e_n) f_n. 
\end{align}
%================================

\todo{Show that if Amin is above T, FES=1, THD=0 \% and ECR=R. If Amin is below T, we can redefine FES  to Fidelity of Ideal Envelope Shape, see metrics}

\begin{figure}
\captionsetup{justification=centering}
\begin{subfigure}{\linewidth}
\centering
\centerline{\subfile{\rootdir/figures/signal_env_fine_struct}}
\caption{Complete signal}
\label{fig:signal_env_fine_struct}
\end{subfigure}
\par\bigskip
\begin{subfigure}{.5\linewidth}
\centering
\subfile{\rootdir/figures/signal_env}
\caption{Envelope}
\label{fig:signal_env}
\end{subfigure}
\begin{subfigure}{.5\linewidth}
\centering
\subfile{\rootdir/figures/signal_fine_struct}
\caption{Fine structure}
\label{fig:signal_fine_struct}
\end{subfigure}%
\caption{Signal $x_n = e_n\cdot f_n$ split into it's envelope $e_n$ and fine structure $f_n$.}
\label{fig:analytic_signal}
\end{figure}
%================================

\subsubsection{RMS detecting compressors}
Divide by crest factor of carrier wave, to get the \emph{instantaneous RMS amplitude}.

\end{document}
