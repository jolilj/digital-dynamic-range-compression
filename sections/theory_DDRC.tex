\documentclass[../main2.tex]{subfiles}
%if compiling standalone, rootdir wil be previous folder,
%if compiling main document, rootdir will already be set by main file
\providecommand{\rootdir}{..}

\begin{document}

\FloatBarrier
\section{Theory - Investigated DDRC Designs} \label{theory_DDRC}
A DDRC maps the dynamic range of a signal to a smaller, compressed, range. There are two possible topologies for implementing a DDRC; a feed-forward or a feedback system. In \cite{reiss2012tutorial} it is shown that the feedback topology cannot implement a look-ahead delay and cannot work as a perfect limiter. Therefore, only variations of the feed-forward topology are treated in this thesis. 

\subsection{Principle of a Feed-Forward DDRC}
A Block diagram of a generic feed-forward DDRC can be seen in Fig. ~\ref{fig:block_genericDDRC}.
%================================
\begin{figure}[h]
\centerline{\subfile{\rootdir/figures/block_genericDDRC}}
\caption{Block diagram of a generic feed-forward DDRC}
\label{fig:block_genericDDRC}
\end{figure}
%================================

Let $x_n$ be the sampled input signal, $y_n$ the output signal and $g_n$ the \emph{gain factor}. The input signal, in general delayed by $d$ samples, is multiplied by the gain factor,
%================================
\begin{align}
y_n = g_nx_{n-d}.
\label{eq:gainfactor}
\end{align}
%================================
The task is then to determine the gain factor $g_n$ corresponding to the desired behaviour of the DDRC. This is the purpose of the \emph{side-chain}, which can be divided into the three steps \emph{level detection}, \emph{gain computation} and \emph{gain smoothing}.

\subfile{\rootdir/sections/theory_level_detection}

\subfile{\rootdir/sections/theory_static}

\subfile{\rootdir/sections/theory_smoothing}

\end{document}
