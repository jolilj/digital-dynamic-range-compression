\documentclass[../main2.tex]{subfiles}
%if compiling standalone, rootdir wil be previous folder,
%if compiling main document, rootdir will already be set by main file
\providecommand{\rootdir}{..}

\begin{document}
\subsection{Result}\label{discussion_results}
It is evident that the two compressors in Fig.~\ref{fig:param_opt} in fact show similar behaviour and thus overcoming the differences observed in Fig.~\ref{fig:param_opt_left}, motivating the use of parameter optimisation when comparing compressor design, as described in~\ref{method_param_opt}. Conversely this is seen when switching input from the test signal defined by Eq.~\ref{eq:test_signal} to a step and down-step as seen in Fig.~\ref{fig:complete_DDRC_opt_step_response}. The step and down-step responses show \emph{completely} different characteristics, thus illustrating the non-linearity of the system. The observed differences in output between the DDRCs are highly dependent on the input.

It is important to note that with the numerical method used, see section~\ref{method_dist_ideal}, only an upper bound could be given of the distance to ideal. There is no guarantee that the minima found is global. In fact, when randomising the start guesses, different minima is found, although they are in the neighbourhood of each other.

In the case of (1) McN84, the extremely good result for the optimised carrier frequency, Fig.~\ref{fig:peak_det_opt_env_zoom} in combination with the poor performance when varying the carrier frequency as described in section~\ref{method_carrier_freq_dep}, especially at lower frequencies, see Fig.~\ref{fig:complete_DDRC_opt_fc_dep}, may be explained by the number of parameters. While making it possible to find a very good fit to the ideal curve with extremely short release times ($\approx 6$ ms for both peak detector and gain smoothing), these settings prove to be less optimal for other carrier frequencies.

The GMR designs did show a larger error in the optimisation, Fig.~\ref{fig:peak_det_opt_env_zoom}, but proved to be more stable when varying the carrier frequency, Fig.~\ref{fig:complete_DDRC_opt_fc_dep}. This may be explained by the longer release times making the design more robust to the slower oscillations of the carrier frequency. 

The FE96 compressor shows larger fluctuations and error at carrier frequencies where the other are stable, see Fig.~\ref{fig:complete_DDRC_opt_fc_dep} where $10^3<f_c<10^4$, apart from the characteristics peaks due to the introduced harmonics of the peak detector. Comparing the design with the GMR compressors, the main difference is the gain smoothing component. As the peak detection in essence is the full wave rectification in both cases, but up-sampled in FE96, the major difference between the two detectors is the reduction of spikes at certain frequencies. The difference in performance is thus believed to be caused by the linear gain smoothing filter. Note however, that since no guarantee can be made regarding global minima of parameter optimisation, there may in theory exist a parameters more optimal than the one found. Furthermore, the effect of the up-sampling detector can be clearly seen in the reduction of spikes at higher frequencies.

Finally, the optimised delay vary between the DDRCs, from 5.6 ms to 0.3 ms see Table~\ref{tab:complete_DDRC_opt_params}, but there is little indication for how the delay actually affects the performance in this setup. The effect of the high delay times for McN84 and GMR12 decoupled can be seen in Fig.~\ref{fig:complete_DDRC_opt_step_response}. The onset is compressed immediately, despite the attack time, and the compression release is applied before the down-step resulting in the peak at the end of the step.

%A very surprising result is the step response in Fig.~\ref{fig:complete_DDRC_opt_step_response}. While all the compressors yield essentially the exact same output when fed with the sinusoidal test signal, the step response shows significant differences. This clearly demonstrates the difficulties of nonlinear systems analysis, and the limits of step responses for describing the behaviour of such systems.

\end{document}
