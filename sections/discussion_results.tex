\documentclass[../main2.tex]{subfiles}
%if compiling standalone, rootdir wil be previous folder,
%if compiling main document, rootdir will already be set by main file
\providecommand{\rootdir}{..}

\begin{document}
\subsection{Result}\label{discussion_results}
\subsubsection{Peak Detection}
All tested peak detectors track the envelope well when optimised. The smooth decoupled detector shows smooth behaviour but do underestimate the envelope by a small amount. The windowed max detector has little variation, but is rather noisy. The analog detector shows interestingly very similar behaviour to the smooth branching detector. This is due to the slow varying envelope. With faster variation, the release trajectory will distinguish these from each other.

As the carrier frequency is varied, Fig.~\ref{fig:peak_det_opt_env_fc10000_dep_error} and Fig.~\ref{fig:peak_det_opt_env_fc10000_dep_fes}, the windowed max filter is the first to experience oscillations in the output resulting in an increased $\EN$. For the detection to be stable, a window of at least $f_c/2$ is required (assuming the phase $\phi=0$). The optimised window parameter results in the limit $f_c =  f_s/2w = 44100/2\cdot 53 \approx 416$ Hz which corresponds to the point where the $\EN$ begins to diverge. The three pole filter is the next filter experiencing increasing $\EN$. The smooth decoupled, smooth branching and analogue detectors show similar behaviour, but inspecting the zoomed version the three are separated with the smooth decoupled showing slightly better performance.

An interesting phenomena seen in the plots are the various peaks in error at the high frequencies. This is the expected behaviour due to the non-linearities of the filters. 

\subsubsection{RMS Detection}
The optimised RMS outputs are very close to the specified ideal RMS. This is due to the fact that $f_m<<f_c$ making the envelope fairly constant during one period of the carrier wave letting the detector extract the crest factor with great accuracy. As the carrier frequency reduces, the detector is not slow enough to pick the entire period of the carrier resulting in an increase in $\EN$ and a decrease in $\FES$ as expected, Fig.~\ref{fig:rms_det_opt_env_fc10000_dep_error-fes}. By optimising the parameters for a carrier wave of much lower frequency, this can be prevented as can be seen in the results where $f_c=20$ Hz, Fig.~\ref{fig:rms_det_opt_env_fc20_dep_error-fes}. This does however increase the amount of look-ahead needed, see Tab.~\ref{tab:rms_det_opt_params}. It is worth to note that the optimised window length is exactly half a period of the carrier wave as expected.

Furthermore, the linearity of the IIR-filter and the single introduced harmonic due to the squaring operation makes the plots smooth for high frequencies as opposed to the peak detectors. It was however surprising that the frequencies above $f_s/2=11,025$ Hz didn't contribute to a larger error as indicated by the introduced double harmonic which would be aliased into the low frequency band, and hence not filtered. To explain this, further analysis needs to be carried out.

\subsubsection{Peak Detection with Gain Computer}
TODO

\subsubsection{Compressor Design}
The first thing to notice is that the outputs are extremely close to ideal for all tested compressors with $\EN < 10^{-3}$ per sample. In Fig.~\ref{fig:complete_compressor_opt_out}, it's impossible to distinguish compressor output from ideal.

The gain factors reveal some more information and qualitative differences, but only when examined on very small scales. The McN84 design are closest to ideal with small ripples in gain factor. The GMR12 decoupled design is smoother but with a small static error, while GMR12 branching and FE96 oscillate the most around the ideal curve.

Looking at the carrier frequency dependency in Fig.~\ref{fig:complete_DDRC_opt_fc_dep}, McN84 is the one suffering the most for low frequencies, followed by FE96. Both GMR designs are clearly the most stable ones. The extremely good result for the optimised carrier frequency in combination with the poor performance for lower frequencies in the case of the McN84 design may be explained by the number of parameters. While making it possible to find a very good fit to the ideal curve with extremely short release times ($\approx 6$ ms for both peak detector and gain smoothing), these settings prove to be less optimal for other signals. The stable results for the GMR designs may be explained by their longer release times. Worth noting is that the branching design manages to maintain a small error for low frequencies with only 1 ms of look-ahead. 

The peaks in error at certain frequencies clearly remains from the level detection step. FE96 with its up-sampling peak detector on the other hand has a more stable curve, as expected.

A very surprising result is the step response in Fig.~\ref{fig:complete_DDRC_opt_step_response}. While all the compressors yield essentially the exact same output when fed with the sinusoidal test signal, the step response shows significant differences. This clearly demonstrates the difficulties of nonlinear systems analysis, and the limits of step responses for describing the behaviour of such systems.

\end{document}
