\documentclass[../main2.tex]{subfiles}
%if compiling standalone, rootdir wil be previous folder,
%if compiling main document, rootdir will already be set by main file
\providecommand{\rootdir}{..}

\begin{document}
\subsection{Giannoulis, Massberg and Reiss 2012}
In \cite{reiss2012tutorial} the design seen in Fig ~\ref{fig:block_reiss_theory} is proposed.
\subsubsection{Level Detection and Gain Computer}
For level detection the absolute value of the signal is taken. No filtering is applied here, with the non-linearity of the full rectification leading to measurement error in the gain computer, see section ~\ref{}. The gain computer works in the logarithmic domain and is constructed as described in section \ref{theory_definitions}. In the evaluation carried out a sharp knee is used. 
\subsubsection{Gain Smoothing}
In \cite{reiss2012tutorial} the gain smoothing is referred to as \emph{level detection}, although it is a \emph{smoothing filter} applied to the calculated gain factor. Two different implementations are proposed, here named in accordance with \cite{reiss2012tutorial}.
\begin{itemize}
\item{Decoupled filter}
\item{Branching filter}
\end{itemize}
\todo[inline, caption={Decoupled}]{Beskriv filtret och bifoga step och downstep response}
%================================
\begin{equation}
\begin{split}
z_k &= \begin{cases}
    x_k 								& x_k > z_{k-1} \\
    \alpha_{r} x_k + (1-\alpha_{r}) z_{k-1} 	& x_k \leq z_{k-1}
\end{cases} \\
y_k &= \alpha_{a} z_k + (1-\alpha_{a}) y_{k-1}
\end{split}
\end{equation}
%================================
\todo[inline, caption={Branching}]{Beskriv filtret och bifoga step och downstep response}
%================================
\begin{equation}
y_k = \begin{cases}
    \alpha_{a} x_k + (1-\alpha_{a}) y_{k-1} 	& x_k > y_{k-1} \\
    \alpha_{r} x_k + (1-\alpha_{r}) y_{k-1} 	& x_k \leq y_{k-1}
\end{cases}
\end{equation}
%================================
\begin{figure}
\centerline{\subfile{\rootdir/figures/block_giannoulis_theory}}
\caption{Block diagram of the proposed compressor design in \cite{reiss2012tutorial}}
\label{fig:block_reiss_theory}
\end{figure}
%================================
\end{document}}