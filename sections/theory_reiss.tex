\documentclass[../main2.tex]{subfiles}
%if compiling standalone, rootdir wil be previous folder,
%if compiling main document, rootdir will already be set by main file
\providecommand{\rootdir}{..}

\begin{document}
\subsection{Giannoulis, Massberg and Reiss 2012}
In \cite{reiss2012tutorial} the design seen in Fig ~\ref{fig:block_reiss_theory} is proposed.
%================================
\begin{figure}
\centerline{\subfile{\rootdir/figures/block_giannoulis_theory}}
\caption{Block diagram of the proposed compressor design in \cite{reiss2012tutorial}}
\label{fig:block_reiss_theory}
\end{figure}
%================================
\subsubsection{Level Detection and Compression Computer}
For level detection the absolute value of the signal is taken. No filtering is applied here, with the non-linearity of the full rectification leading to measurement error in the gain computer, see section ~\ref{}. In order to smooth the gain in the logarithmic domain the output of the gain computer in \cite{reiss2012tutorial} is $\geq 0$ or equally the amount of compression applied. It is therefore referenced as the compression computer and is, with the exception of the output being $C'_n = |G'_n|$, constructed as described in section \ref{theory_definitions}. In the evaluation carried out a sharp knee is used. 

 \subsubsection{Compression Smoothing}
In \cite{reiss2012tutorial} the compression smoothing is referred to as \emph{level detection}, although it is a \emph{smoothing filter} applied to the calculated compression factor, $C'_n$. Two different implementations are proposed, here named in accordance with \cite{reiss2012tutorial}.
\begin{itemize}
\item{Smooth decoupled filter}
\item{Smooth branching filter}
\end{itemize}
The difference equation of the smooth decoupled filter is
%================================
\begin{equation}
\begin{split}
Z_n &= \begin{cases}
   C'_n								& C'_n > Z_{n-1} \\
    \alpha_{r} C'_n + (1-\alpha_{r}) Z_{n-1} 	& C'_n \leq Z_{n-1}
\end{cases} \\
C_n &= \alpha_{a} Z_n + (1-\alpha_{a}) C_{n-1}
\end{split}
\end{equation}
%================================
which result in a smooth transition between the attack and release phase without discontinuities. The name is poorly chosen since it in fact couples the attack and release parameters in the release phase. In \cite{reiss2012tutorial} it is shown that the actual release time is $\approx \tau_a + \tau_r$. The name is a remnant from it's analogue counterpart. To obtain a filter with attack and release parameters independent of each other the branching filter is suggested where 
%================================
\begin{equation}
C_n = \begin{cases}
    \alpha_{a} C'_n + (1-\alpha_{a}) C_{n-1} 	& C'_n > C_{n-1} \\
    \alpha_{r} C'_n + (1-\alpha_{r}) C_{n-1} 	& C'_n \leq C_{n-1}.
\end{cases}
\end{equation}
%================================
The branching conditions does however introduce a discontinuity where the attack phase switches to the release phase. The step and downstep response in Fig ~\ref{fig:step_reiss_filter} shows the different actual release trajectories and the discontinuity of the branching filter at the downstep.
%================================
\begin{figure}
\centerline{\subfile{\rootdir/figures/step_reiss_filter}}
\caption{Step and downstep response of the decoupled and branching filter, with $\tau_a = 50$ ms, $\tau_r = 150$ ms}
\label{fig:step_reiss_filter}
\end{figure}
%================================
\end{document}}