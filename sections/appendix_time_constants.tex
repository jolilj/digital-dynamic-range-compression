\documentclass[../main2.tex]{subfiles}
%if compiling standalone, rootdir wil be previous folder,
%if compiling main document, rootdir will already be set by main file
\providecommand{\rootdir}{..}

\begin{document}

\section{Time Constants}\label{appendix_time_constants}
The time constants, denoted by $\tau$ with various subscripts depending on purpose, can be seen as a measure of the reaction time of the compressor. There are various ways of defining them. In \cite{mcnally1984dynamic} the attack time is defined as the time it takes to achieve 63.2\% of the final change in gain in accordance with the conventional notion. The same applies to the release time such as the time it takes for the output to reach 63.2\% of the input value as it has decreased below the threshold level. The definitions can be understood by noting that with exponential characteristics
%================================
\begin{align}
1-e^{-t / \tau}\rvert_{t=\tau} = 1-e^{-1} \approx 63.2\% \label{eq:time_const}
\end{align}
%================================
%================================
\begin{figure}[h]
\centerline{\subfile{\rootdir/figures/time_constants}}
\caption{Illustration of the defined time constants.}
\label{fig:time_constants}
\end{figure}
%================================
\end{document}
