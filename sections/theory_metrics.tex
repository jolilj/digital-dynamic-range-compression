\documentclass[../main2.tex]{subfiles}
%if compiling standalone, rootdir wil be previous folder,
%if compiling main document, rootdir will already be set by main file
\providecommand{\rootdir}{..}

\begin{document}

\subsection{Metrics}

% ============================
\subsubsection{Fidelity of Envelope Shape} \label{fes}
In~\cite{stone2007quantifying} Fidelity of Envelope Shape (FES) is proposed as a metric of the correlation between the envelope shape before and after compression. It is further used in the design comparison conducted in ~\cite{reiss2012tutorial}.

FES is defined as the correlation between the input signal envelope $E_n$ and the output signal envelope $O_n$ using the Pearson correlation coefficient: \begin{equation}
\FES = \dfrac{\sum(E_n - \langle E_n \rangle)(O_n - \langle O_n \rangle)}{\sqrt{\sum(E_n -\langle E_n \rangle )^2}\sqrt{\sum(O_n - \langle O_n \rangle )^2}},
\end{equation}
where $\langle E_n \rangle$ and $\langle O_n \rangle$ denote the mean value of $E_n$ and $O_n$ respectively.

% ============================
\subsubsection{Total Harmonic Distortion}
A common measure of nonlinearity of a system is the \emph{total harmonic distortion} defined as \cite{dafx02}
\begin{align}
\THDF = \dfrac{\sqrt{a_2^2 + a_3^2 + ... + a_N^2}}{a_1}
\end{align}
which is the square root the ratio of the sum of powers of all harmonic frequencies above the fundamental frequency to the fundamental frequency.

% ============================
\subsubsection{Effective Compression Ratio}
A metric used in \cite{stone1992syllabic} is the \emph{effective compression ratio} defined as
\begin{align}
\ECR = \dfrac{S_\text{I}}{S_\text{O}}
\end{align}
where $S_\text{I}$ and $S_\text{O}$ is the input  and output dynamic range, respectively.

% ============================
\subsubsection{Euclidean Norm}
The euclidean norm, or normalised euclidean distance, between two signals $x_n$, $y_n$ of equal length is defined as
\begin{align}
\EN = ||x-y|| = \sqrt{\frac{1}{N}\sum_{n=0}^{N-1}(x_n-y_n)^2}
\end{align}
where $N$ is the length of the signals in samples.
\end{document}