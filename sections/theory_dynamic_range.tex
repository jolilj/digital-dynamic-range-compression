\documentclass[../main2.tex]{subfiles}
%if compiling standalone, rootdir wil be previous folder,
%if compiling main document, rootdir will already be set by main file
\providecommand{\rootdir}{..}

\begin{document}

\FloatBarrier
\section{Theory - Definitions and Dynamic Range} \label{theory_dynamic_range}
The term \emph{dynamic range} has two different meanings in the context of digital audio. For a DSP \emph{system}, the dynamic range is the ratio between the largest representable signal to the quantisation error and is closely related to \emph{signal to noise ratio} (SNR)  \cite{wilson1993filter}. For an audio \emph{signal} on the other hand, the dynamic range is the ratio between the \emph{level} of the loudest portion and the quietest \cite{davis1989sound}. The goal of a dynamic range controller, e.g. a compressor, is to change (or control) the latter. A clear understanding of it's meaning is thus of importance before the actual process is described. First of all though, some definitions are needed.

\subsection{Peak Amplitude and Envelope}
For a periodic signal, the \emph{peak amplitude} can be unambiguously defined as the maximum absolute value over the signal's period. For a signal $x(t)$ with period $\tau$ the peak amplitude is thus defined as
%================================
\begin{equation}
a_\text{pk} \equiv \max_{0 \leq t < \tau} |x(t)|.
\end{equation}
The envelope $e(t)$ of an \emph{analytic signal} $x(t) = a(t)e^{i(\omega t)}$ is defined as \cite{bedrosian1962analytic}
\begin{align}
e(t) \equiv |x(t)| = a(t)
\end{align}
where $e^{i(\omega t)}$ is referred to as the \emph{fine structure} of the signal. For a real-valued amplitude modulated sine wave
%================================
\begin{align}
x(t) = a_\text{pk}(t)\sin(2 \pi f t)
\end{align}
%================================
the notion of an envelope can be generalised by letting the instantaneous (or time-varying) peak amplitude, $a_\text{pk}(t)$, be referred to as the envelope, with $\sin(2 \pi f t)$ being the fine-structure. An example of a signal consisting of an envelope and a fine-structure is depicted in Fig.~\ref{fig:analytic_signal}. 
%================================
\begin{figure}
\captionsetup{justification=centering}
\begin{subfigure}{\linewidth}
\centering
\centerline{\subfile{\rootdir/figures/signal_env_fine_struct}}
\caption{Complete signal}
\label{fig:signal_env_fine_struct}
\end{subfigure}
\par\bigskip
\begin{subfigure}{.5\linewidth}
\centering
\subfile{\rootdir/figures/signal_env}
\caption{Envelope}
\label{fig:signal_env}
\end{subfigure}
\begin{subfigure}{.5\linewidth}
\centering
\subfile{\rootdir/figures/signal_fine_struct}
\caption{Fine structure}
\label{fig:signal_fine_struct}
\end{subfigure}%
\caption{Signal $x(t) = e(t)\cdot f(t)$ split into it's envelope $e(t)$ and fine structure $f(t)$.}
\label{fig:analytic_signal}
\end{figure}
%================================

For real-world signals however, it is difficult to clearly define an envelope. As expressed in \cite{bedrosian1962analytic}: \emph{"Theoretically, the signal must be analytic to permit envelope detection"}. In such cases it can be seen as a line outlining the extremes of the signal and is acquired by various envelope detection methods, as discussed in section \ref{level_detection}. 
%================================
\subsection{RMS}
The \emph{root-mean-square amplitude} (RMS) of a periodic signal $x(t)$ with period time $T$ is defined as
%================================
\begin{equation}
a_\text{RMS} \equiv \sqrt{ \frac{1}{T} \int_{0}^{T} x(t)^2 dt }.
\end{equation}
%================================
For a sine wave with peak amplitude $a_\text{pk}$ and period $\tau$,
\begin{equation}\label{eq:sine_wave}
x(t) = a_\text{pk} \sin \left( \frac{2 \pi}{\tau}t \right), 
\end{equation}
the RMS amplitude can be calculated as
%================================
\begin{equation}\label{eq:a_RMS}
a_\text{RMS} =
\sqrt{ \frac{1}{T} \int_{0}^{T} a^2 \sin^2 (2 \pi f t) dt } =
\sqrt{ \frac{a^2}{2T} \int_{0}^{T}\left( 1 - \cos (4 \pi f t) \right) dt } =
\frac{a_\text{pk}}{\sqrt 2}.
\end{equation}
The factor $\sqrt 2$ is referred to as the \emph{crest factor} and is defined as
\begin{align}
\crest \equiv \frac{a_\text{pk}}{a_\text{RMS}}. \label{eq:crest_factor}
\end{align}
%================================
\subsection{Sound Level}
While the perceived \emph{loudness} of a sound is a subjective quantity \cite{american1973american}, \emph{sound level} is a physical quantity that can be measured. \emph{Sound pressure level} (SPL) is the effective pressure deviation of an acoustic wave in air relative to a reference value, measured on a logarithmic scale \cite{davis1989sound}. More formally,
%================================
\begin{equation}
L_{p/p0} = 10 \logten \left( \frac{p_\text{RMS}^2}{p_0^2} \right) \dB = 20 \logten \left( \frac{p_\text{RMS}}{p_0} \right) \dB,
\end{equation}
%================================
where $p_\text{RMS}$ is the root-mean-square (RMS) pressure deviation and $p_0 > 0$ the reference value. A common reference value is the approximate threshold of human hearing, $p_0 = 20 \mPa$, denoted here by dB SPL\footnote{Although indicating the reference value by a suffix to the dB-unit is not an accepted SI standard, it is a common practice we will adopt in this thesis.} \cite{davis1989sound}:
%================================
\begin{equation}
L_{p/20 \mu\text{Pa}} = 20 \logten \left( \frac{p_\text{RMS}}{20 \mPa} \right) \dBSPL.
\end{equation}
%================================
For analog electric signals, the level of a signal can be measured in terms of it's power relative a reference:
%================================
\begin{equation}
L_{P/P_0} = 10 \logten \left( \frac{P}{P_0} \right) \dB
\end{equation}
%================================
Since the effective power is proportional to the square of the RMS voltage, the signal level can also be measured as
%================================
\begin{equation}
L_{u/u_0} = 10 \logten \left( \frac{u_\text{RMS}^2}{u_0^2} \right) \dB = 20 \logten \left( \frac{u_\text{RMS}}{u_0} \right) \dB,
\end{equation}
%================================
where $u_\text{RMS}$ is the RMS voltage and $u_0$ a reference voltage \cite{davis1989sound}. Various reference powers and voltages exists, for example 
%================================
\begin{itemize}
\item \textbf{dBm} - $P_0 = 1$ mW corresponding to $u_0 \approx 0.775 V$ for an 600 $\Omega$ impedance
\item \textbf{dBu} - $u_0 = 0.775$ V and no impedance given (u for unloaded)
\item \textbf{dBV} - $u_0 = 1$ V and no impedance given.
\end{itemize}
%================================
\subsection{Signal Level}
A convenient unit in digital audio systems is dB relative to \emph{Full scale} (dB FS)~\cite{db_fullscale}. When measuring \emph{peak signal level}, a peak amplitude of $a_{0, \text{pk}} = 1$ is used as reference. The peak signal level of the sine wave in Eq.~\eqref{eq:sine_wave} is thus
\begin{equation}
A_\text{pk} = 10 \logten \left( \frac{a_\text{pk}^2}{a_{0, \text{pk}}^2} \right) \text{dB FS} = 20 \log_{10} a_\text{pk} \dBFS.
\end{equation}

%================================
When measuring the RMS signal level, $a_{0, \text{rms}} = 1$ is used as reference\footnote{The  AES Standard AES17-1998 uses $a_{0, \text{rms}} = 1/sqrt(2)$ as reference. This convention is not adopted in this thesis, since $a_{0, \text{rms}} = 1$ simplifies conversion between linear and logarithmic RMS values}
For the same sine wave then, the RMS signal level is given by
\begin{equation}
A_\text{RMS} = 20\log_{10} \left( \frac{a_\text{pk}}{\sqrt 2} \right) \approx A_\text{pk}- 3.01 \dBFS.
\end{equation}

For periodic signals, $A_\text{pk}$ and $A_\text{RMS}$, are uniquely determined. For real-world signals however, the issue at hand is to define methods for how these are to be extracted. Essentially, it depends on the \emph{time scale} over where these are measured. This is a fundamental part of dynamic range control and is referred to as \emph{level detection}, discussed in section~\ref{level_detection}. 

To distinguish between linear and logarithmic quantities the former are in this study denoted by lowercase and the latter by uppercase letters\footnote{There are a few exceptions, for example the compression ratio $R$ and sample period $T_s$, but since they do not relate to amplitude or signal level there is little risk of ambiguity.}.

\subsection{Digital Signals}
This thesis covers the topic of \emph{digital} dynamic range compression, and the considered signals will be discrete time. A sampled signal $x$ will be indexed by the suffix $n$ to denote the sample number. For example the sampled, discrete time, version of the sine wave in Eq. \eqref{eq:sine_wave} is denoted
\begin{equation}
x_n = a_\text{pk} \sin \left( \frac{2 \pi}{\tau} n T_s \right),
\end{equation}
where $T_s$ is the sampling period. The sampling frequency will be denoted by $f_s$.

\subsection{Definition of Dynamic Range in DDRC}
Given a time-varying detected peak or RMS level,  denoted by$A_{\text{pk},n}$ and $A_{\text{RMS},n}$ respectively, the dynamic range $S$ of a signal can be defined as
\begin{equation}\label{eq:s_peak}
S_\text{pk} \equiv \max(A_{\text{pk},n}) - \min(A_{\text{pk},n})
\end{equation}
for peak levels, and
\begin{equation}\label{eq:s_RMS}
S_\text{RMS} \equiv \max(A_{\text{RMS},n}) - \min(A_{\text{RMS},n})
\end{equation}
for RMS levels.

\end{document}