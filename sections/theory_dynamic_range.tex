\documentclass[../main2.tex]{subfiles}
%if compiling standalone, rootdir wil be previous folder,
%if compiling main document, rootdir will already be set by main file
\providecommand{\rootdir}{..}

\begin{document}
%================================
\subsection{Peak Amplitude, Envelope and RMS}
Before plunging into the realm of dynamic range compression, a brief discussion of
the peak amplitude and the root-mean-square of time-varying signals are presented as these are used frequently throughout the thesis.
\subsubsection{Peak Amplitude}
For a periodic signal, the \emph{peak amplitude} can be unambiguously defined as the maximum absolute value over the signal's period. For a signal $x(t)$ with period $\tau$ the peak amplitude is thus defined as
%================================
\begin{equation}
a_\text{pk} = \sup \{ |x(t)| :  0 \leq t < \tau \}
\end{equation}
%================================
In the case of a sine wave with frequency $f$ the peak amplitude can be calculated as
%================================
\begin{equation}\label{eq:sine_wave}
a_\text{pk} = \sup \{ |a \sin(2 \pi f t) | : 0 \leq t < \frac{1}{f} \} = a.
\end{equation}
%================================
\subsubsection{Envelope and Fine Structure}\label{theory_envelope}
The envelope of an analytic signal $x(t) = a(t)e^{i(\omega t)}$ is defined as \cite{bedrosian1962analytic}
\begin{align}
e(t) \equiv |x(t)| = a(t)
\end{align}
where $e^{i(\omega t)}$ is referred to as the \emph{fine structure} of the signal.

For a real-valued amplitude modulated sine wave
%================================
\begin{align}
x(t) = a(t)\sin(2 \pi f t)
\end{align}
%================================
the instantaneous (or time-varying) amplitude is referred to as the envelope, with the sine wave being the fine-structure. The idea is illustrated in Fig. ~\ref{fig:signal_env_fine_struct}, ~\ref{fig:signal_env} and~\ref{fig:signal_fine_struct} . 
%================================
\begin{figure}
\captionsetup{justification=centering}
\begin{subfigure}{\linewidth}
\centering
\centerline{\subfile{\rootdir/figures/signal_env_fine_struct}}
\caption{Complete signal}
\label{fig:signal_env_fine_struct}
\end{subfigure}
\par\bigskip
\begin{subfigure}{.5\linewidth}
\centering
\subfile{\rootdir/figures/signal_env}
\caption{Envelope}
\label{fig:signal_env}
\end{subfigure}
\begin{subfigure}{.5\linewidth}
\centering
\subfile{\rootdir/figures/signal_fine_struct}
\caption{Fine structure}
\label{fig:signal_fine_struct}
\end{subfigure}%
\caption{Signal $x_n = e_n\cdot f_n$ split into it's envelope $e_n$ and fine structure $f_n$.}
\label{fig:analytic_signal}
\end{figure}
%================================

With real-world signals it is however difficult to clearly define an envelope. As expressed in \cite{bedrosian1962analytic}: \emph{"Theoretically, the signal must be analytic to permit envelope detection"}. In such cases it can be seen as the line outlining the extremes of the signal and is acquired by various envelope detection methods, as discussed in \ref{level_detection}. 
%================================
\subsubsection{RMS}
The \emph{root-mean-square amplitude} (RMS) of a periodic signal $x(t)$ with period time $T$ is defined as
%================================
\begin{equation}
a_\text{rms} = \sqrt{ \frac{1}{T} \int_{0}^{T} x(t)^2 dt }.
\end{equation}
%================================
For the signal in Eq.~\eqref{eq:sine_wave}, the RMS amplitude can be calculated as
%================================
\begin{equation}
a_\text{rms} =
\sqrt{ \frac{1}{T} \int_{0}^{T} a^2 \sin^2 (2 \pi f t) dt } =
\sqrt{ \frac{a^2}{2T} \int_{0}^{T}\left( 1 - \cos (4 \pi f t) \right) dt } =
\frac{a_\text{pk}}{\sqrt 2}.
\end{equation}
%================================
\subsection{Dynamic Range and Signal Level}
The term \emph{dynamic range} has two different meanings in the context of digital audio. For a DSP \emph{system}, the dynamic range is the ratio between the largest representable signal to the quantisation error and is closely related to \emph{signal to noise ratio} (SNR)  \cite{wilson1993filter}. For an audio \emph{signal} on the other hand, the dynamic range is the ratio between the \emph{level} of the loudest portion and the quietest \cite{davis1989sound}. The goal of a dynamic range controller, e.g. a compressor, is to change (or control) the latter. A clear understanding of it's meaning is thus of importance before the actual process is described. 

To distinguish between linear and logarithmic quantities the former are in general denoted by lowercase and the latter by uppercase letters\footnote{There are a few exceptions, for example the compression ratio $R$ and sample period $T_s$.}.
%================================
\subsubsection{Sound Level}
The problem lies at defining the level of a signal. While the perceived \emph{loudness} of a sound is a subjective quantity \cite{american1973american}, \emph{sound level} is a physical quantity that can be measured. \emph{Sound pressure level} (SPL) is the effective pressure deviation of an acoustic wave in air relative to a reference value, measured on a logarithmic scale. More formally,
%================================
\begin{equation}
L_{p/p0} = 10 \logten \left( \frac{p_\text{rms}^2}{p_0^2} \right) \dB = 20 \logten \left( \frac{p}{p_0} \right) \dB,
\end{equation}
%================================
where $p_\text{rms}$ is the root-mean-square (RMS) pressure deviation and $p_0$ the reference value \cite{XXXXX}. A common reference value is the threshold of human hearing, $p_0 = 20 \mPa$ \cite{XXXX}, denoted here by dB SPL\footnote{Although indicating the reference value by a suffix to the dB-unit is not an accepted SI standard, it is a common practice we will adopt in this thesis.}:
%================================
\begin{equation}
L_{p/20 \mu\text{Pa}} = 20 \logten \left( \frac{p_\text{rms}}{20 \mPa} \right) \dBSPL.
\end{equation}
%================================
For analog electric signals, the level of a signal can be measured in terms of it's power relative a reference:
%================================
\begin{equation}
L_{P/P_0} = 10 \logten \left( \frac{P}{P_0} \right) \dB
\end{equation}
%================================
Since the effective power is proportional to the square of the RMS voltage, the signal level can also be measured as
%================================
\begin{equation}
L_{u/u_0} = 10 \logten \left( \frac{u_\text{rms}^2}{u_0^2} \right) \dB = 20 \logten \left( \frac{u_\text{rms}}{u_0} \right) \dB,
\end{equation}
%================================
where $u_\text{rms}$ is the RMS voltage and $u_0$ a reference voltage. Various reference powers and voltages exists, for example 
%================================
\begin{itemize}
\item \textbf{dBm} - $P_0 = 1$ mW corresponding to $u_0 \approx 0.775 V$ for an 600 $\Omega$ impedance
\item \textbf{dBu} - $u_0 = 0.775$ V and no impedance given (u for unloaded)
\item \textbf{dBV} - $u_0 = 1$ V and no impedance given.
\end{itemize}
%================================
\subsubsection{Signal Level}
A convenient unit in digital audio systems is dB \emph{Full scale}~\cite{db_fullscale} or dB FS used in this thesis for measuring \emph{signal level} with peak amplitude of $a_{0, \text{pk}} = 1$ as reference. The signal level of the peak amplitude of a sine wave in Eq.~\eqref{eq:sine_wave} is thus
\begin{equation}
A_\text{pk} = 10 \logten \left( \frac{a_\text{pk}^2}{a_{0, \text{pk}}^2} \right) \text{dB FS} = 20 \log_{10} a_\text{pk} \dBFS.
\end{equation}
%================================
With this definition, the RMS amplitude of a sine wave is
\begin{equation}
A_\text{rms} = 20\log_{10} \left( \frac{a_\text{pk}}{\sqrt 2} \right) \approx A_\text{pk} - 3.01 \dBFS.
\end{equation}
This is not in accordance with the EBU 3341 standard, but is used  since it simplifies the conversion between linear and logarithmic RMS values.

For periodic signals, $A_\text{pk}$ and $A_\text{rms}$, are uniquely determined. For real-world signals however, the issue at hand is to define methods for how these are to be extracted. Essentially, it depends on the \emph{time scale} over where these are measured. This is a fundamental part of dynamic range control and is referred to as \emph{level detection} and is discussed in section~\ref{level_detection}.

\end{document}
