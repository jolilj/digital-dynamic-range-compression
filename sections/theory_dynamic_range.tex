\documentclass[../main2.tex]{subfiles}
%if compiling standalone, rootdir wil be previous folder,
%if compiling main document, rootdir will already be set by main file
\providecommand{\rootdir}{..}

\begin{document}

\subsection{Dynamic Range, Signal Level and Loudness}\label{theory_dynamic_range}
As the name suggests, the goal of dynamic range compression is to change the dynamic range of an audio signal. The term \emph{dynamic range} has two different meanings in the context of digital audio. For a DSP \emph{system}, the dynamic range is the ratio between the largest representable signal to the quantisation error and is closely related to \emph{signal to noise ratio} (SNR)  \cite{wilson1993filter}. For an audio \emph{signal} on the other hand, the dynamic range is the ratio between levels of the loudest portion and the quietest \cite{davis1989sound}. The latter quantity is what is referred to in the term DRC and throughout this thesis. Unfortunately, even with this definition the term is still ambiguous as the "level of a portion of audio" is not a well defined concept.

\subsubsection{Signal Level}
\emph{Todos}
\begin{itemize}
\item Amplitude -> periodic signals -> peak and RMS
\item For more complicated, aperiodic signals, we have no period to measure these quantities over. Therefore, their value depends greatly upon the timeframe chosen.
\item reference the article of RMS in different timespans. Own experiment?
\end{itemize}

\subsubsection{Peak Amplitude}
With \emph{peak amplitude} we will mean the maximum absolute value of the signal. For a sine wave
\begin{equation}\label{eq:sine_wave}
x = a \sin(2 \pi f t),
\end{equation}
the peak amplitude is equal to $a$.

When expressed in decibel, we will use the peak amplitude of $a_0=1$ as reference and call this quantity dB relative to \emph{Full Scale} (dB FS) \cite{XXXX}. The peak amplitude of the sine wave in Eq.~\eqref{eq:sine_wave} is thus
\begin{equation}
A = 10 \logten \left( \frac{a^2}{a_0^2} \right) \text{dB FS} = 20 \log_{10} a \dBFS.
\end{equation}
Throughout the thesis, quantities relating to signal level or amplitude will be denoted by upper-case letters in the logarithmic domain, and lower-case letters in the linear.

\subsubsection{Envelope and Fine Structure}
The envelope of an analytic signal $x(t) = a(t)e^{i(\omega t)}$ is defined as \cite{bedrosian1962analytic}
\begin{align}
e(t) \equiv |x(t)| = a(t)
\end{align}
where $e^{i(\omega t)}$ is referred to as the \emph{fine structure} of the signal.

With more complex signals it is however difficult to clearly define an envelope. As expressed in \cite{bedrosian1962analytic}: \emph{"Theoretically, the signal must be analytic to permit envelope detection"}. In such cases it can be seen as the line outlining the extremes of the signal and is acquired by various envelope detection methods, as discussed in \ref{level_detection}. The idea is illustrated in Fig. ~\ref{fig:signal_env_fine_struct}, ~\ref{fig:signal_env} and~\ref{fig:signal_fine_struct} . 
%================================
\begin{figure}
\captionsetup{justification=centering}
\begin{subfigure}{\linewidth}
\centering
\centerline{\subfile{\rootdir/figures/signal_env_fine_struct}}
\caption{Complete signal}
\label{fig:signal_env_fine_struct}
\end{subfigure}
\par\bigskip
\begin{subfigure}{.5\linewidth}
\centering
\subfile{\rootdir/figures/signal_env}
\caption{Envelope}
\label{fig:signal_env}
\end{subfigure}
\begin{subfigure}{.5\linewidth}
\centering
\subfile{\rootdir/figures/signal_fine_struct}
\caption{Fine structure}
\label{fig:signal_fine_struct}
\end{subfigure}%
\caption{Signal $x_n = e_n\cdot f_n$ split into it's envelope $e_n$ and fine structure $f_n$.}
\label{fig:analytic_signal}
\end{figure}
%================================

\subsubsection{RMS}
The \emph{root-mean-square amplitude} (RMS) of a periodic signal $x(t)$ with period time $T$ is defined as
\begin{equation}
a_\text{rms} = \sqrt{ \frac{1}{T} \int_{0}^{T} x(t)^2 dt }.
\end{equation}
For the signal in Eq.~\eqref{eq:sine_wave}, the RMS amplitude can be calculated as
\begin{equation}
a_\text{rms} =
\sqrt{ \frac{1}{T} \int_{0}^{T} a^2 \sin^2 (2 \pi f t) dt } =
\sqrt{ \frac{a^2}{2T} \int_{0}^{T}\left( 1 - \cos (4 \pi f t) \right) dt } =
\frac{a}{\sqrt 2}.
\end{equation}

The square wave is a special case with $a_\text{peak} = a_\text{rms}$. When expressed in decibel, we will use the RMS amplitude of a square wave with amplitude $a_\text{peak} = a_\text{rms} = 1$ as reference:
\begin{equation}
A_\text{rms} =
20 \log_{10} \left( \frac{a_\text{rms}}{a_{0, \text{rms}}} \right) \text{dB FS} =
20 \log_{10} \left( a_\text{rms} \right) \text{dB FS}.
\end{equation}.
With this definition, the RMS amplitude of a sine wave is thus
\begin{equation}
A_\text{rms} = 20\log_{10} \left( \frac{a_\text{peak}}{\sqrt 2} \right) = A - 3.01 \text{dB FS}
\end{equation}
This is not in accordance with the EBU 3341 standard, but is used here as setting $a_{0, \text{rms}} = 1$ simplifies conversion between linear RMS values and logarithmic.

\subsubsection{Timescales}
Note that we will not cover the concept of perceived \emph{loudness} as defined by for example the ITU recommendation BS 1770 standard, or the Phon, Sone and dB(A) measures.

\emph{todos}
\begin{itemize}
	\item Dynamic Range, not signal to noise ratio, but rather Level variability
	\item Sound Level (dBFS, peak and rms), and Loudness
	\item Envelope
\end{itemize}

\end{document}