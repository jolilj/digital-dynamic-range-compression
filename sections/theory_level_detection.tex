\documentclass[../main2.tex]{subfiles}
%if compiling standalone, rootdir wil be previous folder,
%if compiling main document, rootdir will already be set by main file
\providecommand{\rootdir}{..}

\begin{document}
\subsection{Level Detection}\label{level_detection}
Level detection (logarithmic domain) or envelope estimation (linear domain), see Fig.~\ref{fig:block_genericDDRC}, is the process of extracting the level, or envelope, of the input signal. Various methods have been proposed for this task in the literature. Here follows a summary of the algorithms investigated in this thesis and a short analysis of their properties. The input signal will be denoted $x_n$, the estimated envelope $\tilde{e}_n$, the rms $\hat{x}$ and the detected level, either envelope based or rms-based, $L_n = 20\logten(\tilde{e})$ or $\hat{L}_n=20\logten(\hat{x})$, respectively. 

\subsubsection{Full Wave Rectification}\label{full_wave_rect}
Full wave rectification is performed in the digital domain by simply taking the absolute value of the input signal. As shown in Appendix~\ref{non_lin_ops}, the nonlinearity of this operation introduces a phase dependent oscillating or static error at certain frequencies as their harmonics get aliased to near or at 0 Hz. Nevertheless, it is used without modification in~\cite{reiss2012tutorial}, and as a first step in all following peak detectors.

\subsubsection{Peak Detection by Nonlinear Filtering}\label{peak_detection}
In \cite{mcnally1984dynamic, zolzer1997digital}, the higher frequency harmonics from the full wave rectification is filtered out by a nonlinear IIR-filter, estimating the envelope. The difference equation for the filter is
\begin{equation}\label{eq:analog_det}
\tilde{e}_n = \begin{cases}
    \alpha_\text{la} |x_n| + (1- (\alpha_\text{la} + \alpha_\text{lr})) \tilde{e}_{n-1}  	& |x_n| \geq \tilde{e}_{n-1} \\
    (1-\alpha_\text{lr}) \tilde{e}_{n-1} 							& |x_n| < \tilde{e}_{n-1}.
\end{cases}
\end{equation}
where the coefficients $\alpha_\text{la}$ and $\alpha_\text{lr}$ are defined by
\begin{align}
\alpha_\text{la} &= 1 - e^{-1/\tau_\text{la} f_s } \label{eq:attack_coeff_def}\\
\alpha_\text{lr} &= 1 - e^{-1/\tau_\text{lr} f_s } \label{eq:release_coeff_def}.
\end{align}
For a discussion about the meaning of the time constants $\tau_\text{lr}$ and $\tau_\text{la}$, see Appendix~\ref{appendix_time_constants}.

The corresponding block representation of Eq.~\eqref{eq:analog_det} can be seen in Fig.~\ref{fig:block_mcnally_theory_peak}. A problem with the design is that the release time constant is leaked into the attack phase, which introduces a static error, as shown in \cite{reiss2012tutorial}. However, in \cite{mcnally1984dynamic}, the attack time is assumed to be $\sim 10$ ms and the release time   $\sim 100$ ms thus making $\alpha_r << \alpha_a$. As shown in \cite{reiss2012tutorial}, the design can be derived from an ideal model of a simple analog RC-circuit design, and is therefore named the \emph{analog peak detector} in this thesis.
%================================
\begin{figure}
\centerline{\subfile{\rootdir/figures/block_mcnally_theory_peak}}
\caption{Block diagram of the peak detector in \cite{mcnally1984dynamic}, referred to as the analog peak detector in this thesis.}
\label{fig:block_mcnally_theory_peak}
\end{figure}
%================================

In \cite{zolzer2008digital} the release time coefficient, $\alpha_\text{lr}$, is removed from the attack phase so that the level measurement and attack time is no longer scaled with release time
\begin{equation}
\tilde{e}_n = \begin{cases}
    \alpha_\text{la} |x_n| + (1-\alpha_\text{la}) \tilde{e}_{n-1} 	& |x_n| > \tilde{e}_{n-1} \\
    (1-\alpha_\text{lr}) \tilde{e}_{n-1} 						& |x_n| \leq \tilde{e}_{n-1}.
\end{cases}
\end{equation}
This filter is discussed in \cite{reiss2012tutorial} and is referred to as the \emph{branching peak detector}. The same name will be used in this thesis for consistency.

Another variant is described in \cite{stikvoort1986digital}, where the attack is instant, and the release phase is smoothed by a third order IIR filter:
\begin{equation}\label{eq:stikvoort_release}
\begin{split}
u_n &= \text{max}(|x_n|, (1-\alpha_\text{lr}) u_{n-1}) \\
v_n &= \text{max}(|x_n|, \alpha_\text{lr} u_n + (1-\alpha_\text{lr}) v_{n-1} \\
\tilde{e}_n &= \text{max}(|x_n|, \alpha_\text{lr} v_n + (1-\alpha_\text{lr}) \tilde{e}_{n-1}. \\
\end{split}
\end{equation}
The release coefficient, $\alpha_\text{lr}$, is as before defined by Eq.~\eqref{eq:release_coeff_def}, but it is worth noting that the actual release time is approximately equal to $ 3.25 \tau_\text{lr}$ \cite{stikvoort1986digital}.

The filter functions as both a peak-hold and a smooth release effect as can be seen in Fig.~\ref{fig:step_stikvoort_release}. This detector will be referred to as the \emph{third order peak detector}.
%=========================
\begin{figure}[h]
\centerline{\subfile{\rootdir/figures/step_stikvoort_release}}
\caption{Step and downstep response of third order release filter as defined in Eq.~\eqref{eq:stikvoort_release}, with $\tau_\text{lr} = 150 \text{ms}$. Notice the slight hold-effect between $t=0.5$ s and $t \approx 0.7$ s.}
\label{fig:step_stikvoort_release}
\end{figure}
%=========================

\subsubsection{Upsampling Peak Detector}\label{theory_upsampling_peak_det}
None of the previously described peak detectors handles the error introduced by harmonics aliased to near or at 0 Hz in the full wave rectification step. \cite{frindle1996implementation} notes that this problem decreases with increasing sample rate, and suggests up-sampling before peak detection as a solution. This however increases the computational load by the up-sampling factor, and thus another similar but more computationally efficient solution is proposed.

The input signal is filtered by 4 different 5th order FIR filters with a phase shift of 2.125, 2.375, 2.625 and 2.875 samples respectively. The filters are constructed by taking a sinc-filter with cutoff frequency at $f_s/2$ and window it with a raised cosine window function. In mathematical terms:
\begin{equation}
\begin{split}
f^1_n &= a_0 x_{n} + a_1 x_{n-1} + a_2 x_{n-2} + a_3 x_{n-3} + a_4 x_{n-4} + a_5 x_{n-5} \\
f^2_n &= b_0 x_{n} + b_1 x_{n-1} + b_2 x_{n-2} + b_3 x_{n-3} + b_4 x_{n-4} + b_5 x_{n-5} \\
f^3_n &= c_0 x_{n} + c_1 x_{n-1} + c_2 x_{n-2} + c_3 x_{n-3} + c4 x_{n-4} + c_5 x_{n-5} \\
f^4_n &= d_0 x_{n} + d_1 x_{n-1} + d_2 x_{n-2} + d_3 x_{n-3} + d_4 x_{n-4} + d_5 x_{n-5} \\
\end{split}
\end{equation}
where the coefficents are given by
\begin{equation}
\begin{split}
a_k &= \frac{\sin (2 \pi \frac{1}{2} (0.125+k-3) )}{(0.125+k-3)\pi} \cdot \frac{1}{2}\left[1+\cos \left(\frac{2 \pi (0.125+k-3)}{6} \right) \right] \\
b_k &= \frac{\sin (2 \pi \frac{1}{2} (0.375+k-3) )}{(0.375+k-3)\pi} \cdot \frac{1}{2}\left[1+\cos \left(\frac{2 \pi (0.375+k-3)}{6} \right) \right] \\
c_k &= \frac{\sin (2 \pi \frac{1}{2} (0.625+k-3) )}{(0.625+k-3)\pi} \cdot \frac{1}{2}\left[1+\cos \left(\frac{2 \pi (0.625+k-3)}{6} \right) \right] \\
d_k &= \frac{\sin (2 \pi \frac{1}{2} (0.875+k-3) )}{(0.125+k-3)\pi} \cdot \frac{1}{2}\left[1+\cos \left(\frac{2 \pi (0.875+k-3)}{6} \right) \right]. \\
\end{split}
\end{equation}
The peak estimate is then taken as the maximum absolute value of the four filter outputs:
\begin{equation}
\tilde{e}_n = \text{max}(|f^1_n|, |f^2_n|, |f^3_n|, |f^4_n|)
\end{equation}
Since this is equivalent to up-sampling by a factor of 8 and taking the maximum absolute value of the 1st, 3rd, 5th and 7th interpolated value as peak value, it will be referred to as the \emph{upsampling peak detector}.

Lastly, the calculated peak envelope is subject to hysteresis of 1 $\dB$ to further decrease the oscillating peak estimation errors. The exact implementation of this hysteresis curve is not revealed in the paper and cannot be further analysed here.

\subsubsection{Windowed Max Peak Detector}
In \cite{hamalainen2002smoothing} a windowed max filter is implemented as a component in a envelope detector constructed to avoid clipping in the output of a digital limiter. The \emph{windowed max peak detector} takes the maximum value of the absolute value of the input signal over a window of $M$ samples:
\begin{align}\label{eq:window_max_det}
\tilde{e}_n = \text{max}(|x_{n-M/2}|,|x_{n-M/2+1}|,..., |x_{n+M/2-1}|).
\end{align}

\subsubsection{RMS Detection}
The RMS of a signal computed on a window of $M$ samples is defined as
\begin{equation}
\hat{x}_n \equiv \sqrt{ \frac{1}{M} \sum_{m=-M/2}^{M/2-1} x_{n-m}^2}
\end{equation}
and is used in \cite{reiss2010rev, bosi1991low}. It will be referred to as the \emph{Windowed RMS detector}.

Another approach proposed in \cite{mcnally1984dynamic} is to only take previous samples into account and weight them by an IIR-filter:
%================================
\begin{align*}
\hat{x}_n = \sqrt{\alpha_{\text{av}} x_{n}^2+ (1-\alpha_{\text{av}}) \hat{x}_{n-1}^2}
\end{align*}
%================================
where $\alpha_\text{av}$ is the filter/averaging coefficient. It is related to an averaging time constant, $\tau_\text{av}$, by 
\begin{equation}
\alpha_\text{av} = 1 - e^{-1/\tau_\text{av} f_s } \label{eq:average_coeff_def}.
\end{equation}
A block representation of this RMS detector, referred to as the \emph{IIR RMS detector}, is shown in Fig.~\ref{fig:block_mcnally_theory_rms}.
%================================
\begin{figure}[h]
\centerline{\subfile{\rootdir/figures/block_mcnally_theory_rms}}
\caption{Block diagram of the RMS level detector in \cite{mcnally1984dynamic}}
\label{fig:block_mcnally_theory_rms}
\end{figure}
%================================
\end{document}
