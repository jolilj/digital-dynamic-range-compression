\documentclass[../main2.tex]{subfiles}
%if compiling standalone, rootdir wil be previous folder,
%if compiling main document, rootdir will already be set by main file
\providecommand{\rootdir}{..}

\begin{document}
\subsection{Level Detector}\label{level_detection}
Level detection (logarithmic domain), envelope estimation (linear domain), is the process of extracting the level, or envelope, of the input signal. Various methods have been proposed for this task in the literature. Here follows a summary of the algorithms investigated in this thesis and a short analysis of their properties. The input signal will be denoted $x_n$, the estimated envelope $\tilde{e}_n$, the rms $\hat{x}$ and the detected level, either envelope based or rms-based, $L_n = 20\logten(\tilde{e})$ or $\hat{L}_n=20\logten(\hat{x})$. 

\subsubsection{Full Wave Rectification}
Full wave rectification is performed in the digital domain by simply taking the absolute value of the input signal. As shown in appendix A, the nonlinearity of this operation introduces a phase dependent oscillating or static error at certain frequencies as their harmonics get aliased to near or at 0 Hz. Nevertheless, it is used without modification in~\cite{reiss2012tutorial}, and as a first step in the peak detectors in \cite{mcnally1984dynamic} \cite{stikvoort1986digital}. \todo{add references to zoelzers stuff? Lu? Other?}\subsubsection{Peak Detection}\label{peak_detection}
In \cite{mcnally1984dynamic} and \cite{zolzer1997digital}, the higher frequency harmonics from the full wave rectification is filtered out by an IIR-filter, estimating the envelope, defined as
\begin{equation}
\tilde{e}_n = \begin{cases}
    \alpha_{a} |x_n| + (1- (\alpha_{a} + \alpha_{r})) \tilde{e}_{n-1}  	& |x_n| \geq \tilde{e}_{n-1} \\
    (1-\alpha_{r}) \tilde{e}_{n-1} 							& |x_n| < \tilde{e}_{n-1}.
\end{cases}
\end{equation}
Its corresponding block representation can be seen in Fig.~\ref{fig:block_mcnally_theory_peak}. The release time constant is leaked into the attack phase, which introduces a static error, as shown in \cite{reiss2012tutorial}. However, in \cite{mcnally1984dynamic}, the attack time is assumed to be $\sim 10$ ms and the release time   $\sim 100$ ms thus making $\alpha_r << \alpha_a$. As shown in \cite{reiss2012tutorial}, the design can be derived from an ideal model of a simple analog RC-circuit design, and is therefor named the \emph{analog peak detector} accordingly.
%================================
\begin{figure}
\centerline{\subfile{\rootdir/figures/block_mcnally_theory_peak}}
\caption{Block diagram of the peak level detector in \cite{mcnally1984dynamic}}
\label{fig:block_mcnally_theory_peak}
\end{figure}
%================================

In \cite{zolzer2008digital} the release time constant is removed from the attack phase so that the level measurement and attack time is no longer scaled with release time
\begin{equation}
\tilde{e}_n = \begin{cases}
    \alpha_{a} |x_n| + (1-\alpha_{a}) \tilde{e}_{n-1} 	& |x_n| > \tilde{e}_{n-1} \\
    (1-\alpha_{r}) \tilde{e}_{n-1} 					& |x_n| \leq \tilde{e}_{n-1}.
\end{cases}
\end{equation}
This filter is discussed in \cite{reiss2012tutorial} and referred to as the \emph{branching peak detector}.

Another variant is described in \cite{stikvoort1986digital}, where the attack is instant, and the release phase is smoothed by a third order IIR filter:\begin{equation}\label{eq:stikvoort_release}
\begin{split}
u_n &= \text{max}(|x_n|, (1-\alpha_{\text{r}}) u_{n-1}) \\
v_n &= \text{max}(|x_n|, \alpha_{\text{r}} u_n + (1-\alpha_{\text{r}}) v_{n-1} \\
\tilde{e}_n &= \text{max}(|x_n|, \alpha_{\text{r}} v_n + (1-\alpha_{\text{r}}) \tilde{e}_{n-1}. \\
\end{split}
\end{equation}
The time constant $\alpha_{\text{rel}}$ is defined as
\begin{equation}
\alpha_\text{rel} = 1-e^{-1/\tau_0 f_s}
\end{equation}
where the downstep has fallen to 63.2 \% of it's final value after approximately $3.25 \tau_0$. The filter functions as both peak-hold and smooth release effect as can be seen in Fig.~\ref{fig:step_stikvoort_release}.
%=========================
\begin{figure}
\centerline{\subfile{\rootdir/figures/step_stikvoort_release}}
\caption{Step and downstep response of third order release filter as defined in Eq.~\eqref{eq:stikvoort_release}, with $\tau_0 = 150 \text{ms}$.}
\label{fig:step_stikvoort_release}
\end{figure}
%=========================

\subsubsection{Upsampling Peak Detector}
None of the previously described peak detectors handles the error introduced by harmonics aliased to the filtering's passband. \cite{frindle1996implementation} notes that this problem decreases with increasing sample rate, and suggests up-sampling before peak detection as a solution. This however increases the computational load by the up-sampling factor, and thus another similar but more computationally effectient solution is proposed.

The input signal is filtered by 4 different 5th order FIR filters with a phase shift of 2.125, 2.375, 2.625 and 2.875 samples respectively. The filters are constructed by taking a sinc-filter with cutoff frequency at $f_s/2$ and window it with a raised cosine window function. In mathematical terms:
\begin{equation}
\begin{split}
f^1_n &= a_0 x_{n} + a_1 x_{n-1} + a_2 x_{n-2} + a_3 x_{n-3} + a_4 x_{n-4} + a_5 x_{n-5} \\
f^2_n &= b_0 x_{n} + b_1 x_{n-1} + b_2 x_{n-2} + b_3 x_{n-3} + b_4 x_{n-4} + b_5 x_{n-5} \\
f^3_n &= c_0 x_{n} + c_1 x_{n-1} + c_2 x_{n-2} + c_3 x_{n-3} + c4 x_{n-4} + c_5 x_{n-5} \\
f^4_n &= d_0 x_{n} + d_1 x_{n-1} + d_2 x_{n-2} + d_3 x_{n-3} + d_4 x_{n-4} + d_5 x_{n-5} \\
\end{split}
\end{equation}
where the coefficents are given by
\begin{equation}
\begin{split}
a_k &= \frac{\sin (2 \pi \frac{1}{2} (0.125+k-3) )}{(0.125+k-3)\pi} \cdot \frac{1}{2}\left[1+\cos \left(\frac{2 \pi (0.125+k-3)}{6} \right) \right] \\
b_k &= \frac{\sin (2 \pi \frac{1}{2} (0.375+k-3) )}{(0.375+k-3)\pi} \cdot \frac{1}{2}\left[1+\cos \left(\frac{2 \pi (0.375+k-3)}{6} \right) \right] \\
c_k &= \frac{\sin (2 \pi \frac{1}{2} (0.625+k-3) )}{(0.625+k-3)\pi} \cdot \frac{1}{2}\left[1+\cos \left(\frac{2 \pi (0.625+k-3)}{6} \right) \right] \\
d_k &= \frac{\sin (2 \pi \frac{1}{2} (0.875+k-3) )}{(0.125+k-3)\pi} \cdot \frac{1}{2}\left[1+\cos \left(\frac{2 \pi (0.875+k-3)}{6} \right) \right]. \\
\end{split}
\end{equation}
The peak estimate is then taken as the maximum absolute value of the four filter outputs:
\begin{equation}
\tilde{e}_n = \text{max}(|f^1_n|, |f^2_n|, |f^3_n|, |f^4_n|)
\end{equation}
This can be shown to be equivalent to up-sampling by a factor of 8 and taking the maximum absolute value of the 1st, 3rd, 5th and 7th interpolated value as peak value. \todo{show this here? in an appendix?}

Lastly, the calculated peak envelope is subject to hysteresis of 1dB to further decrease the oscillating peak estimation errors. Sadly, the exact implementation of this hysteresis curve is not revealed in the paper and cannot be further analysed here.

\subsubsection{Peak Windowed Max Detector}
In \cite{hamalainen2002smoothing} a windowed max filter is implemented as a component in a envelope detector constructed to avoid clipping in the output of a digital limiter. The windowed peak max filter takes the maximum value of the absolute value of the input signal over a window of $M$ samples.
\begin{align}
\tilde{e}_n = \text{max}(x_{n-M/2},x_{n-M/2+1},..., x_{n+M/2-1})
\end{align}
\subsubsection{Peak Windowed Max Filtered Detector}
The windowed max detector above can be filtered by a first order averaging filter to achieve a smoother signal
\begin{align}
u_n &= \text{max}(x_{n-M/2},x_{n-M/2+1},..., x_{n+M/2-1}) \\
\tilde{e}_n &= \alpha_{av} u_n + (1-\alpha_{av})\tilde{e}_{n-1}.
\end{align}

\subsubsection{RMS}
The RMS of a signal computed on a window of $M$ samples is defined as
\begin{equation}
\hat{x}_n = \sqrt{ \frac{1}{M} \sum_{m=-M/2}^{M/2-1} x_{n-m}}
\end{equation}
and is used in \cite{reiss2010rev}\cite{bosi1991low}.

Another approach used in \cite{mcnally1984dynamic} is to only take previous samples into account and weight them by an IIR-filter:
%================================
\begin{align*}
\hat{x}_n = \sqrt{\alpha_{\text{av}} x_{n}^2+ (1-\alpha_{\text{av}}) \hat{x}_{n-1}^2}
\end{align*}
%================================
where $\alpha_{av}$ is the filter/averaging coefficient. A block representation of the RMS detector can be seen in Fig.~\ref{fig:block_mcnally_theory_rms}.%================================
\begin{figure}[h]
\centerline{\subfile{\rootdir/figures/block_mcnally_theory_rms}}
\caption{Block diagram of the RMS level detector in \cite{mcnally1984dynamic}}
\label{fig:block_mcnally_theory_rms}
\end{figure}
%================================
\end{document}