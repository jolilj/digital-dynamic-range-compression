\documentclass[../main2.tex]{subfiles}
%if compiling standalone, rootdir wil be previous folder,
%if compiling main document, rootdir will already be set by main file
\providecommand{\rootdir}{..}

\begin{document}
%================================
\subsection{Ideal Compression} \label{ideal_compression}
In order to develop a method for comparing DDRC designs, a definition of ideal compression is needed. Let $x$ be the input and $y$ be the compressed output. Further assume that $x$ can be expressed as a product of it's envelope $e$ and fine-structure $f$ \begin{align}
x_n = e_n\cdot f_n.
\end{align}
The idea is illustrated in Fig. ~\ref{fig:signal_env_fine_struct}, ~\ref{fig:signal_fine_struct} and ~\ref{fig:signal_env}. 

Let $G(x)$ be the gain computer, see section \ref{gain_computer}, and $g(x) = 10^{G/20}$. The ideal compression is then given by
\begin{align}
y_{n,I} \equiv g(e_n) f_n. 
\end{align}
%================================
\begin{figure}[ht]
\centerline{\subfile{\rootdir/figures/signal_env_fine_struct}}
\caption{Signal $x_n = e_n\cdot f_n.$}
\label{fig:signal_env_fine_struct}
\end{figure}
%================================
\begin{figure}[ht]
\captionsetup{justification=centering}
\centerline{
\begin{minipage}[t]{.5\textwidth}
 \centering
\subfile{\rootdir/figures/signal_fine_struct}
\caption{Fine structure of the signal} 
\label{fig:signal_fine_struct}
\end{minipage}%
\begin{minipage}[t]{.5\textwidth}
\centering
\subfile{\rootdir/figures/signal_env}
\caption{Envelope of the signal} 
\label{fig:signal_env}
\end{minipage}
}
\end{figure}
%================================
\end{document}