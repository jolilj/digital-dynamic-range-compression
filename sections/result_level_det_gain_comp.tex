\documentclass[../main2.tex]{subfiles}
%if compiling standalone, rootdir wil be previous folder,
%if compiling main document, rootdir will already be set by main file
\providecommand{\rootdir}{..}

\begin{document}

\subsection{Peak Level and Gain Computer}

fs = 44100 Hz, fc = 10,000 Hz fm = 2 Hz
Amin = -10 dB Amax = 0dB
T = -12dB
R = 4
As can be seen in the optimised parameter tables, the threshold and release is now not equal to the specified $T$ and $R$. Only hard knee was allowed since the envelope was constantly above the threshold anyway.

Observe that the length in the beginning that was skipped in the evaluation may affect the optimal values.

\begin{table}[h]
\begin{center}
\caption{Optimised parameters for the various peak detectors, $f_c=\text{10,000}$ Hz, $f_m=2$ Hz}
\label{tab:peak_det_gain_computer_opt_params}
\caption*{(a) Attack and release peak detectors}
\begin{tabular}{| l | c c c c c | c |}
	\hline
	Detector 	& $\tau_\text{a}$ [ms] & $\tau_\text{r}$ [ms] & $d$ [samples] & $T$ & $R$ & $\EN$ [1/sample]\\
	\hline
	
	Analog 			& 0 		& 135.6 	& 16	(0.36 ms)	& & &?	\\ 
	Branching smooth 	& 0	 	& 49.6 	& 17	(0.39 ms)	& & &?	\\ 
	Cascaded smooth	& 1.2	& 49.5 	& 69	(1.56 ms)	& & &?	\\
	\hline
\end{tabular}
\end{center}

\begin{center}
\caption*{(b) Windowed peak detectors}
 \begin{tabular}{| l | c c c c c | c |}
	\hline
	Detector & $w$ [samples] & $\tau_\text{av}$ [ms] & $d$ [samples] & $T$& $R$ &$\EN$ [1/sample] \\
	\hline
	Win. Max		& 69		& -		& 34	(0.77 ms)	& & &?	\\
	\hline
\end{tabular}
\end{center}

\end{table}

%======================
\begin{figure}[h]
\centerline{\subfile{\rootdir/figures/peak_det_opt_gain}}
\caption{Optimised peak detector gains, vertical offset by 0.2 for clarity. The vertical lines at 0.05 s and 0.95 s mark the start and end of the region for which $\EN$ was calculated.}
\label{fig:peak_det_opt_gain}
\end{figure}
%======================
\begin{figure}

\captionsetup{justification=centering}
\begin{subfigure}{\linewidth}
\centering
\centerline{\subfile{\rootdir/figures/peak_det_opt_gain_min}}
\caption{Peak detector gain, zoomed in at the maximum gain reduction}
\end{subfigure}

\par\bigskip

\captionsetup{justification=centering}
\begin{subfigure}{\linewidth}
\centering
\centerline{\subfile{\rootdir/figures/peak_det_opt_gain_max}}
\caption{Peak detector gain, zoomed in at the minumum gain reduction}
\end{subfigure}

\caption{Zoomed in versions of optimised gain factors, vertically offset by 0.0032 for clarity}
\label{fig:peak_det_opt_gain_zoom}
\end{figure}
%======================

\end{document}