\documentclass[../main2.tex]{subfiles}
%if compiling standalone, rootdir wil be previous folder,
%if compiling main document, rootdir will already be set by main file
\providecommand{\rootdir}{..}

\begin{document}
\subsection{Evaluation of RMS Detectors}\label{method_rms_detectors}
The RMS operation on the fine structure results in the peak amplitude divided by the crest factor, see Eq. \eqref{eq:crest_factor}. Although the RMS is highly dependent on desired window length, a sensible evaluation\footnote{See section~\ref{discussion} for a discussion regarding this matter.} of the performance of the RMS detector is it's ability to track the envelope changes. The ideal RMS can thus be defined as
\begin{align}
\hat{x}_{I,n} \equiv \frac{e_n}{f_\text{crest}}
\end{align}
where $f_\text{crest}$ is the crest factor of the fine structure.

Apart from this, the evaluation was carried out in the same manner as in the peak detection case.
\end{document}