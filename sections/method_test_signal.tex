\documentclass[../main2.tex]{subfiles}
%if compiling standalone, rootdir wil be previous folder,
%if compiling main document, rootdir will already be set by main file
\providecommand{\rootdir}{..}

\begin{document}
%================================
\subsection{Test Signal} \label{test_signal}
To comply with the metrics proposed the test signal should exhibit the following properties
\begin{itemize}
\item{The signal should be a product of an envelope $> 0$ and a fine structure}
\item{ECR measures the modulating depth of the signal in the logarithmic domain. It is therefor beneficial having the envelope defined in the logarithmic domain}
\end{itemize}
\todo{Should we let the envelope be modulated above threshold and test FES aswell?}

The test signal $x_n$ is thus constructed as
%================================
\begin{align}
E_n &= A_{\text{max}} + \left(A_{\text{min}} - A_{\text{max}}\right)\left(1+\cos \left(2\pi\frac{f_m}{f_s}n\right)\right)\\
e_n &= 10^{E_n/20}\\
f_n &= \sin \left( 2\pi \frac{f_c}{f_s} t\right) \\
x_n &= e_n \cdot f_n
\end{align}
%================================
where $A_{\text{max}}$ is the upper bound and $A_{\text{min}}$ is the lower bound of the envelope in the logarithmic domain, $f_m$ is the modulation frequency and $f_c$ the carrier, or fine structure, frequency.
\end{document}