\documentclass[../main2.tex]{subfiles}
%if compiling standalone, rootdir wil be previous folder,
%if compiling main document, rootdir will already be set by main file
\providecommand{\rootdir}{..}

\begin{document}
%================================
\subsection{Test Signal} \label{test_signal}
To comply with the metrics proposed the test signal should exhibit the following properties
\begin{itemize}
\item{The signal should be a product of an envelope $e > 0$ and a real-valued fine structure $f$}
\item{Having maximum and minimum amplitude, $A_\text{max}$ and  $A_\text{min}$, defined in the logarithmic domain with $0 > A_\text{max} > A_\text{min}$.}
\item{If FES is to be measured $A_\text{min} > T$.}
\end{itemize}

The test signal $x_n$ is thus constructed as
%================================
\begin{align}
a_\text{max} &= 10^{A_\text{max}/20} \\
a_\text{min} &= 10^{A_\text{min}/20} \\
e_n &=  a_\text{max} +  (a_\text{min}  - a_\text{max} )\left(1+\cos \left(2\pi f_m \frac{n}{f_s}\right)\right)\\
f_n &= \sin \left( 2\pi f_c\frac{n}{f_s}\right) \\
x_n &= e_n \cdot f_n \label{eq:eq_test_signal}
\end{align}
%================================
where $f_m$ is the modulation frequency and $f_c$ the carrier, or fine structure, frequency.
\end{document}