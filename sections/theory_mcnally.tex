\documentclass[../main2.tex]{subfiles}
%if compiling standalone, rootdir wil be previous folder,
%if compiling main document, rootdir will already be set by main file
\providecommand{\rootdir}{..}

\begin{document}
\subsection{McNally 1984}
In \cite{mcnally1984dynamic} a DDRC is introduced and is constructed as follows. The input signal is split into two parts. The first enters the side-chain where the to be applied gain is calculated, the second is delayed with the look-ahead and enters the amplifier where the signal is multiplied with the gain calculated in the side-chain. This is illustrated in Fig ~\ref{fig:block_mcnally_theory1}. %================================
\begin{figure}
\centerline{\subfile{\rootdir/figures/block_mcnally_theory}}
\caption{Block diagram of the DDRC in \cite{mcnally1984dynamic}}
\label{fig:block_mcnally_theory1}
\end{figure}
%================================

The sidechain takes the signal as input, $x_n$, and outputs the gain, $g_n$ and is composed of the following components:%================================
\begin{itemize}
\item{Peak/RMS detector}
\item{Gain computer}
\item{Adaptive filtering/gain smoothing}
\end{itemize}
%================================
In \cite{mcnally1984dynamic} two level detection methods are discussed based on digital first order infinite impulse response, IIR, filters.
\subsubsection{RMS Detector}
The difference equation for the first order IIR filter used in the rms detector in \cite{mcnally1984dynamic} is
%================================
\begin{align*}
f_n^2 = \alpha' f_{n-1}^2+ (1-\alpha') x_n^2
\end{align*}
%================================
where $x_n$ is the input signal and $\alpha$ is the filter coefficient. Assuming a step as input, the output is
%================================
\begin{align*}
f_0^2 &= \alpha' \cdot 0 + (1-\alpha')x_0^2 = 0 \\
f_1^2 &= \alpha' \cdot f_0^2 + (1-\alpha')x_1^2 = 1-\alpha'\\
f_2^2 &= \alpha' f_1^2+ (1-\alpha')x_2^2 = 1-\alpha'^2 \\
f_n^2 &= 1-\alpha'^{n}
\end{align*}
%================================
With $n = \tau f_s$ and the definition of the time constants, \eqref{eq:time_const}, the relation between the filter coefficient $\alpha$ and the time constant, here $\tau = \tau_a = \tau_r$, can be derived as
%================================
\begin{align*}
1-e^{-1} &= 1-\alpha'^{\tau f_s} \Longrightarrow \\
\alpha' &= e^{-1/\tau f_s}.
\end{align*}
%================================
It is however common to define the filter coefficient as
\begin{align}
\alpha &\equiv 1-e^{-1/\tau f_s}.
\end{align}
which corresponds to the difference equation
%================================
\begin{align}
f_n^2 = \alpha x_n^2 + (1-\alpha) f_{n-1}^2
\end{align}
%================================
This definition of $\alpha$ will be used hereafter.

A block representation of the rms detector can be seen in Fig ~\ref{fig:block_mcnally_theory_rms}. The input is, as mentioned, squared and passed through the filter. The logarithm of the output of the filter is taken and the squaring operation is compensated for by division by 2 in the logarithmic domain.
%================================
\begin{figure}
\centerline{\subfile{\rootdir/figures/block_mcnally_theory_rms}}
\caption{Block diagram of the RMS level detector in \cite{mcnally1984dynamic}}
\label{fig:block_mcnally_theory_rms}
\end{figure}
%================================
\subsubsection{Peak Detector}
In order to get a fast peak detection with a slower release time a detector with two time constants is used. The difference equation for the peak detector in \cite{mcnally1984dynamic} is
%================================
\begin{equation}
f_n = \begin{cases}
    \alpha_{a} f_n + (1- (\alpha_{a} + \alpha_{r})) f_{n-1}  	& x_n \geq f_{n-1} \\
    (1-\alpha_{r}) f_{n-1} 								& x_n < f_{n-1}
\end{cases}
\end{equation}
%================================
and its corresponding block representation can be seen in Fig ~\ref{}. A couple of things are worth to note. The release time constant is leaked into the attack phase, which introduces a static error, as shown in \cite{giannoulis2012}. However, as mentioned in \cite{mcnally1984dynamic}, the attack time is assumed to be $\sim 10$ ms and the release time   $\sim 100$ ms thus making $\alpha_r << \alpha_a$. \todo{How refer to ourselves?}We believe the peak detector was constructed this way to optimize performance, since at the time computational power was limited. In a modern context, the peak detector can easily be modified with a simple \emph{if}-statement, removing the "leakage" of $\alpha_r$ into the attack phase. For this implementation, see section \todo{future reference to Reiss}\ref{}.
%================================
\begin{figure}
\centerline{\subfile{\rootdir/figures/block_mcnally_theory_peak}}
\caption{Block diagram of the peak level detector in \cite{mcnally1984dynamic}}
\label{fig:block_mcnally_theory_rms}
\end{figure}
%================================
\subsection{Effect of Non-Linear Operations}
The non-linear operations carried out in the rms and peak detector introduces higher harmonics. A simple example of a pure sinusoidal signal illustrates this for the squaring operation
%================================
\begin{align}
\sin^2{\omega t} = \frac{1}{2}(1-\cos{2\omega t}) 
\end{align}
%================================
showing that only the second harmonic is generated. This can be eliminated by the lowpass filter of the rms detector described above. A typical time constant of 50 ms would result in a measurement error $< 0.1$ dB \cite{mcnally1984dynamic}.

For the magnitude operation carried out in the peak detector, the Fourier expansion of a sinusoidal input yields
%================================
\begin{align}
\left |\sin{\omega t}\right | = \frac{2}{\pi}\left (1 - \sum_{n=1}^{\infty}\frac{4}{4n^2-1} n\omega t \right) 
\end{align}
%================================
introducing an infinite number of even harmonics. A problem arises if the generated harmonic is close to the sample frequency, $f_s$ it would be aliased into zero frequency creating a static measurement error. This cannot be removed by a lowpass filter. Furthermore, the non-linear filter of the peak detector introduces distortion as well, but to a much lesser degree\cite{mcnally1984dynamic} and is not analysed further. Higher harmonics would also contribute to the error. 
%================================
\subsubsection{Gain Computer}
The gain computer calculates the gain reduction, $g_n$, to be applied with respect to the static characteristics of the compressor. These are described in section \ref{theory_definitions} and Fig ~\ref{fig:typical_static_detailed}.
\subsubsection{Adaptive Lowpass Filter}
To reduce the distortion described above a first order lowpass filter is added at the end of the sidechain in the linear domain. Here is where the attack and release parameters are introduced. A block diagram of the filter can be seen in ~\ref{fig:block_mcnally_theory_adap_filter}. The difference equation of the filter is
\begin{equation}
g_n = \begin{cases}
    \alpha_{a} g'_n + (1-\alpha_{a}) g_{n-1} 	& g'_n > g_{n-1} \\
    \alpha_{r} g'_n + (1-\alpha_{r}) g_{n-1} 	& g'_n \leq g_{n-1}
\end{cases}
\label{eq:mcnally_gain_smoothing}
\end{equation}
The block diagrams in \cite{mcnally1984dynamic}\cite{dafx02}\cite{dagx11}\cite{zolzer1997digital}\cite{zolzer1997digital}\cite{zolzer2008digital} all suggest that the branching condition is determined by $g'_n > g'_{n-1}$, that is, by comparing the current and previous \emph{input} samples instead of comparing the current input with the previous \emph{output}. However, when described in a code snippet appearing for the first time in \cite{dafx11}, the branching condition of equation \eqref{eq:mcnally_gain_smoothing} is actually used. In \cite{bitzer2006parameter} the branching condition of equation \eqref{eq:mcnally_gain_smoothing} is used while referring to \cite{mcnally1984dynamic}\cite{dafx02}.
%================================
\begin{figure}
\centerline{\subfile{\rootdir/figures/block_mcnally_theory_adap_filter}}
\caption{Block diagram of the adaptive filter in \cite{mcnally1984dynamic}}
\label{fig:block_mcnally_theory_rms}
\end{figure}
%================================
\end{document}}