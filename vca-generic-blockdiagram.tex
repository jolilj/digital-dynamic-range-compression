\tikzset{%
  amp/.style	= {draw, regular polygon, regular polygon sides=3,
	shape border rotate=-90, node distance = 25mm},
  input/.style	= {coordinate}, % Input
  output/.style	= {coordinate} % Output
}

\begin{tikzpicture}[auto, thick, node distance=2cm, >=triangle 45]
\draw
	% Drawing the blocks of first filter :
	node at (0,0)[right=-3mm]{\Large \textopenbullet}
	node [input, name=input1] {} 

	node [amp, right of=input1, name=vca1, ] {VCA}
	node [output, right of=vca1, name=output1, xshift=10mm]{}
	
	node [below of=vca1, name=cv1, yshift=-5mm]{\Large \textopenbullet}
	;
    % Joining blocks. 
    % Commands \draw with options like [->] must be written individually
	\draw[->](input1) -- node {\it{in}}(vca1);
 	\draw[->](vca1) -- node {\it{out}} (output1);
	\draw[->](cv1) -- node [label={[xshift=5mm, yshift=-5mm]\it{cv}}]{} (vca1);

\end{tikzpicture}